
\chapter{Numeric algorithms}
\Label{cha:numeric}

The algorithms that we considered so far only \emph{compared}, \emph{read} or
\emph{copied} values in sequences.
In this chapter, we consider so-called \emph{numeric} algorithms of the 
\cxx Standard Library \cite[\S 29.8]{cxx-17-draft} that use arithmetic
operations on \valuetype to
combine the elements of sequences.

\begin{listing}[hbt]
\begin{center}
\begin{minipage}{0.5\textwidth}
\begin{lstlisting}[style=acsl-block]
    #define VALUE_TYPE_MAX  INT_MAX 
    #define VALUE_TYPE_MIN  INT_MIN
\end{lstlisting}
\end{minipage}
\end{center}
\vspace*{-0.5cm}
\caption{\Label{lst:value-type-limits}Limits of \valuetype}
\end{listing}

In order to refer to potential arithmetic overflows we introduce the
two constants shown in Listing~\ref{lst:value-type-limits}
which refer to the numeric limits of \valuetype 
(see also \S\ref{sec:types}).

We consider the following algorithms.

\begin{itemize}

\item \iotai 
writes sequentially increasing values into a range
(\S\ref{sec:iotai})
 
\item \accumulate 
computes the sum of the elements in a range
(\S\ref{sec:accumulate})

\item \innerproduct 
computes the inner product of two ranges
(\S\ref{sec:innerproduct})

\item \partialsum 
computes the sequence of partial sums of a range
(\S\ref{sec:partialsum})

\item \adjacentdifference 
computes the differences of adjacent elements in a range 
(\S\ref{sec:adjacentdifference})

\item
Finally, in \S\ref{sec:partialsuminv} we show that under
appropriate preconditions the algorithms \partialsum and
\adjacentdifference are inverse to each other.

\end{itemize}

The formal specifications of these algorithms raise new questions.
In particular, we now have to deal with arithmetic overflows in \valuetype.

\clearpage


\section{The \iotai algorithm}
\Label{sec:iotai}

The \iotai algorithm in the \cxx Standard Library \cite[\S 29.8.12]{cxx-17-draft} assigns sequentially
increasing values to a range,
where the initial value is user-defined.
%
Our version of the original signature
reads:

\begin{lstlisting}[style=acsl-block]

  void iota(value_type* a, size_type n, value_type v);
\end{lstlisting} 


Starting at \inl{v}, the function assigns consecutive
integers to the elements of the range \inl{a}.
When specifying \iotai we must be careful to deal with possible overflows of the argument~\inl{v}.


\subsection{Formal specification of \iotai}

The specification of \iotai relies on the logic function \logicref{IotaGenerate}
that is defined in the following listing.

\input{Listings/IotaGenerate.acsl.tex}

The specification of \iotai is shown in the following listing.
It uses the logic function \logicref{IotaGenerate} in order to express the
postcondition~\inl{increment}.

\input{Listings/iota.h.tex}

The specification of \iotai refers to \inl{VALUE_TYPE_MAX} which is
the maximum value of the underlying integer type (see Listing~\ref{lst:value-type-limits}).
In order to avoid integer overflows 
the sum  \inl{v+n} must not be greater than the constant \inl{VALUE_TYPE_MAX}.

\clearpage

\subsection{Implementation of \iotai}

The following listing shows an implementation of the \iotai function.

\input{Listings/iota.c.tex}

The loop invariant \inl{increment} describes that in each iteration of the loop the current 
value~\inl{v} is equal to the sum of the value \inl{v} in state of function
entry and the loop index \inl{i}.
We have to refer here to \inl{\\at(v,Pre)} which is the value on entering \iotai.

\clearpage



\section{The \accumulate algorithm}
\Label{sec:accumulate}

The \accumulate algorithm in the \cxx Standard Library \cite[\S 29.8.2]{cxx-17-draft} computes
the sum of an given initial value and the elements in a range.
%
Our version of the original signature
reads:

\begin{lstlisting}[style=acsl-block]

  value_type
  accumulate(const value_type* a, size_type n, value_type init);
\end{lstlisting} 

The result of \accumulate shall equal the value
$\displaystyle{ \mathtt{init} + \sum_{i = 0}^{\mathtt{n}-1} \mathtt{a}[i] }$.
This implies that \accumulate will return \inl{init} for an empty range.

%\clearpage

\subsection{The logic function \Accumulate}

As in the case of \specref{counti}  we specify \accumulate by first defining the
\emph{logic function} \logicref{Accumulate} that formally defines
the summation of elements in an array.

\input{Listings/Accumulate.acsl.tex}

With this definition the following equation holds for $n \geq 0$
\begin{align}
\Label{eq:accumulate}
    \mathtt{Accumulate}(\mathtt{a}, \mathtt{n}, \mathtt{init})
    &= \mathtt{init} + \sum_{i = 0}^{\mathtt{n-1}} \mathtt{a}[i]
\end{align}

The predicate \logicref{AccumulateBounds} that we will subsequently use
in order to compactly express requirements that exclude numeric
overflows while accumulating value.
This predicate states that  for $0 \leq i < n$ the \emph{partial sums} 
%
\begin{gather}
\Label{eq:accumulate1}
\mathtt{init} + \sum_{k = 0}^{\mathtt{i}} \mathtt{a}[k]
\end{gather}
%
do not overflow.
If one of them did, one couldn't guarantee that the result of \isoc implementation
of \accumulate equals the mathematical description of \Accumulate.

%\clearpage

\subsection{\AccumulateDefault ---a variant of \Accumulate}

The following listing shows another version of \logicref{Accumulate},
called \logicref{AccumulateDefault}.

\input{Listings/AccumulateDefault.acsl.tex}

The function \AccumulateDefault uses~\inl{a[0]} as default value of \inl{init}.
Thus, for \AccumulateDefault we have

\begin{align}
\Label{eq:accumulate-default}
    \mathtt{AccumulateDefault}(\mathtt{a}, \mathtt{n})
    &= \sum_{i = 0}^{\mathtt{n-1}} \mathtt{a}[i]
\end{align}
We will use this version for the specification of the algorithm \specref{partialsum}.

This listing also includes additional properties of observable
\AccumulateDefault behavior, here given as a lemmas.
It also contains the predicate \logicref{AccumulateDefaultBounds}
with corresponding numeric limits for the predicate~\AccumulateDefault.

%\clearpage

\subsection{Formal specification of \accumulate}

Using the logic function \Accumulate and the predicate \AccumulateBounds,
the specification of \accumulate is then as simple
as shown in the following listing.

\input{Listings/accumulate.h.tex}

\clearpage

\subsection{Implementation of \accumulate}

The following listing shows an implementation of the
\accumulate function with corresponding loop annotations.

\input{Listings/accumulate.c.tex}

Note that loop invariant \inl{partial} claims that in the $i$-th iteration step \inl{result}
equals the accumulated value of Equation~\eqref{eq:accumulate1}.
This depends on the property \inl{bounds} of \specref{accumulate} which expresses that
there is no numeric overflow when updating the variable \inl{init}.

\clearpage


\section{The \innerproduct algorithm}
\Label{sec:innerproduct}

The \innerproduct algorithm in the \cxx Standard Library \cite[\S 29.8.4]{cxx-17-draft} computes
the \emph{inner product}\footnote{
  Also referred to as \emph{dot product}, see \url{http://en.wikipedia.org/wiki/Dot_product}
}
of two ranges.
%
Our version of the original signature
reads:

\begin{lstlisting}[style=acsl-block]

  value_type
  inner_product(const value_type* a, const value_type* b,
                size_type n, value_type init);
\end{lstlisting} 

The result of \innerproduct equals the value
\begin{gather*}
\mathtt{init} + \sum_{i = 0}^{\mathtt{n}-1} \mathtt{a}[i] \cdot \mathtt{b}[i]
\end{gather*}
thus, \innerproduct will return \inl{init} for empty ranges.

%\clearpage

\subsection{The logic function \InnerProduct}

As in the case of \specref{accumulate} we specify \innerproduct
by defining in the following listing the logic function \InnerProduct
that formally expresses the summation of the element-wise product of two arrays.

Predicate \logicref{ProductBounds} expresses that for $0 \leq i < n$ the products 
%
\begin{align}
\Label{eq:innerproduct1}
\mathtt{a}[i] \cdot \mathtt{b}[i] 
\end{align}
%
do not overflow. 
Predicate \logicref{InnerProductBounds}, on the other hand, states that for $0 \leq i < n$
the following sums do not overflow.
cc%
\begin{align}
\Label{eq:innerproduct2}
\mathtt{init} + \sum_{k = 0}^{\mathtt{i}} \mathtt{a}[k] \cdot \mathtt{b}[k]
\end{align}

Otherwise, one cannot guarantee that the result of our implementation
of \implref{innerproduct} equals the mathematical description of \InnerProduct.
Finally, Lemma \logicref{InnerProductUnchanged} states that the result of the \InnerProduct only
depends on the values of \inl{a[0..n-1]} and \inl{b[0..n-1]}.

\input{Listings/InnerProduct.acsl.tex}

\subsection{Formal specification of \innerproduct}

Using the logic function \logicref{InnerProduct}, we specify \innerproduct as shown 
in the following listing.
Note that we needn't require that \inl{a} and \inl{b} are separated.

\input{Listings/inner_product.h.tex}

\clearpage

\subsection{Implementation of \innerproduct}

The following listing shows an implementation of \innerproduct
with corresponding loop annotations.

\input{Listings/inner_product.c.tex}

Note that the loop invariant \inl{inner} claims
that in the $i$-th iteration step the current value of \inl{init}
equals the accumulated value of Equation~\eqref{eq:innerproduct2}.
This depends of course on the properties \inl{bounds} in the contract
of \specref{innerproduct}, which express that there is no arithmetic overflow
when computing the updates of the variable \inl{init}.

\clearpage



\section{The \partialsum algorithm}
\Label{sec:partialsum}

The \partialsum algorithm in the \cxx Standard Library \cite[\S 29.8.6]{cxx-17-draft} computes
the sum of a given initial value and the elements in a range.
%
Our version of the original signature
reads:

\begin{lstlisting}[style=acsl-block]

  size_type
  partial_sum(const value_type* a, size_type n, value_type* b);
\end{lstlisting} 

After executing the function \partialsum the array \inl{b[0..n-1]} holds the following values
\begin{align}
\Label{eq:partialsum}
   \mathtt{b}[i] &= \sum_{k = 0}^{\mathtt{i}} \mathtt{a}[k]
\end{align}
for $0 \leq i < n$.
%
Equations~\eqref{eq:partialsum} and~\eqref{eq:accumulate-default}
suggest that we define in the following listing the \acsl predicate \PartialSum
by using the logic function \logicref{AccumulateDefault}.

\input{Listings/PartialSum.acsl.tex}

\clearpage

\subsection{Formal specification of \partialsum}

The specification of \specref{partialsum} demands that the arrays
\inl{a[0..n-1]} and \inl{b[0..n-1]} 
are separated, that is, they do not overlap.
Note that is a stricter requirement than in the case of the original
\cxx version of \partialsum, which allows that~\inl{a} equals~\inl{b},
thus allowing the computation of partial sums \emph{in place}.

\input{Listings/partial_sum.h.tex}

\clearpage

\subsection{Implementation of \partialsum}

The following listing shows an implementation of \partialsum with corresponding loop annotations.

\input{Listings/partial_sum.c.tex}

\clearpage



\section{The \adjacentdifference algorithm}
\Label{sec:adjacentdifference}

The \adjacentdifference algorithm in the \cxx Standard Library \cite[\S 29.8.11]{cxx-17-draft}
computes the differences of adjacent elements in a range.
%
Our version of the original signature reads:

\begin{lstlisting}[style=acsl-block]

size_type
adjacent_difference(const value_type* a, size_type n, value_type* b);
\end{lstlisting} 

After executing the function \adjacentdifference the array \inl{b[0..n-1]} holds the following values
\begin{align}
   \mathtt{b}[0] &= \mathtt{a}[0] \nonumber \\
   \mathtt{b}[1] &= \mathtt{a}[1] - \mathtt{a}[0] \nonumber \\
                 &\vdotswithin{=} \nonumber \\
   \mathtt{b}[n-1] &= \mathtt{a}[n-1] - \mathtt{a}[n-2] \nonumber \\
\end{align}

\subsection{The predicate \AdjacentDifference}

We start with the definition of the logic function \Difference whose definition
is shown in the following listing.

\input{Listings/Difference.acsl.tex}

\clearpage 

Building on top of \Difference we now introduce the predicate \AdjacentDifference.
We also provide the predicate \AdjacentDifferenceBounds
that captures conditions that prevent numeric overflows
while computing differences of the form \inl{a[i] - a[i-1]}.

\input{Listings/AdjacentDifference.acsl.tex}

Lemmas \logicref{AdjacentDifferenceStep} and \logicref{AdjacentDifferenceSection}
will help us later in the verification of \implref{adjacentdifferenceinv}.

\clearpage

\subsection{Formal specification of \adjacentdifference}

Using the predicates \logicref{AdjacentDifference} and \logicref{AdjacentDifferenceBounds}
we can provide in the following listing a concise formal specification of \adjacentdifference.
As in the case of the specification of \specref{partialsum}
we require that the arrays \inl{a[0..n-1]} and \inl{b[0..n-1]} 
are separated.

\input{Listings/adjacent_difference.h.tex}

\clearpage

\subsection{Implementation of \adjacentdifference}

The following listing shows an implementation of \adjacentdifference
with corresponding loop annotations.
In order to achieve the verification of the loop invariant \inl{difference} we 
rely on
\begin{itemize}
\item the assertions \inl{bound} and \inl{difference}
\item the lemmas \logicref{AdjacentDifferenceStep} and \logicref{AdjacentDifferenceSection}
\item a statement contract with the two postconditions labeled as \inl{step}
\end{itemize}

\input{Listings/adjacent_difference.c.tex}

\clearpage



\section{Inverting \partialsum and \adjacentdifference}
\Label{sec:partialsuminv}
\Label{sec:adjacentdifferenceinv}


In this section we show that under appropriate preconditions
the algorithms \partialsum and\\
\adjacentdifference are inverse to each other.

\subsection{Inverting \partialsum}

Let \inl{a[0..n-1]} and \inl{b[0..n-1]} be the respective input and output 
of \partialsum.
We have in other words
\begin{align*}
   \mathtt{b}[0] &= \mathtt{a}[0] \\
   \mathtt{b}[1] &= \mathtt{a}[0] + \mathtt{a}[1] \\
                 &\vdotswithin{=} \\
   \mathtt{b}[n-1]  &= \mathtt{a}[0] + \mathtt{a}[1] + \ldots + \mathtt{a}[n-1] \\
\end{align*}

If we apply now the algorithm \adjacentdifference to \inl{b[0..n-1]}, then
we find for its output \inl{a'[0..n-1]}
\begin{alignat*}{3}
   \mathtt{a'}[0] &= \mathtt{b}[0]                        &\quad &=\quad \mathtt{a}[0] \\
   \mathtt{a'}[1] &= \mathtt{b}[1] - \mathtt{b}[0]        &\quad &=\quad \mathtt{a}[1] \\
                 &\vdotswithin{=}  \\
   \mathtt{a'}[n-1] &= \mathtt{b}[n-1] - \mathtt{b}[n-2]  &\quad &=\quad \mathtt{a}[n-1]
\end{alignat*}


Before we start show the \acsl lemmas of our claim, we present
the predicate \logicref{DefaultBounds} in order to express that the values
in the input (and output!) array~\inl{a[0..n-1]} do not overflow.

\input{Listings/DefaultBounds.acsl.tex}

Lemma \PartialSumInverse from the following listing
expresses as \acsl lemmas
that the algorithms \partialsum and \adjacentdifference are inverse to each other.

\input{Listings/NumericInverse.acsl.tex}

%\clearpage

The following listing now shows \isoc function \partialsuminv
(both the contract and the implementation).
This function calls first \partialsum and then \adjacentdifference.

\input{Listings/partial_sum_inv.c.tex}

The contract of \partialsuminv formulates preconditions that shall guarantee
that during the computation neither arithmetic overflows (property~\inl{bounds})
nor unintended aliasing of arrays (property~\inl{sep}) occur.
Under these precondition, \framac shall verify
that the final call to \specref{adjacentdifference} just restores the original contents
of~\inl{a[0..n-1]} that we supplied for the initial call to \specref{partialsum}.

\clearpage

\subsection{Inverting \adjacentdifference}

After executing the function \specref{adjacentdifference} on
the input array \inl{a[0..n-1]} the output array \inl{b[0..n-1]}
holds the following values
\begin{align*}
   \mathtt{b}[0] &= \mathtt{a}[0] \\
   \mathtt{b}[1] &= \mathtt{a}[1] - \mathtt{a}[0] \\
                 &\vdotswithin{=} \\
   \mathtt{b}[n-1] &= \mathtt{a}[n-1] - \mathtt{a}[n-2] \\
\end{align*}

If we call now \partialsum with the array \inl{b[0..n-1]}
as input, then we obtain for its output \inl{a'[0..n-1]}
\begin{alignat*}{3}
   \mathtt{a'}[0] &= \mathtt{b}[0]
                  &\quad &=\quad \mathtt{a}[0] \\
   \mathtt{a'}[1] &= \mathtt{b}[0] + \mathtt{b}[1]
                  &\quad &=\quad \mathtt{a}[1] \\
                  &\vdotswithin{=} \\
   \mathtt{a'}[n-1]  &=\ \mathtt{b}[0] + \mathtt{b}[1] + \ldots + \mathtt{b}[n-1]
                     &\quad &=\quad \mathtt{a}[n-1] 
\end{alignat*}
%
which means that applying \specref{partialsum} on the output of
\adjacentdifference produces the original input.
Lemma \logicref{AdjacentDifferenceInverse} expresses this property as a lemma.

The function \implref{adjacentdifferenceinv} first calls \adjacentdifference and then\\
\partialsum.
The contract of this function formulates preconditions that shall guarantee
that during the computation neither arithmetic overflows (property~\inl{bound})
nor unintended aliasing of arrays (property~\inl{sep}) occur.
In order to improve the automatic verification of 
\adjacentdifferenceinv we also use lemma \logicref{UnchangedTransitive}.
Lemma \logicref{AdjacentDifferenceInverseBounds} simplifies
the verification of the precondition \inl{bounds} of \partialsum.

\input{Listings/adjacent_difference_inv.c.tex}



