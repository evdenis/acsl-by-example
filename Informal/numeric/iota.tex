
\section{The \iotai algorithm}
\Label{sec:iotai}

The \iotai algorithm in the \cxx Standard Library \cite[\S 29.8.12]{cxx-17-draft} assigns sequentially
increasing values to a range,
where the initial value is user-defined.
%
Our version of the original signature
reads:

\begin{lstlisting}[style=acsl-block]

  void iota(value_type* a, size_type n, value_type v);
\end{lstlisting} 


Starting at \inl{v}, the function assigns consecutive
integers to the elements of the range \inl{a}.
When specifying \iotai we must be careful to deal with possible overflows of the argument~\inl{v}.


\subsection{Formal specification of \iotai}

The specification of \iotai relies on the logic function \logicref{IotaGenerate}
that is defined in the following listing.

\input{Listings/IotaGenerate.acsl.tex}

The specification of \iotai is shown in the following listing.
It uses the logic function \logicref{IotaGenerate} in order to express the
postcondition~\inl{increment}.

\input{Listings/iota.h.tex}

The specification of \iotai refers to \inl{VALUE_TYPE_MAX} which is
the maximum value of the underlying integer type (see Listing~\ref{lst:value-type-limits}).
In order to avoid integer overflows 
the sum  \inl{v+n} must not be greater than the constant \inl{VALUE_TYPE_MAX}.

\clearpage

\subsection{Implementation of \iotai}

The following listing shows an implementation of the \iotai function.

\input{Listings/iota.c.tex}

The loop invariant \inl{increment} describes that in each iteration of the loop the current 
value~\inl{v} is equal to the sum of the value \inl{v} in state of function
entry and the loop index \inl{i}.
We have to refer here to \inl{\\at(v,Pre)} which is the value on entering \iotai.

\clearpage

