
\section{The \shuffle algorithm}
\Label{sec:shuffle}

The \shuffle  algorithm in the \cxx Standard Library \cite[\S 28.6.13]{cxx-17-draft}
randomly rearranges the elements of a given range, that is,
it randomly picks one of its possible orderings.
For our purposes we have modified
the generic implementation
to that of a range of type \valuetype.
The signature now reads:

\begin{lstlisting}[style=acsl-block]

  void shuffle(value_type* a, size_type n, unsigned short* rand);
\end{lstlisting}

The argument \inl{rand} holds the state of a simple random number generator
that is used in the implementation of \shuffle.

Figure~\ref{fig:shuffle} illustrates an example run of \shuffle.
In this figure, the values 1, 2, 3, and 4 occur 
twice, once, once, and three times, respectively, both before and
after the \shuffle run.
This expresses that the range has been reordered.

\begin{figure}[hbt]
\centering
\includegraphics[width=0.60\textwidth]{Figures/shuffle.pdf}
\caption{\Label{fig:shuffle} Effects of \shuffle}
\end{figure}

\clearpage

\subsection{The predicate \MultisetReorder}
\Label{sec:multisetunchanged}

The \shuffle algorithm is the first example in this document
where we have to specify a \emph{rearrangement} or \emph{reordering}
of the elements of a given range.
We say that an array has been reordered between two states
if the number of each element in the array remains unchanged.
In other words, reordering leaves the \emph{multiset}\footnote{
 See \url{http://en.wikipedia.org/wiki/Multiset}
}
of elements in the range unchanged.

We use the predicate \logicref{MultisetReorder} 
to formally describe this property.
This predicate, which is given in two overloaded versions,
relies on the logic function \logicref{Count}.
We list here several lemma with basic properties of \MultisetReorder.
We will use these lemmas during the verification of various algorithms.

\input{Listings/MultisetReorder.acsl.tex}

\clearpage

\subsection{Formal specification of \shuffle}

In the specification of the \specref{shuffle} algorithm we demand that the range \inl{a[0..n-1]}
is valid for reading and writing.
We use the predicate \logicref{MultisetReorder} to express that the
contents of \inl{a[0..n-1]} is just permuted, i.e., the number of occurrences
of each of its members remains unchanged. 
The array \inl{rand} contains a seed for the random number generator used to randomize the
shuffle.
By specifying that the function assigns to \inl{rand}
we capture that the function may return a different permutation every time.

Note that our specification only states that the resulting range is
a reordering of the input range; nothing more and nothing less.
Ideally, we would also specify that sequence of reorderings
obtained by repeated
calls of \shuffle is required to be random.
The informal specification\cite[\S 28.6.13]{cxx-17-draft} of \shuffle states that
\emph{that each possible permutation of those elements has 
equal probability of appearance}.
However, \acsl does currently not support the specification of
temporal properties related to repeated call results.

\input{Listings/shuffle.h.tex}

More generally speaking, it is not trivial to capture
the notion of randomness in a mathematically precise way.
As a typical example, we refer to a paper\cite[p.6--8]{Moses.Oakford.1963},
which just gives four statistical tests indicating the randomness of the
permutations computed with their algorithm.
From a theoretical point of view, a sequence of permutations
can be called ``random'' if its Kolmogorov complexity exceeds
a certain measure, however, this property is undecidable\cite{Li.Vitanyi.1997}.

\clearpage

\subsection{Implementation of \shuffle}

The following listing shows our implementation of the function \implref{shuffle}.
It repeatedly calls the function \specref{swap} to \emph{transpose}
(randomly) selected elements.
For details of out source of randomness we refer to the function \specref{randomnumber}.

\input{Listings/shuffle.c.tex}

The loop invariants \inl{reorder} and \inl{unchanged} of \shuffle 
are necessary for the verification of the postcondition~\inl{reorder}:
in the \inl{i}th loop cycle, the subrange \inl{a[0..i-1]} has been
reordered, while the remaining subrange \inl{a[i..n-1]} is yet unchanged.
We also formulate several auxiliary assertions \inl{reorder} which use the
the predefined label \inl{LoopCurrent},
to guide the automatic verification the loop invariant \inl{reorder}.
Please not the empty \inl{else}-branch hat only contains an assertion \inl{reorder}.
We introduced this assertion to support the verification of the \inl{reorder} property.


\clearpage

In addition, we rely on the predicate \logicref{ArraySwap}
rather than the literal postcondition of \specref{swap}, since this leads to
to more concise annotations and better a performance of the automatic provers.

\input{Listings/ArraySwap.acsl.tex}

The lemma \logicref{MultisetSwapMiddle} states that swapping the elements
\inl{a[i]} and \inl{a[k]} is a particular
kind of reordering on the range \inl{a[i..k]}.

\input{Listings/MultisetSwap.acsl.tex}

\clearpage

