
\section{The \rotatei algorithm}
\Label{sec:rotatei}

The algorithm \rotatei is an \emph{in-place} variant of the algorithm \specref{rotatecopy}.
We have modified the generic specification of \rotatei \cite[\S 28.6.11]{cxx-17-draft}
such that it refers to a range of objects of \valuetype.
%
The signature now reads:

\begin{lstlisting}[style=acsl-block]

  size_type rotate(const value_type* a, size_type m, size_type n);
\end{lstlisting}


\subsection{Formal specification of \rotatei}

Figure~\ref{fig:rotate} shows informally the behavior of \rotatei.
The figure is of course very similar to the one for
\rotatecopy (see Figure~\ref{fig:rotatecopy}).
The notable difference is that \rotatei operates \emph{in place} of the array
\inl{a[0..n-1]}.

\begin{figure}[hbt]
\centering
\includegraphics[width=0.75\textwidth]{Figures/rotate.pdf}
\caption{\Label{fig:rotate} Effects of \rotatei}
\end{figure}

The specification of \rotatei is shown in the following listing.

\input{Listings/rotate.h.tex}

\subsection{Implementation of \rotatei}

The following listing shows an implementation of the \rotatei function
together with several \acsl annotations.
Actually, there are several ways to implement \rotatei.
We have chosen a particularly simple one that is derived from an implementation of
\inl{std::rotate} for \emph{bidirectional iterators} \cite[\S 27.2.6]{cxx-17-draft}
and which essentially consists of several calls to the algorithm~\specref{reverse}.

Note the statement contract of the final call of \specref{reverse}.
Here we use both the labels~\inl{Pre} and~\inl{Old} which refer
to the pre-states of~\reverse and the function \rotatei itself, respectively.

\input{Listings/rotate.c.tex}

\clearpage

