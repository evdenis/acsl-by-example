
\section{The \copyi algorithm}
\Label{sec:copyi}

The \copyi  algorithm in the \cxx Standard Library \cite[\S 28.6.1]{cxx-17-draft} implements
a duplication algorithm for general sequences.
For our purposes we have modified
the generic implementation
to that of a range of type \valuetype.
The signature now reads:

\begin{lstlisting}[style=acsl-block]

  void copy(const value_type* a, size_type n, value_type* b);
\end{lstlisting}

Informally, the function copies every element from the source range \inl{a[0..n-1]} to the
destination range~\inl{b[0..n-1]}, as shown in Figure~\ref{fig:copy}.

\begin{figure}[hbt]
\centering
\includegraphics[width=0.50\textwidth]{Figures/copy.pdf}
\caption{\Label{fig:copy} Effects of \copyi}
\end{figure}

\subsection{Formal specification of \copyi}

Figure~\ref{fig:copy} might suggest that the ranges \inl{a[0..n-1]} and \inl{b[0..n-1]}
must not overlap.
However, since the informal specification requires that elements are copied in the
order of increasing indices only a weaker condition is necessary.
To be more specific, it is required that the pointer~\inl{b} does not refer
to elements of \inl{a[0..n-1]} as shown in the example in Figure~\ref{fig:copy-overlap}.

\begin{figure}[hbt]
\centering
\includegraphics[width=0.60\textwidth]{Figures/copy-overlap.pdf}
\caption{\Label{fig:copy-overlap} Possible overlap of \copyi ranges}
\end{figure}

\FloatBarrier

The specification of \copyi is shown in the following listing.
The \copyi algorithm expects that the ranges \inl{a} and \inl{b} are valid for reading
and writing, respectively.
Note the precondition~\inl{sep} that expresses the previously discussed non-overlapping property.

\input{Listings/copy.h.tex}

Again, we can use the \logicref{Equal} predicate to express that the
array~\inl{a} equals~\inl{b} after \copyi has been called.
Nothing else must be altered.
To state this we use the \inl{assigns}-clause.

%\clearpage

\subsection{Implementation of \copyi}

The following listing shows an implementation of the \copyi function.

\input{Listings/copy.c.tex}

For the postcondition \equal to be true, we must ensure that for every index
\inl{i}, the value \inl{a[i]} must not yet have been changed before it is 
copied to \inl{b[i]}.
We express this by using the  \Unchanged predicate.\footnote{
Alternatively, this could also be expressed by changing the
\inl{loop assigns} clause to \inl{i, b[0..i-1]}; however,
\framac doesn't yet support \inl{loop assigns} clauses
containing the loop variable.
}

The \inl{assigns} clause ensures that nothing but the range \inl{b[0..n-1]}
and the loop variable \inl{i} is modified.
Keep in mind, however, that parts of the source range \inl{a[0..n-1]} might change
due to its potential overlap with the destination range.

\clearpage

