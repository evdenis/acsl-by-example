
\section{The \removecopyii algorithm (number of copied elements)}
\label{sec:removecopyii}

In this section we improve the contract of \specref{removecopy}
by formally specifying the number~\inl{\\result} of elements copied by \removecopy.

The number of copied elements equals of course the number of elements
in the input range \inl{a[0..n-1]} that are different from~\inl{v}.
One can formally describe this number by relying on the logic function~\logicref{Count}.

\begin{lstlisting}[style=acsl-block]

  logic integer
  CountNotEqual(value_type* a, integer n, value_type v) =  n - Count(a, n, v);
\end{lstlisting}

In fact, we have used this kind of definition in earlier version of this document.
We have found it, however, worthwhile to provide a separate definition of \CountNotEqual
and express the relationship with \Count as a lemma.
This definition is shown in the Listings~\ref{logic:CountNotEqual-1}
and~\ref{logic:CountNotEqual-2}.

\begin{logic}[hbt]
\begin{minipage}{\textwidth}
\lstinputlisting[linerange={1-34}, style=acsl-block, frame=single]{Source/CountNotEqual.acsl}
\end{minipage}
\caption{\Label{logic:CountNotEqual-1} The logic function \CountNotEqual (1)}
\input{Listings/CountNotEqual.acsl.labels.tex}
\input{Listings/CountNotEqual.acsl.index.tex}
\end{logic}


\clearpage

The  above mentioned relationship with \logicref{Count} is expressed as
lemma \logicref{CountNotEqualCount} in the following listing.

\begin{logic}[hbt]
\begin{minipage}{\textwidth}
\lstinputlisting[linerange={35-60}, style=acsl-block, frame=single]{Source/CountNotEqual.acsl}
\end{minipage}
\caption{\Label{logic:CountNotEqual-2} The logic function \CountNotEqual (2)}
\end{logic}

\FloatBarrier


\subsection{Formal specification of \removecopyii}

We extend our formal specification by using \logicref{CountNotEqual}
and add the new postcondition \inl{size}, which states that the returning value
of \removecopyii equals \CountNotEqual.
The following listing shows the formal specification of \specref{removecopyii}.

\input{Listings/remove_copy2.h.tex}

\subsection{Implementation of \removecopyii}

The following listing shows the implementation of our
extended of \removecopyii.
Here we added the loop invariant \inl{size} which corresponds to the 
postcondition in \specref{removecopyii}.
In order to ensure that the loop invariant \inl{size} can be verified
we have added the assertions \inl{size} and \inl{unchanged}.

\input{Listings/remove_copy2.c.tex}

While we now can precisely speak of the number of copied elements,
it is still not possible to say something about the exact relationship between
the elements of range~\inl{a[0..n-1]} and range~\inl{b[0..n-1]}.
We will address this question the contract of \specref{removecopyiii}.

\clearpage

