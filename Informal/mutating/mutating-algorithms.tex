
\chapter{Mutating algorithms} \Label{cha:mutating}

Let us now turn our attention to another class of algorithms,
viz.\ \emph{mutating} algorithms of the \cxx Standard Library \cite[\S
28.6]{cxx-17-draft}, i.e.,
algorithms that change one or more ranges.
%
In \framac, you can explicitly specify that, e.g., entries in an array
\inl{a} may be modified by a function
\inl{f},
by including the following \emph{assigns clause} into the 
\inl{f}'s specification:

\begin{lstlisting}[style=acsl-block]

     assigns a[0..length-1];
\end{lstlisting} %
The expression \inl{length-1} refers to the value of \inl{length}
when \inl{f} is entered, see 
\cite[\S2.3.2]{ACSLSpec}.
Below are the algorithms we will discuss in this chapter.

\begin{itemize}

\item
In order to allow for a finer control of which parts of an array,
we introduce in \S\ref{sec:unchanged} the auxiliary predicate \Unchanged.

\item \filli in \S\ref{sec:filli}
initializes each element of an array by a given fixed value.

\item \swap in \S\ref{sec:swap} exchanges two values.

\item \swapranges 
in \S\ref{sec:swapranges} exchanges the contents of the arrays of equal length, element
by element.
We use this example to present ``modular verification'',
as \swapranges reuses the verified properties of \swap.

\item \copyi 
in \S\ref{sec:copyi} 
copies a source array to a destination array.

\item \copybackward 
in \S\ref{sec:copybackward} also
copies a source array to a destination array. 
This version, however, uses another separation condition than \copyi.

\item \reversecopy and \reverse 
in \S\ref{sec:reversecopy} and~\S\ref{sec:reverse}, respectively,
reverse an array.
Whereas \reversecopy copies the result to a separate destination array, 
the \reverse algorithm works in place.

\item \rotatecopy 
in \S\ref{sec:rotatecopy}
rotates a source array by \inl{m} positions and copies the results to a
destination array.

\item \rotatei 
in \S\ref{sec:rotatei}
rotates \emph{in place} a source array by \inl{m} positions.

\item \replacecopy and \replace
in \S\ref{sec:replacecopy} and~\S\ref{sec:replace}, respectively,
substitute each occurrence of a value by a given new value.
Whereas \replacecopy copies the result to a separate array, 
the \replace algorithm works in place.

\item \removecopy and \remove in \S\ref{sec:removecopy}--\S\ref{sec:remove}
\emph{filter} all occurrences of a given value from an array.
Whereas \removecopy copies the result to a separate array, 
the \remove algorithm works in place.
Note that we provide altogether three versions of how to specify \removecopy.
This shall help the reader to understand that finding appropriate contracts
is an iterative process and that it is usually a good idea to \emph{not} strive
for a ``complete'' contract right from the beginning.

\item \shuffle in \S\ref{sec:shuffle} randomly reorders the elements of an array
thereby relying on the simple random number generator \randomnumber in 
\S\ref{sec:randomnumber}.

\end{itemize}

\clearpage


\section{The predicate \Unchanged}
\Label{sec:unchanged}

Many of the algorithms in this section iterate sequentially over one or several sequences.
For the verification of such algorithms it is often important to express that a section
of an array, or the complete array, have remained \emph{unchanged}.
As this cannot always be expressed by an \inl{assigns} clause,
we introduce in the following listing the overloaded predicate
\logicref{Unchanged}.
The expression \inl{Unchanged\{K,L\}(a,m,n)} is true if the range
\inl{a[m..n-1]} in state~\inl{K} is element-wise equal to that range in state~\inl{L}.

\input{Listings/Unchanged.acsl.tex}

In some situations we use the predicate \ArrayUpdate,
which relies on the predicate \Unchanged and the the logic function \logicref{At},
to concisely describe which parts of an array have changed or remained unchanged
when updating an individual array element.


\input{Listings/ArrayUpdate.acsl.tex}

\clearpage

In the following listing we show a few lemmas for \logicref{Unchanged}
that we need for the verification of various algorithms.

\input{Listings/UnchangedLemmas.acsl.tex}

\begin{itemize}
\item
Lemma~\logicref{UnchangedShrink} states that if the range~\inl{a[m..n-1]} does
not change when going from state~\inl{K} to state~\inl{L}, then~\inl{a[p..q-1]}
does not change either, provided the latter is a subrange of the
former, i.e.\ provided $0 \leq m \leq p \leq q \leq n$ holds.

\item
Lemma~\logicref{UnchangedExtend} expresses the simple fact that ``unchangedness'' is an inductive property.

\item
Lemma~\logicref{UnchangedShift} states how \Unchanged behaves under pointer additions.

\item
Lemmas~\logicref{UnchangedSymmetric} and~\logicref{UnchangedTransitive} express respectively
the symmetry and transitivity of \Unchanged with respect to program states.

\end{itemize}


\clearpage



\section{The \filli algorithm}
\Label{sec:filli}

The \filli  algorithm in the \cxx
Standard Library \cite[\S 28.6.6]{cxx-17-draft} initializes general
sequences with a particular value.
The signature of our modified variant reads:

\begin{lstlisting}[style=acsl-block]

  void fill(value_type* a, size_type n, value_type v);
\end{lstlisting}

\subsection{Formal specification of \filli}

The following listing shows the formal specification of \specref{filli}.
We can express the postcondition of \filli simply by using the overloaded
predicate \logicref{AllEqual}.

\input{Listings/fill.h.tex}

The \inl{assigns}-clauses formalize that \filli modifies only the
entries of the range \inl{a[0..n-1]}.
In general, when more than one \emph{assigns clause} appears
in a function's specification,
it is permitted to modify any of the referenced memory locations.
However, if no \emph{assigns clause} appears at all,
the function is free to modify any memory location, see
\cite[\S 2.3.2]{ACSLSpec}.
To forbid a function to do any modifications outside its
scope, a clause \inl{assigns \\nothing;}
must be used, as we practised
in the example specifications in Chapter~\ref{cha:non-mutating}.

\subsection{Implementation of \filli}

The implementation of \implref{filli} comes with the loop invariant
\inl{constant} expresses that for each iteration the array is
\emph{filled} with the value of \inl{v} up to the index \inl{i} of the iteration. 
Note that we use here again the predicate \logicref{AllEqual}.

\input{Listings/fill.c.tex}

%\clearpage



\section{The \swap algorithm}
\Label{sec:swap}

The \swap algorithm \cite[\S 28.6.3]{cxx-17-draft}
in the \cxx Standard Library exchanges the contents of
two variables.
Similarly, the \iterswap algorithm \cite[\S 28.6.3]{cxx-17-draft}
exchanges the contents referenced by two pointers.  
Since \isoc and hence \acsl, does not support an \inl{&} type constructor (``declarator''),
we will present an algorithm that processes pointers
and refer to it as \swap.


\subsection{Formal specification of \swap}

The contract of \specref{swap} is shown in the following listing.
The preconditions state that both pointer arguments of \swap must be dereferenceable.

\input{Listings/swap.h.tex}

Upon termination of \swap the entries must be mutually exchanged.
The expression \inl{\\old(*p)} refers to the value of \inl{*p}
before \swap has be called.
By default, a postcondition refers the values after the functions has been terminated.

\subsection{Implementation of \swap}

The following listing shows the straight-forward implementation of \implref{swap}.
No interspersed \acsl annotations are needed achieve a verification by \wpframac.

\input{Listings/swap.c.tex}

\clearpage



\section{The \swapranges algorithm}
\Label{sec:swapranges}

The \swapranges algorithm
in the \cxx Standard Library \cite[\S 28.6.3]{cxx-17-draft} exchanges the
contents of two expressed ranges element-wise.
%
After translating \cxx reference types and iterators to \isoc,
our version of the original signature reads:

\begin{lstlisting}[style=acsl-block]

  void swap_ranges(value_type* a, size_type n, value_type* b);
\end{lstlisting} 
%
We do not return a value since it would equal \inl{n}, anyway.


\subsection{Formal specification of \swapranges}

The following listing shows a specification for the \specref{swapranges} algorithm.

\input{Listings/swap_ranges.h.tex}

The \swapranges algorithm works correctly only if \inl{a} and \inl{b} do not overlap.
Because of that fact we use the clause \inl{sep} to
tell \framac that \inl{a} and \inl{b} must not overlap.

With the \inl{assigns}-clause we postulate
that the \swapranges algorithm alters the elements contained
in two distinct ranges, modifying the corresponding
elements and nothing else.

The postconditions of \swapranges specify that the content
of each element in its post-state must equal the pre-state
of its counterpart.
We can use the predicate \logicref{Equal} together with the
label \inl{Old} and \inl{Here} to express the postcondition of \swapranges.
In our specification, for example, we specify that the array \inl{a} in the memory
state that corresponds to the label \inl{Here} is equal
to the array~\inl{b} at the label \inl{Old}.
Since we are specifying a postcondition \inl{Here} refers to the post-state
of \swapranges whereas \inl{Old} refers to the pre-state.

\clearpage

\subsection{Implementation of \swapranges}

The implementation of \implref{swapranges} together with the necessary
loop annotations is shown in the following listing.
Unsurprisingly, we are repeatedly calling \specref{swap}.

\input{Listings/swap_ranges.c.tex}

For the postcondition \specref{swapranges} to hold, 
our loop invariants must ensure that at each iteration all of the
corresponding elements that have already been visited are swapped.

Note that there are two additional loop invariants which claim
that all the elements that have not visited yet equal their original values.
This annotation allows us to prove the postconditions of \swapranges
despite the fact that the loop assigns is coarser than it should be.
The predicate \logicref{Unchanged} is used to express this property.

\clearpage



\section{The \copyi algorithm}
\Label{sec:copyi}

The \copyi  algorithm in the \cxx Standard Library \cite[\S 28.6.1]{cxx-17-draft} implements
a duplication algorithm for general sequences.
For our purposes we have modified
the generic implementation
to that of a range of type \valuetype.
The signature now reads:

\begin{lstlisting}[style=acsl-block]

  void copy(const value_type* a, size_type n, value_type* b);
\end{lstlisting}

Informally, the function copies every element from the source range \inl{a[0..n-1]} to the
destination range~\inl{b[0..n-1]}, as shown in Figure~\ref{fig:copy}.

\begin{figure}[hbt]
\centering
\includegraphics[width=0.50\textwidth]{Figures/copy.pdf}
\caption{\Label{fig:copy} Effects of \copyi}
\end{figure}

\subsection{Formal specification of \copyi}

Figure~\ref{fig:copy} might suggest that the ranges \inl{a[0..n-1]} and \inl{b[0..n-1]}
must not overlap.
However, since the informal specification requires that elements are copied in the
order of increasing indices only a weaker condition is necessary.
To be more specific, it is required that the pointer~\inl{b} does not refer
to elements of \inl{a[0..n-1]} as shown in the example in Figure~\ref{fig:copy-overlap}.

\begin{figure}[hbt]
\centering
\includegraphics[width=0.60\textwidth]{Figures/copy-overlap.pdf}
\caption{\Label{fig:copy-overlap} Possible overlap of \copyi ranges}
\end{figure}

\FloatBarrier

The specification of \copyi is shown in the following listing.
The \copyi algorithm expects that the ranges \inl{a} and \inl{b} are valid for reading
and writing, respectively.
Note the precondition~\inl{sep} that expresses the previously discussed non-overlapping property.

\input{Listings/copy.h.tex}

Again, we can use the \logicref{Equal} predicate to express that the
array~\inl{a} equals~\inl{b} after \copyi has been called.
Nothing else must be altered.
To state this we use the \inl{assigns}-clause.

%\clearpage

\subsection{Implementation of \copyi}

The following listing shows an implementation of the \copyi function.

\input{Listings/copy.c.tex}

For the postcondition \equal to be true, we must ensure that for every index
\inl{i}, the value \inl{a[i]} must not yet have been changed before it is 
copied to \inl{b[i]}.
We express this by using the  \Unchanged predicate.\footnote{
Alternatively, this could also be expressed by changing the
\inl{loop assigns} clause to \inl{i, b[0..i-1]}; however,
\framac doesn't yet support \inl{loop assigns} clauses
containing the loop variable.
}

The \inl{assigns} clause ensures that nothing but the range \inl{b[0..n-1]}
and the loop variable \inl{i} is modified.
Keep in mind, however, that parts of the source range \inl{a[0..n-1]} might change
due to its potential overlap with the destination range.

\clearpage



\section{The \copybackward algorithm}
\Label{sec:copybackward}

The \copybackward  algorithm in the \cxx Standard Library \cite[\S 28.6.1]{cxx-17-draft} implements
another duplication algorithm for general sequences.
For our purposes we have modified
the generic implementation
to that of a range of type \valuetype.
The signature now reads:

\begin{lstlisting}[style=acsl-block]

  void copy_backward(const value_type* a, size_type n, value_type* b);
\end{lstlisting}

The main reason for the existence of \copybackward is to allow
copying when the start of the destination range \inl{a[0..n-1]} is contained
in the source range \inl{b[0..n-1]}.
In this case, \copyi can't be employed since its precondition
\inl{sep} is violated, as can be seen in the contract of \specref{copyi}.

The informal specification of \copybackward states that copying
starts at the end of the source range.
For this to work, however, the pointer \inl{b+n} must not be contained
in the source range.
Note that the order of operation (or procedure) calls cannot be
specified in \acsl.\footnote{
       The \textsf{Aoraï} specification language and the
       corresponding \framac plugin are provided to specify and verify
       temporal properties of code; however, they are beyond the scope
       of this tutorial.
}
A similar remark about order of operations tacitly applied
to earlier functions
as well, e.g.\ to \copyi, where the \cxx order was prescribed by confining
the signature to a \inl{ForwardIterator}.


Figure~\ref{fig:copybackward} gives an example where \copybackward, but \emph{not} \copyi,
can be applied.

\begin{figure}[hbt]
\centering
\includegraphics[width=0.60\textwidth]{Figures/copy_backward.pdf}
\caption{\Label{fig:copybackward} Possible overlap of \copybackward ranges}
\end{figure}

Note that in the original signature the argument~\inl{b} refers to one 
past the end of the destination range.
Here, however, it refers to its start.
The reason for this change is that in \cxx \copybackward is defined
for \emph{bidirectional iterators} which do not provide
random access operations such as adding or subtracting an index.
Since our \isoc version works on pointers we do not consider it
as necessary to use the one past the end pointer.

%\clearpage

\subsection{Formal specification of \copybackward}

The specification of \copybackward is shown in the following listing.
The \copybackward algorithm expects that the ranges \inl{a[0..n-1]} and \inl{b[0..n-1]}
are valid for reading and writing, respectively.
Precondition \inl{sep} formalizes the constraints on the overlap of
the source and destination ranges as discussed at the beginning of this section.

\input{Listings/copy_backward.h.tex}

The function \copybackward assigns the elements from the source
range \inl{a} to the destination range \inl{b}, modifying the memory of the
elements pointed to by \inl{b}.
Again, we can use the \logicref{Equal} predicate to express that the
array~\inl{a} equals~\inl{b} after \copybackward has been called.


\subsection{Implementation of \copybackward}

The following listing shows an implementation of the
\copybackward function.

\input{Listings/copy_backward.c.tex}

We have loop invariants similar to \copyi, stating the loop variable's range
(\inl{bound})
and the area that has already been copied in each cycle (\equal).

\clearpage



\section{The \reversecopy algorithm}
\Label{sec:reversecopy}

The \reversecopy
algorithm of the \cxx Standard Library \cite[\S 28.6.10]{cxx-17-draft} inverts the order of elements
in a sequence.
\reversecopy does not change the input sequence, and
copies its result to the output sequence.
For our purposes we have modified the generic implementation
to that of a range of type \valuetype.
The signature now reads:

\begin{lstlisting}[style=acsl-block]

  void reverse_copy(const value_type* a, size_type n, value_type* b);
\end{lstlisting}

%\subsection{The predicate \Reverse}

Informally, \reversecopy copies the elements from the array \inl{a} into
array \inl{b} such that the copy is a reverse of the original array. 
In order to concisely formalize these conditions we define in the following
listing the predicate \logicref{Reverse} (see also Figure~\ref{fig:Reverse}).

\input{Listings/Reverse.acsl.tex}

We also define several overloaded variants of \Reverse that 
provide default values for some of the parameters.
These overloaded versions enable us to write later more concise \acsl annotations.

\begin{figure}[hbt]
\centering
\includegraphics[width=0.60\textwidth]{Figures/reverse_logic.pdf}
\caption{\Label{fig:Reverse} Sketch of predicate~\Reverse}
\end{figure}

\FloatBarrier

\subsection{Formal specification of \reversecopy}

The specification of \specref{reversecopy} is shown in the following listing
We use the second version of predicate \logicref{Reverse}
in order to formulate the postcondition of \reversecopy.

\input{Listings/reverse_copy.h.tex}

\subsection{Implementation of \reversecopy}

The implementation of \implref{reversecopy} is shown in the following listing.
%
For the postcondition to be true, we must ensure that for
every element \inl{i}, the comparison \inl{b[i] == a[n-1-i]} holds.
This is formalized by the loop invariant~\inl{reverse} where we employ
the first version of \logicref{Reverse}.

\input{Listings/reverse_copy.c.tex}

\clearpage



\section{The \reverse algorithm}
\Label{sec:reverse}

The \reverse algorithm of the \cxx Standard Library
\cite[\S 28.6.10]{cxx-17-draft} inverts the order of elements \emph{within} a sequence.
The signature of our version of \reverse reads.

\begin{lstlisting}[style=acsl-block]

  void reverse(value_type* a, size_type n);
\end{lstlisting}


\subsection{Formal specification of \reverse}

The specification for the \specref{reverse} function is shown in the following listing.

\input{Listings/reverse.h.tex}

\subsection{Implementation of \reverse}

Since the implementation of \implref{reverse} operates \emph{in place}
we use \specref{swap} in order to exchange the elements of the first half
of the array with the corresponding elements of the second half.
We reuse the predicates \logicref{Reverse} and \logicref{Unchanged}
in order to write concise loop invariants.

\input{Listings/reverse.c.tex}



\section{The \rotatecopy algorithm}
\Label{sec:rotatecopy}

The \rotatecopy algorithm of the \cxx Standard Library \cite[\S 28.6.11]{cxx-17-draft} copies, in a particular way,
the elements of one sequence of length~$n$ into a separate sequence.
More precisely,

\begin{itemize}
\item the first~$m$ elements of the first sequence become the last~$m$ elements of the second sequence, and
\item the last~$n-m$ elements of the first sequence become the first~$n-m$ elements of the second sequence.
\end{itemize}

Figure~\ref{fig:rotatecopy} illustrates the effects of \rotatecopy
by highlighting how the initial and final segments of the array~\inl{a[0..n-1]} are mapped
to corresponding subranges of the array~\inl{b[0..n-1]}.

\begin{figure}[hbt]
\centering
\includegraphics[width=0.62\textwidth]{Figures/rotate_copy.pdf}
\caption{\Label{fig:rotatecopy} Effects of \rotatecopy}
\end{figure}

For our purposes we have modified the generic implementation
to that of a range of type \valuetype.
%
The signature now reads:

\begin{lstlisting}[style=acsl-block]

  void rotate_copy(const value_type* a, size_type m, size_type n, value_type* b);
\end{lstlisting}


%\clearpage

\subsection{Formal specification of \rotatecopy}

The specification of \rotatecopy is shown in the following listing.
Note that we require explicitly that both ranges do not overlap and that we are only
able to \emph{read} from the range~\inl{a[0..n-1]}.

\input{Listings/rotate_copy.h.tex}

\subsection{Implementation of \rotatecopy}

The following listing shows an implementation of the \rotatecopy function.
The implementation simply calls the function \copyi twice.

\input{Listings/rotate_copy.c.tex}

%\clearpage



\section{The \rotatei algorithm}
\Label{sec:rotatei}

The algorithm \rotatei is an \emph{in-place} variant of the algorithm \specref{rotatecopy}.
We have modified the generic specification of \rotatei \cite[\S 28.6.11]{cxx-17-draft}
such that it refers to a range of objects of \valuetype.
%
The signature now reads:

\begin{lstlisting}[style=acsl-block]

  size_type rotate(const value_type* a, size_type m, size_type n);
\end{lstlisting}


\subsection{Formal specification of \rotatei}

Figure~\ref{fig:rotate} shows informally the behavior of \rotatei.
The figure is of course very similar to the one for
\rotatecopy (see Figure~\ref{fig:rotatecopy}).
The notable difference is that \rotatei operates \emph{in place} of the array
\inl{a[0..n-1]}.

\begin{figure}[hbt]
\centering
\includegraphics[width=0.75\textwidth]{Figures/rotate.pdf}
\caption{\Label{fig:rotate} Effects of \rotatei}
\end{figure}

The specification of \rotatei is shown in the following listing.

\input{Listings/rotate.h.tex}

\subsection{Implementation of \rotatei}

The following listing shows an implementation of the \rotatei function
together with several \acsl annotations.
Actually, there are several ways to implement \rotatei.
We have chosen a particularly simple one that is derived from an implementation of
\inl{std::rotate} for \emph{bidirectional iterators} \cite[\S 27.2.6]{cxx-17-draft}
and which essentially consists of several calls to the algorithm~\specref{reverse}.

Note the statement contract of the final call of \specref{reverse}.
Here we use both the labels~\inl{Pre} and~\inl{Old} which refer
to the pre-states of~\reverse and the function \rotatei itself, respectively.

\input{Listings/rotate.c.tex}

\clearpage



\section{The \replacecopy algorithm}
\Label{sec:replacecopy}

The \replacecopy algorithm of the \cxx Standard Library \cite[\S 28.6.5]{cxx-17-draft} substitutes
specific elements from general sequences.
%
Here, the general implementation
has been altered to process \valuetype ranges.
The new signature reads:

\begin{lstlisting}[style=acsl-block]

  size_type replace_copy(const value_type* a, size_type n, value_type* b,
                         value_type v, value_type w);
\end{lstlisting}

The \replacecopy algorithm copies the elements from the range \inl{a[0..n]}
to range {\inl{b[0..n]}}, substituting every occurrence of \inl{v} by \inl{w}.
The return value is the length of the range.
As the length of the range is already a parameter of
the function this return value does not contain new
information.


\begin{figure}[hbt]
\centering
\includegraphics[width=0.50\textwidth]{Figures/replace.pdf}
\caption{\Label{fig:replace} Effects of \replace}
\end{figure}

Figure~\ref{fig:replace} shows the behavior of \replacecopy at hand of an example
where all occurrences of the value~3 in~\inl{a[0..n-1]} are replaced with the
value~2 in~\inl{b[0..n-1]}.


\subsection{The predicate \Replace}

We start with defining in the following listing the predicate \logicref{Replace}
that describes the intended relationship between the input array \inl{a[0..n-1]}
and the output array \inl{b[0..n-1]}.
Note the introduction of \emph{local bindings} \inl{\\let ai = ...}
and \inl{\\let bi = ...} in the definition of \Replace (see \cite[\S 2.2]{ACSLSpec}).

\input{Listings/Replace.acsl.tex}

This listing also contains a second, overloaded version of \Replace
which we will use for the specification of the related in-place
algorithm \specref{replace}.

%\clearpage

\subsection{Formal specification of \replacecopy}

Using predicate \Replace the specification of \specref{replacecopy}
is as simple as shown in the following listing.
Note that we also require that the input range \inl{a[0..n-1]} and
output range \inl{b[0..n-1]} do not overlap.

\input{Listings/replace_copy.h.tex}

\subsection{Implementation of \replacecopy}

The implementation (including loop annotations) of \implref{replacecopy}
is shown in the following listing.
Note how the structure of the loop annotations resembles
the specification of \specref{replacecopy}.

\input{Listings/replace_copy.c.tex}

\clearpage



\section{The \replace algorithm}
\Label{sec:replace}

The \replace algorithm of the \cxx Standard Library \cite[\S 28.6.5]{cxx-17-draft} substitutes
specific values in a general sequence.
%
Here, the general implementation
has been altered to process \valuetype ranges.
The new signature reads

\begin{lstlisting}[style=acsl-block]

void replace(value_type* a, size_type n, value_type v, value_type w);
\end{lstlisting}

The \replace algorithm substitutes all elements from the range \inl{a[0..n-1]}
that equal~\inl{v} by~\inl{w}.

\subsection{Formal specification of \replace}

Using the second predicate \logicref{Replace} the specification of
\specref{replace} can be expressed as in the following listing.

\input{Listings/replace.h.tex}

\subsection{Implementation of \replace}

The implementation of \implref{replace} is shown in the following listing.
The loop invariant \inl{unchanged} expresses
that when entering iteration \inl{i} the elements \inl{a[i..n-1]}
have not yet changed.

\input{Listings/replace.c.tex}

\clearpage



\section{The \removecopy algorithm (basic contract)}
\Label{sec:removecopy}

The \removecopy algorithm of the \cxx Standard Library \cite[\S 28.6.8]{cxx-17-draft}
copies all elements of a sequence other than a given value.
Here, the general implementation has been altered to process \valuetype ranges.
The new signature reads:

\begin{lstlisting}[style=acsl-block]

  size_type
  remove_copy(const value_type* a, size_type n, value_type* b, value_type v);
\end{lstlisting}

The requirements of \removecopy are:

\begin{table}[hbt]
  \begin{center}
    \begin{tabular}{|l|p{0.6\textwidth}|}
\hline
\textbf{Requirements} & \textbf{Description}
\\\hline
\hline
\namedlabel{itm:remove-size}
{\textbf{Remove Copy Size}} &
The output range has to fit in all the elements of
the input range, except the ones that equal the value~\inl{v}
        by \removecopy.
\\\hline
        \namedlabel{itm:remove-separation}{\textbf{Remove Copy Separated}} &
        The input range and the output range do not overlap
\\\hline
        \namedlabel{itm:remove-elements}{\textbf{Remove Copy Elements}} &
        The \removecopy algorithm copies elements that
        are not equal to \inl{v}
        from range
        \inl{a[0..n-1]} to the range {\inl{b[0..\\result-1]}}.
\\\hline
        \namedlabel{itm:remove-order}{\textbf{Remove Copy Stability}} &
        The algorithm is stable, that is, the
        relative order of the elements in \inl{b} is the same as in \inl{a}.
\\\hline
        \namedlabel{itm:remove-return}{\textbf{Remove Copy Return}} &
        The return value is the length of the resulting range.                  
\\\hline
        \namedlabel{itm:remove-complexity}{\textbf{Remove Copy Complexity}} &
        The algorithm takes $n$ comparisons in every case.
\\\hline
      \end{tabular}
    \end{center}
  \caption{\label{tbl:remove_copy_props}Properties of \removecopy}
\end{table}
\FloatBarrier

Figure~\ref{fig:removecopy} shows an example of how \removecopy is supposed
to copy elements that differ from~\inl{4} from the input range to the output range.

\begin{figure}[hbt]
\centering
\includegraphics[width=0.75\textwidth]{Figures/remove_copy.pdf}
\caption{\Label{fig:removecopy}Effects of \removecopy}
\end{figure}

\FloatBarrier

\subsection{Formal specification of \removecopy}

The following listing shows our first attempt to specify \removecopy.
In postcondition~\inl{discard} we use of the predicate \logicref{NoneEqual}
to show that the value \inl{v} does not occur in the range \inl{b[0..\\result]}.

\input{Listings/remove_copy.h.tex}

One shortcoming of this specification is that the postcondition \inl{bound}
only makes very general and not very precise statements about the number of copied elements.
We will address this problem in the contract of \specref{removecopyii}.
A more serious shortcoming is, however, that we haven't specified what
the relationship between the elements of the input range \inl{a[0..n-1]}
and the output range \inl{b[0..\\result-1]} looks like.
This problem will be tackled in the contract of \specref{removecopyiii}.

%\clearpage

\subsection{Implementation of \removecopy}

An implementation of \removecopy is shown in the following listing.

\input{Listings/remove_copy.c.tex}

Here we also need to add another loop invariant~\inl{discard} which basically 
checks if \inl{v} occurs in \inl{b[0..k]} for each iteration of the loop.

%\clearpage



\section{The \removecopyii algorithm (number of copied elements)}
\label{sec:removecopyii}

In this section we improve the contract of \specref{removecopy}
by formally specifying the number~\inl{\\result} of elements copied by \removecopy.

The number of copied elements equals of course the number of elements
in the input range \inl{a[0..n-1]} that are different from~\inl{v}.
One can formally describe this number by relying on the logic function~\logicref{Count}.

\begin{lstlisting}[style=acsl-block]

  logic integer
  CountNotEqual(value_type* a, integer n, value_type v) =  n - Count(a, n, v);
\end{lstlisting}

In fact, we have used this kind of definition in earlier version of this document.
We have found it, however, worthwhile to provide a separate definition of \CountNotEqual
and express the relationship with \Count as a lemma.
This definition is shown in the Listings~\ref{logic:CountNotEqual-1}
and~\ref{logic:CountNotEqual-2}.

\begin{logic}[hbt]
\begin{minipage}{\textwidth}
\lstinputlisting[linerange={1-34}, style=acsl-block, frame=single]{Source/CountNotEqual.acsl}
\end{minipage}
\caption{\Label{logic:CountNotEqual-1} The logic function \CountNotEqual (1)}
\input{Listings/CountNotEqual.acsl.labels.tex}
\input{Listings/CountNotEqual.acsl.index.tex}
\end{logic}


\clearpage

The  above mentioned relationship with \logicref{Count} is expressed as
lemma \logicref{CountNotEqualCount} in the following listing.

\begin{logic}[hbt]
\begin{minipage}{\textwidth}
\lstinputlisting[linerange={35-60}, style=acsl-block, frame=single]{Source/CountNotEqual.acsl}
\end{minipage}
\caption{\Label{logic:CountNotEqual-2} The logic function \CountNotEqual (2)}
\end{logic}

\FloatBarrier


\subsection{Formal specification of \removecopyii}

We extend our formal specification by using \logicref{CountNotEqual}
and add the new postcondition \inl{size}, which states that the returning value
of \removecopyii equals \CountNotEqual.
The following listing shows the formal specification of \specref{removecopyii}.

\input{Listings/remove_copy2.h.tex}

\subsection{Implementation of \removecopyii}

The following listing shows the implementation of our
extended of \removecopyii.
Here we added the loop invariant \inl{size} which corresponds to the 
postcondition in \specref{removecopyii}.
In order to ensure that the loop invariant \inl{size} can be verified
we have added the assertions \inl{size} and \inl{unchanged}.

\input{Listings/remove_copy2.c.tex}

While we now can precisely speak of the number of copied elements,
it is still not possible to say something about the exact relationship between
the elements of range~\inl{a[0..n-1]} and range~\inl{b[0..n-1]}.
We will address this question the contract of \specref{removecopyiii}.

\clearpage



\section{The \removecopyiii algorithm (final contract)}
\label{sec:removecopyiii}

In this section we extend the contracts of \specref{removecopy}
and \specref{removecopyii} by introducing a 
logic function, which describes the relationship between the elements of
input range \inl{a[0..n-1]} and the output range \inl{b[0..\\result-1]}.
Note that we have shown in the previous section that
\inl{\\result} equals \inl{CountNotEqual(a, n, v)}.

\subsection{A closer look on the properties of \removecopy}
\label{sec:formal-view-remove}

Figure~\ref{fig:removecopy-trip} shows a modified version of the
Figure~\ref{fig:removecopy}. We left out the indices of values that were not copied into the
target array. Furthermore we have added a dashed arrow which points to the
index that corresponds to the \emph{one past the end} location of the input and
output range.

\begin{figure}[hbt]
\begin{center}
\includegraphics[width=0.70\textwidth]{Figures/remove_copy_partition.pdf}
\end{center}
\caption{\label{fig:removecopy-trip} Partitioning the input of \removecopy}
\end{figure}
\FloatBarrier

These arrows between the indices of the array~\inl{b} and array~\inl{a} define
the following sequence~$p$ of seven indices. The index of the \emph{one past the end} is underlined. 
$p = (1, 2, 5, 7, 8, 10, \underline{11})$


%\clearpage 

More generally, we refer to the sequence~$p$ as \emph{partitioning sequence} of \removecopy for the array \inl{a[0..n-1]}.
For the \textbf{length of a partitioning sequence} $m$ we get the following \textbf{strictly monotone increasing} sequence:
\begin{align}
  \label{eq:remove-monotone}
  0 &\leq  p_0 < ... < p_{m} = n \\
  \intertext{and the open index intervals}
  %\label{eq:remove-interval}
  \nonumber
  (p_i,&p_{i+1}) && \forall i: 0 \leq i < m\\
  \intertext{mark \textbf{consecutive ranges} of the value \inl{v}
  in the source array, that is,}
  \label{eq:between}
  a[k] &= v &&\forall k: p_i < k < p_{i+1}\\
  \intertext{Additionally, the half open interval}
  \nonumber
  [0,&p_{0})\\ 
  \intertext{also marks another \textbf{consecutive range} of the value \inl{v} in the
  source array:}
  \label{eq:beginning}
  a[k] &= v &&\forall k: 0 \leq k < p_{0}\\
  \intertext{Another observation is that} 
  \label{eq:a_nv}
  a[p_i] &\neq v &&\forall i: 0 \leq i < m\\
  \intertext{holds. Finally, we have}
  \label{eq:b_eqa}
  a[p_i] &= b[i] &&\forall i: 0 \leq i < m\\
  \intertext{which, together with the inequality~\eqref{eq:a_nv} 
  states, that the target does not contain the value~\inl{v}}
  \nonumber
  %\label{eq:final}
  b[i] &\neq v &&\forall i:0 \leq i < m 
\end{align} 

\subsection{More lemmas on \CountNotEqual}

Our formalization the properties of \S\ref{sec:formal-view-remove}
relies on the logic function \logicref{CountNotEqual}.
We also rely on the logic function \logicref{FindNotEqual} and
the lemmas of \logicref{CountFindNotEqual} in the following listing that provide more
facts about \CountNotEqual and \FindNotEqual.

\input{Listings/CountFindNotEqual.acsl.tex}

\clearpage

\subsection{Formalizing the properties of the partitions}

The function~\RemovePartition, whose axiomatic definition is given in 
Listings~\ref{logic:RemovePartition-1} and~\ref{logic:RemovePartition-2}
defines the partitioning sequence~$p$ from \S\ref{sec:formal-view-remove}.


\begin{logic}[hbt]
\begin{minipage}{\textwidth}
\lstinputlisting[linerange={1-53}, style=acsl-block, frame=single]{Source/RemovePartition.acsl}
\end{minipage}
\caption{\Label{logic:RemovePartition-1} The logic function \RemovePartition (1)}
\input{Listings/RemovePartition.acsl.labels.tex}
\input{Listings/RemovePartition.acsl.index.tex}
\end{logic}

\FloatBarrier

Before we begin to relate the various lemmas
to the formulas from \S\ref{sec:formal-view-remove} we want to remind
the reader that logic functions (and predicates) must be total that is they must
be defined for all possible argument values.

\begin{logic}[hbt]
\begin{minipage}{\textwidth}
\lstinputlisting[linerange={54-103}, style=acsl-block, frame=single]{Source/RemovePartition.acsl}
\end{minipage}
\caption{\Label{logic:RemovePartition-2}The logic function \RemovePartition (2)}
\end{logic}

\FloatBarrier

The lemmas for \RemovePartition are related to the properties of 
\S\ref{sec:formal-view-remove} in the following way.

\begin{itemize}
  \item Property \eqref{eq:remove-monotone} is expressed by the lemmas
  \RemovePartitionEmpty, \RemovePartitionLeft
  \RemovePartitionRight, and \RemovePartitionStrictlyWeakIncreasing

  \item Properties~\eqref{eq:between} and~\eqref{eq:beginning}
  are described by lemmas \RemovePartitionSegment.

  \item Property~\eqref{eq:a_nv} is expressed by lemma \RemovePartitionNotEqual.

  \item Property~\eqref{eq:b_eqa} is formulated using the predicate \logicref{Remove}.
\end{itemize}

\clearpage

We would like to point out lemma \RemovePartitionCore which subsumes
the statements of the subsequent lemmas \RemovePartitionUpper,
\RemovePartitionNotEqual,\\
and \RemovePartitionCount.
While these three lemmas add nothing new 
we have kept them because they correspond directly to individual properties
of \S\ref{sec:formal-view-remove}.
The question may arise why there is the lemma \RemovePartitionCore in the first place.
The answer is that we found the individual properties so intertwined
that we were not able to verify them separately but only their joint embodiment.


\subsection{The predicate \Remove}

The predicate \logicref{Remove} primarily serves 
in order to improve the readability of our specification \specref{removecopyiii}.
As mentioned before this predicate encapsulates the Property~\eqref{eq:b_eqa}
from \S\ref{sec:formal-view-remove}.
Note that \logicref{Remove} also contains an overloaded version of
\Remove which will be used for the specification of the \emph{in-place}
variant \specref{remove} of \removecopy.

\input{Listings/Remove.acsl.tex}

\clearpage

\subsection{Formal specification of \removecopyiii}

The following listing shows the formal specification of \specref{removecopy}.
The additional postcondition \inl{remove} makes use of the predicate \logicref{Remove}
which we have just described.
Furthermore, we have again the postcondition~\inl{unchanged} which states that
the source array~\inl{a[0..n-1]} does not change.

\input{Listings/remove_copy3.h.tex}

\subsection{Implementation of \removecopyiii}
\label{sec:removecopyiii:impl}

We discuss now some aspects of the implementation of \implref{removecopyiii}.
We introduce the loop invariant~\inl{mapping}.
This invariant states that the variable~\inl{i} will
always be smaller or equal to the result of \inl{RemovePartition(a, n, v, k)}.
We also add the assertion~\inl{mapping} to our implementation as stepping stone
for the provers to verify the correctness of this loop invariant.

Somewhat surprisingly, in order to reduce excessive verification times we had to add
an else-branch to our implementation that besides the assertion \inl{unchanged}
is empty.

Regarding the assertion \inl{update}, one might wonder why we do not simply write
\inl{\\at(a[i], Pre)}. 
However, this expression would be wrong because the index~\inl{i}
would then be interpreted as
\inl{\\at(i,Pre)} which doesn't makes sense for a local variable.
\wpframac consequently rejects this expression with the following error message.

\begin{small}
\begin{verbatim}
       Warning: unbound logic variable i. Ignoring code annotation
\end{verbatim}
\end{small}

\clearpage

We could explicitly refer to the current value of~\inl{i} by using
the subexpression \inl{\\at(i,Here)} inside the assertion~\inl{update}.
We felt, however, tow introduce the predicate \logicref{At}
to simplify the comparison of array elements in programme states 
where the particular index variable isn't visible.

\input{Listings/At.acsl.tex}

The second argument \At is interpreted at the programme point here it appears,
that is, \inl{Here}.
Using this auxiliary logic function the assertion \inl{update}
is arguably more readable.

\input{Listings/remove_copy3.c.tex}

\clearpage



\section{The \remove algorithm}
\Label{sec:remove}

The \cxx Standard Library also
contains a function \remove\cite[28.6.8]{cxx-17-draft} performing
the same operation as \removecopy as an in-place algorithm.
Its signature is very similar to that of \removecopy,
except that there is no need for an output array.

\begin{lstlisting}[style=acsl-block]

  size_type remove(value_type* a, size_type n, value_type v);
\end{lstlisting}

Figure~\ref{fig:remove} shows how \remove is supposed
to remove all occurrences of the given value~4 from a range.

\begin{figure}[hbt]
\centering
\includegraphics[width=0.85\textwidth]{Figures/remove.pdf}
\caption{\Label{fig:remove}Effects of \remove}
\end{figure}

\FloatBarrier

\subsection{Formal specification of \remove}

The following listing shows a formal specification of the function \specref{remove}.
Our specification is very similar to the one of \specref{removecopyiii}
except that we using a version of \logicref{Remove} that takes only one pointer argument.

\input{Listings/remove.h.tex}

\clearpage

\subsection{Implementation of \remove}

In the following listing we show our implementation of \implref{remove} together with
the additional loop annotations.
Again, the annotations are very similar to those of the
implementation of \implref{removecopyiii}.

\input{Listings/remove.c.tex}

Also note the use of the predicate \logicref{At} in the loop invariant \inl{unchanged}
and the assertion \inl{update}.

\clearpage



\section{The \shuffle algorithm}
\Label{sec:shuffle}

The \shuffle  algorithm in the \cxx Standard Library \cite[\S 28.6.13]{cxx-17-draft}
randomly rearranges the elements of a given range, that is,
it randomly picks one of its possible orderings.
For our purposes we have modified
the generic implementation
to that of a range of type \valuetype.
The signature now reads:

\begin{lstlisting}[style=acsl-block]

  void shuffle(value_type* a, size_type n, unsigned short* rand);
\end{lstlisting}

The argument \inl{rand} holds the state of a simple random number generator
that is used in the implementation of \shuffle.

Figure~\ref{fig:shuffle} illustrates an example run of \shuffle.
In this figure, the values 1, 2, 3, and 4 occur 
twice, once, once, and three times, respectively, both before and
after the \shuffle run.
This expresses that the range has been reordered.

\begin{figure}[hbt]
\centering
\includegraphics[width=0.60\textwidth]{Figures/shuffle.pdf}
\caption{\Label{fig:shuffle} Effects of \shuffle}
\end{figure}

\clearpage

\subsection{The predicate \MultisetReorder}
\Label{sec:multisetunchanged}

The \shuffle algorithm is the first example in this document
where we have to specify a \emph{rearrangement} or \emph{reordering}
of the elements of a given range.
We say that an array has been reordered between two states
if the number of each element in the array remains unchanged.
In other words, reordering leaves the \emph{multiset}\footnote{
 See \url{http://en.wikipedia.org/wiki/Multiset}
}
of elements in the range unchanged.

We use the predicate \logicref{MultisetReorder} 
to formally describe this property.
This predicate, which is given in two overloaded versions,
relies on the logic function \logicref{Count}.
We list here several lemma with basic properties of \MultisetReorder.
We will use these lemmas during the verification of various algorithms.

\input{Listings/MultisetReorder.acsl.tex}

\clearpage

\subsection{Formal specification of \shuffle}

In the specification of the \specref{shuffle} algorithm we demand that the range \inl{a[0..n-1]}
is valid for reading and writing.
We use the predicate \logicref{MultisetReorder} to express that the
contents of \inl{a[0..n-1]} is just permuted, i.e., the number of occurrences
of each of its members remains unchanged. 
The array \inl{rand} contains a seed for the random number generator used to randomize the
shuffle.
By specifying that the function assigns to \inl{rand}
we capture that the function may return a different permutation every time.

Note that our specification only states that the resulting range is
a reordering of the input range; nothing more and nothing less.
Ideally, we would also specify that sequence of reorderings
obtained by repeated
calls of \shuffle is required to be random.
The informal specification\cite[\S 28.6.13]{cxx-17-draft} of \shuffle states that
\emph{that each possible permutation of those elements has 
equal probability of appearance}.
However, \acsl does currently not support the specification of
temporal properties related to repeated call results.

\input{Listings/shuffle.h.tex}

More generally speaking, it is not trivial to capture
the notion of randomness in a mathematically precise way.
As a typical example, we refer to a paper\cite[p.6--8]{Moses.Oakford.1963},
which just gives four statistical tests indicating the randomness of the
permutations computed with their algorithm.
From a theoretical point of view, a sequence of permutations
can be called ``random'' if its Kolmogorov complexity exceeds
a certain measure, however, this property is undecidable\cite{Li.Vitanyi.1997}.

\clearpage

\subsection{Implementation of \shuffle}

The following listing shows our implementation of the function \implref{shuffle}.
It repeatedly calls the function \specref{swap} to \emph{transpose}
(randomly) selected elements.
For details of out source of randomness we refer to the function \specref{randomnumber}.

\input{Listings/shuffle.c.tex}

The loop invariants \inl{reorder} and \inl{unchanged} of \shuffle 
are necessary for the verification of the postcondition~\inl{reorder}:
in the \inl{i}th loop cycle, the subrange \inl{a[0..i-1]} has been
reordered, while the remaining subrange \inl{a[i..n-1]} is yet unchanged.
We also formulate several auxiliary assertions \inl{reorder} which use the
the predefined label \inl{LoopCurrent},
to guide the automatic verification the loop invariant \inl{reorder}.
Please not the empty \inl{else}-branch hat only contains an assertion \inl{reorder}.
We introduced this assertion to support the verification of the \inl{reorder} property.


\clearpage

In addition, we rely on the predicate \logicref{ArraySwap}
rather than the literal postcondition of \specref{swap}, since this leads to
to more concise annotations and better a performance of the automatic provers.

\input{Listings/ArraySwap.acsl.tex}

The lemma \logicref{MultisetSwapMiddle} states that swapping the elements
\inl{a[i]} and \inl{a[k]} is a particular
kind of reordering on the range \inl{a[i..k]}.

\input{Listings/MultisetSwap.acsl.tex}

\clearpage



\section{Verifying a random number generator}
\Label{sec:randomnumber}

We describe in this section \specref{randomnumber} which implements
a simple random-number generator.
As in the case of \specref{shuffle} itself, we do not formulate precise
properties of randomness and only require its result to
be in the specified range \inl{[0..n-1]}.
Again, the \inl{assigns} clause to the array \inl{state} models the
dependency on an additional state.

Note that in the following listing, we also provide the rather simple
specification of the function \randominit that is called to initialize the 
state of the random generator.

\input{Listings/random_number.h.tex}

The implementations of \randomnumber and \randominit are shown in the following listing.
Internally, we rely on a custom implementation of the POSIX.1
random number generator \inl{lrand48()}\footnote{
  See \url{http://pubs.opengroup.org/onlinepubs/9699919799/functions/lrand48.html}
}
This random number generator is a linear congruence
generator with a 48~bit state and the iteration procedure
\begin{equation}
\label{eq:random}
x_{n+1}=ax_n+c\bmod 2^{48}
\end{equation}
%
where $a=25214903917$ and $c=11$ are relatively prime integers.

As a part of the iteration procedure in Equation~\eqref{eq:random}
an unsigned overflow may occur.
This does not affect the result as we are only interested in its lowest 48 bits.
However, as one of the options we use, \inl{-warn-unsigned-overflow},
causes \wpframac  assert the absence of unsigned overflow this algorithm does not verify under
the same options used for the other algorithms.
%
As an exception, we have therefore decided to disable \inl{-warn-unsigned-overflow}
for this function as the unsigned overflow is both benign and
well-defined (cf.\ \cite[\S 6.2.5, 9]{isoc}).

\input{Listings/random_number.c.tex}

Note that we use the custom acsl lemma \logicref{RandomNumberModulo}
from the following listing to support the verification of some assertions.

\input{Listings/C_Bit.acsl.tex}

\clearpage



