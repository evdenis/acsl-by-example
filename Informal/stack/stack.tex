
\chapter{The \stacktype data type}
\Label{cha:stack}

So far we have used the \acsl specification language for the task of specifying and verifying
one single \isoc function at a time.
%
However, in practice we are also faced with the task to implement a
family of functions, usually around some sophisticated data structure,
which have to obey certain rules of interdependence.
%
In this kind of task, we are not only interested in the properties of a
single function but also in properties describing how several function play
together.

The \cxx Standard Library provides  a generic container adaptor
\inl{stack} \cite[\S 26.6.6]{cxx-17-draft} whose signature and
behavior we try to follow as far as our \isoc implementation it allows.
For a more detailed discussion of our approach to the formal verification
of \stacktype we refer to Kim Völlinger's thesis \cite{Voellinger_2013_da}.
%

A \emph{stack} is a data type that can hold objects and has the
property that, if an object $a$ is \emph{pushed} on a stack
\emph{before} object~$b$, then~$a$ can only be removed (\emph{popped})
after~$b$.
%
A stack is, in other words, a \emph{first-in, last-out}
data type (see~Figure~\ref{fig:stack}).
%
The \emph{top} function of a stack returns the last element
that has been pushed on a stack.

\begin{figure}[hbt]
\centering
\includegraphics[width=0.55\linewidth]{Figures/stack.pdf}
\caption{\Label{fig:stack} Push and pop on a stack}
\end{figure}

We consider only stacks that have a finite \emph{capacity}, that is,
that can only hold a maximum number~$c$ 
of elements that is constant throughout their lifetime.
%
This restriction allows us to define a stack without relying
on dynamic memory
allocation.
%
When a stack is \emph{created} or \emph{initialized}, it contains
no elements, i.e.,
its \emph{size} is~0.
%
The function \emph{push} and \emph{pop} increases and decreases
the size of a stack by at most one, respectively.

\clearpage

\section{Methodology overview}
\Label{sec:Methodology Overview}

Figure~\ref{fig:Methodology Overview}
gives an overview of our methodology to specify and verify abstract
data types (verification of one axiom shown only).

\begin{figure}[hbt]
\centering
\includegraphics[width=0.99\linewidth]{Figures/push_pop_axiom.pdf}
\caption{Methodology Overview}
\Label{fig:Methodology Overview}
\end{figure}

What we will basically do is:
\begin{enumerate}
\item  specify axioms about how the stack functions should
  interact with each other
  (\S\ref{sec:stack-axioms}),
\item  define a basic implementation of \isoc data structures
  (only one in our example, viz.\\
  \inl{struct Stack};
  see \S\ref{sec:stack-definition})
  and some invariants the instances of them have to obey
  (\S\ref{sec:stack-invariants}),
\item  provide for each stack function an \acsl contract and a 
  \isoc implementation 
  (\S\ref{sec:stack-functions}),
\item  verify each function against its contract
  (\S\ref{sec:stack-functions}),
\item  transform the axioms into \acsl-annotated \isoc code
  (\S\ref{sec:stack-verification}), and
\item  verify that code, using access function contracts and
  data-type invariants as necessary
  (\S\ref{sec:stack-verification}).
\end{enumerate}

\S\ref{sec:stack-equality}
provides an \acsl-predicate deciding whether two instances of a
\inl{struct Stack} are considered to be equal (indication by ``$\approx$''
in Figure~\ref{fig:Methodology Overview}), while
\S\ref{sec:stackequal} gives a
corresponding \isoc implementation.
The issue of an appropriate definition of equality of data instances
is familiar to any \isoc programmer who had to replace a faulty
comparison \inl{if(s1 == s2)} by the correct 
\inl{if(strcmp(s1,s2) == 0)} to compare two strings 
\inl{char *s1,*s2} for equality.

\clearpage


\section{Stack axioms}
\Label{sec:stack-axioms}

To specify the interplay of the stack access functions,
we use a set of axioms\footnote{
There is an analogy in geometry:
Euclid (e.g.\ \cite{Fitzpatrick.2008}) invented the use of
axioms there, but still kept definitions of \emph{point},
\emph{line}, \emph{plane}, etc.
Hilbert \cite{Hilbert.1968} recognized that the latter are not
only unformalizable, but also unnecessary, and dropped them,
keeping only the formal descriptions of relations between them.
},
all but one of them having the form of a conditional equation.


Let $V$ denote an arbitrary type.
We denote by $S_c$ the type of stacks with capacity $c > 0$ of
elements of type $V$.
The aforementioned functions then have the following signatures.

\begin{align*}
\mathrm{init} &:  S_c \rightarrow S_c, \\
\mathrm{push} &: S_c\times V \rightarrow S_c, \\
\mathrm{pop} &: S_c \rightarrow S_c,\\
\mathrm{top} &: S_c \rightarrow V, \\
\mathrm{size} &: S_c \rightarrow \mathbb{N}.\\
\end{align*}

With $\mathbb{B}$ denoting the \emph{boolean}
type we will also define two auxiliary functions
\begin{align*}
\mathrm{empty} &: S_c \rightarrow \mathbb{B},\\
\mathrm{full} &: S_c \rightarrow \mathbb{B}.
\end{align*}

To qualify as a stack these functions must satisfy the following rules
which are also referred to as \emph{stack axioms}.

\subsection{Stack initialization}

After a stack has been initialized its size is~0.
\begin{align}
\Label{eq:stack-init-size}
\mathrm{size}(\mathrm{init}(s)) &= 0.
\end{align}

The auxiliary functions $\mathrm{empty}$ and $\mathrm{full}$
are defined as follows
\begin{align}
\Label{eq:empty-stack}
\mathrm{empty}(s), & \qquad\text{iff}\qquad \mathrm{size(s)} = 0,  \\
\Label{eq:full-stack}
\mathrm{full}(s), & \qquad\text{iff}\qquad  \mathrm{size(s)} = c.
\end{align}

We expect that for every stack $s$ the following condition holds
\begin{align}
\Label{eq:stack-invariant}
 0 \leq \mathrm{size}(s) \leq c.
\end{align}

\clearpage

\subsection{Adding an element to a stack}

To push an element $v$ on a stack the stack must not be full.
If an element has been pushed on an eligible stack, its size increases by~1
\begin{align}
\Label{eq:stack-size-push}
\mathrm{size}(\mathrm{push}(s, v)) &= \mathrm{size}(s)+1, 
  &\text{if}\quad  \neg\mathrm{full}(s).\\
\intertext{Moreover, the element pushed on a stack is the top element of the resulting stack}
\Label{eq:stack-top-push}
\mathrm{top}(\mathrm{push}(s, v)) &= v, 
  &\text{if}\quad \neg\mathrm{full}(s).
\end{align}

\subsection{Removing an element from a stack}

An element can only be removed from a non-empty stack.
If an element has been removed from an eligible stack the
stack size decreases by~1
\begin{align}
\Label{eq:stack-size-pop}
\mathrm{size}(\mathrm{pop}(s)) &= \mathrm{size}(s)-1,
  &&\text{if}\quad \neg\mathrm{empty}(s).
%
\intertext{
If an element is pushed on a stack and immediately afterwards
an element is removed from the resulting stack then the final stack
is equal to the original stack}
%
\Label{eq:stack-pop-push}
\mathrm{pop}(\mathrm{push}(s, v)) &= s,
  &&\text{if}\quad \neg\mathrm{full}(s). \\
\intertext{Conversely, if an element is removed from a non-empty stack
  and if afterwards the top element of the original stack is
  pushed on the new stack
  then the resulting stack is equal to the original stack.}
%
\Label{eq:stack-push-pop-top}
\mathrm{push}(\mathrm{pop}(s),\mathrm{top}(s)) &= s, &&\text{if}\quad \neg\mathrm{empty}(s). 
\end{align}

\clearpage

\subsection{A note on exception handling}

We don't impose a requirement on \inl{push(s, v)} if \inl{s}
is a full stack, nor on \inl{pop(s)} or \inl{top(s)} if \inl{s} is an
empty stack.
%
Specifying the behavior in such \emph{exceptional} situations is a
problem by its own; a variety of approaches is discussed in the
literature.
%
We won't elaborate further on this issue, but only give an example to warn
about ``innocent-looking'' exception specifications that may lead to
undesired results.

If we'd introduce an additional error value \inl{err} in the element type 
$V$ and require \inl{top(s) = err} if \inl{s} is empty, we'd be faced
with the problem of specifying the behavior of \inl{push(s, err)}.
%
At first glance, it would seem a good idea to have \inl{err} just been
ignored by \inl{push}, i.e.\ to require
\begin{align}
\Label{eq:err}
\mathrm{push}(s,\mathrm{err}) & = s.
\end{align}

However, we then could derive for any non-full and non-empty stack \inl{s}, that

\begin{align*}
\mathrm{size}(s)
   &=  \mathrm{size}(\mathrm{pop}(\mathrm{push}(s, \mathrm{err}))) && \text{by \ref{eq:stack-pop-push}}\\
   &=  \mathrm{size}(\mathrm{pop}(s)) && \text{as assumed in \ref{eq:err}}\\
   &=  \mathrm{size}(s) - 1 && \text{by \ref{eq:stack-size-pop}}\\
\end{align*}
%
i.e.\ no such stacks could exist, or all \inl{int} values would be equal.


\clearpage

\section{The structure \stacktype and its associated functions}
\Label{sec:stack-definition}

We now introduce one possible \isoc implementation of the above axioms.
It is centred around the \isoc structure \stacktype shown in the following listing.

\begin{listing}[hbt]
\centering
\begin{minipage}{0.9\textwidth}
\lstinputlisting[style=acsl-block, frame=single]{Source/stack.h}
\end{minipage}
\caption{\Label{lst:stack-definition}Definition of type \stacktype}
\end{listing}

This struct holds an array \inl{obj} of positive length called \inl{capacity}. 
The capacity of a stack is the maximum number of elements this stack can hold.
The field \inl{size} indicates the number elements that
are currently in the stack.
See also Figure~\ref{fig:stack-struct} which attempts to interpret
this definition according to Figure~\ref{fig:stack}.

\begin{figure}[hbt]
\centering
\includegraphics[width=0.60\linewidth]{Figures/stack-struct.pdf}
\caption{\Label{fig:stack-struct}Interpreting the data structure \stacktype}
\end{figure}

\FloatBarrier
\clearpage

Based on the stack functions from \S\ref{sec:stack-axioms},
we declare in the next listing the following functions as part of
our \stacktype data type.

\begin{listing}[hbt]
\centering
\begin{minipage}{0.9\textwidth}
\lstinputlisting[style=acsl-block, frame=single]{Source/stack_functions.h}
\end{minipage}
\caption{\Label{lst:stack-functions}Declaration of functions of type \stacktype}
\end{listing}

Most of these functions directly correspond to methods of the
\cxx \inl{std::stack} template class \cite[\S 26.6.6.1]{cxx-17-draft}.
The function \stackequal corresponds to the comparison operator~\inl{==},
whereas one use of \stackinit is to bring a stack into a
well-defined initial state.
The function \stackfull has no counterpart in \inl{std::stack}.
This reflects the fact that we 
avoid dynamic memory allocation, while \inl{std::stack} does not.

\clearpage

\section{Stack invariants}
\Label{sec:stack-invariants}

Not every possible instance of type \stacktype is considered a
valid one, e.g., with our definition of \stacktype in Listing~\ref{lst:stack-definition},
\inl{Stack s = \{\{0,0,0,0\},4,5\}} is not.
In the following listing, we present
basic logic functions and predicates that we will use
throughout this chapter
In particular, we  define the predicate \logicref{StackInvariant} that
discriminates valid and invalid instances.

\input{Listings/StackInvariant.acsl.tex}

We start, with the auxiliary logic function
\StackCapacity, \StackSize and \StackStorage
which we can use in specifications to refer
to the fields \inl{capacity}, \inl{size} and \inl{obj} of \stacktype, respectively.
%
This listing also contains the logic function \StackTop which defines
the array element with index \inl{size - 1} as the top place of a stack.

The reader can consider this as an attempt to hide
implementation details from
the specification.
%
We intentionally use here integer as a return value of these logic functions.
Inaccurate use of logic functions with bounded types in axioms with
arithmetic operations may lead to inconsistencies.

We also introduce the predicates \logicref{StackEmpty} and \logicref{StackFull}
that express the concepts of empty and full stacks
by referring to a stack's size and capacity (see Equations~\eqref{eq:empty-stack}
and~\eqref{eq:full-stack}).

There are some obvious invariants that must be fulfilled by every
valid object of type \stacktype:
\begin{itemize}
\item The stack capacity shall be strictly greater than zero
      (an empty stack is ok but a stack that cannot hold anything is not useful).

\item The pointer \inl{obj} shall refer to an array of length \inl{capacity}.

\item The number of elements \inl{size} of a stack the must
      be non-negative and not greater than its capacity.
\end{itemize}

These invariants are all formalized in the predicate \logicref{StackInvariant}.

Note how the use of the previously defined logic functions and predicates
allows us to define the stack invariant without
directly referring to the fields of \stacktype.

We sometimes wish to express that there is no \emph{memory aliasing} between two stacks.
If there were aliasing, then modifying one stack could modify the other
stack in unexpected ways.
In order to express that there is no aliasing between two stacks,
we define the predicate \StackSeparated in the next listing.

\input{Listings/StackUtility.acsl.tex}

This listing also contains the predicate \logicref{StackUnchanged}
that we will use to describe cases that the contents of a stack hasn't changed.

\clearpage

\section{Equality of stacks}
\Label{sec:stack-equality}

Defining equality of instances of non-trivial data types, in
particular in object-oriented languages, is not an easy task.
%
The book \emph{Programming in Scala}\cite[Chapter~28]{OderskyEtAl2008} 
devotes to this topic a whole chapter of more than twenty pages.
%
In the following two sections we give a few hints how \acsl
and \framac can help to
correctly define equality for a simple data type.

We consider two stacks as equal if they have the same size and if they contain the same objects.
%
To be more precise, let~\inl{s} and~\inl{t} two pointers of type \stacktype,
then we define the predicate \StackEqual as in the following listing.

\input{Listings/Stack.acsl.tex}

Our use of labels in this listing makes
the specification somewhat hard to read (in particular in the last line
where we reuse the predicate \logicref{Equal}.
%
However, this definition of \StackEqual will allow us later to compare 
the same stack object at different points of a program.
%
The logical expression \inl{StackEqual\{A,B\}(s,t)}
reads informally as: 
{The stack object \inl{*s} at program point \inl{A}
equals the stack object \inl{*t} at program point \inl{B}}.

The reader might wonder why we exclude the capacity of a stack
into the definition of stack equality.
This approach can be motivated with the behavior of the method
\inl{capacity} of the class \inl{std::vector<T>}.
There, equal instances of type \inl{std::vector<T>} may very well 
have different capacities.\footnote{
See \url{http://www.cplusplus.com/reference/vector/vector/capacity}
}

If equal stacks can have different capacities then, according to our
definition of the predicate \logicref{StackFull}, 
we can have to equal stacks where one is full and the other one is not.

A finer, but very important point in our specification of equality
of stacks is that the elements of the arrays \inl{s->obj} and \inl{t->obj}
are compared only up to \inl{s->size} and \emph{not} up  to
\inl{s->capacity}.
Thus the two stacks \inl{s} and \inl{t} in Figure~\ref{fig:equal-stacks}
are considered
equal although there is are obvious differences in their internal arrays.

\begin{figure}[hbt]
\centering
\includegraphics[width=0.75\linewidth]{Figures/stack12.pdf}
\caption{\Label{fig:equal-stacks} Example of two equal stacks}
\end{figure}

\FloatBarrier

If we define an equality relation $(=)$ of objects for a data
type such as \stacktype,
we have to make sure that the following rules hold.

\begin{subequations}
\Label{eq:equivalence-relation}
\begin{align}
   \text{reflexivity}\qquad && \forall s \in S&: s = s,\\
   \text{symmetry}\qquad &&    \forall s,t \in S&: s = t \implies t = s,\\
   \text{transitivity}\qquad &&    \forall s,t,u \in S&: s = t \land t = u \implies s = u.
\end{align}
\end{subequations}


Any relation that satisfies the conditions~\eqref{eq:equivalence-relation}
is referred to as an \emph{equivalence relation}.
%
The mathematical set of all instances that are considered equal to
some given instance \inl{s} is called the equivalence class of \inl{s}
with respect to that relation.

Our formalization of \logicref{StackEquality} shows 
these three rules for the relation \StackEqual;
it can be automatically verified that they are a consequence of the
definition of \StackEqual.

The two stacks in Figure~\ref{fig:equal-stacks} show that
an equivalence class of \StackEqual
can contain more than one element.\footnote{
    This is a common situation in mathematics. For example,
    the equivalence class of
    the rational number $\frac{1}{2}$ contains infinitely many elements,
    viz.\ $\frac{1}{2},
    \frac{2}{4}, \frac{7}{14}, \ldots$.
}
The stacks \inl{s} and \inl{t} in Figure~\ref{fig:equal-stacks}
are also referred to as two \emph{representatives} of the
same equivalence class.
In such a situation, the question arises whether a function
that is defined on
a set with an equivalence relation can be defined in such a
way that its definition
is \emph{independent of the chosen representatives}.\footnote{
    This is why mathematicians know that
    $\frac{1}{2} + \frac{3}{5}$
    equals $\frac{7}{14} + \frac{3}{5}$.
}
We ask, in other words, whether the function is \emph{well-defined}
on the set of all equivalence classes of the relation \StackEqual.\footnote{
  See \url{http://en.wikipedia.org/wiki/Well-definition}.
}
The question of well-definition
will play an important role when verifying the functions of
the \stacktype (see \S\ref{sec:stack-functions}).

\clearpage

\section{Verification of stack functions}
\Label{sec:stack-functions}

In this section we verify the functions
\begin{itemize}
\item \stackequal (\S\ref{sec:stackequal})
\item \stackinit (\S\ref{sec:stackinit})
\item \stacksize (\S\ref{sec:stacksize})
\item \stackfull (\S\ref{sec:stackfull})
\item \stackempty (\S\ref{sec:stackempty}) 
\item \stacktop (\S\ref{sec:stacktop})
\item \stackpush (\S\ref{sec:stackpush})
\item \stackpop (\S\ref{sec:stackpop})
\end{itemize}

of the data type \stacktype.
To be more precise, we provide for each of function \inl{stack_foo}:
\begin{itemize}
\item an \acsl specification of \inl{stack_foo} 
\item a \isoc implementation of \inl{stack_foo}
\item a \isoc function \inl{stack_foo_wd}\footnote{
  The suffix \inl{_wd} stands for \emph{well definition}
}
accompanied by a an \acsl contract that expresses that
the implementation of \inl{stack_foo} is well-defined.
Figure~\ref{fig:methodology-wd}
shows our methodology for the verification of
well-definition in the \inl{pop} example,
$(\approx)$ again indicating the user-defined \stacktype equality.
\end{itemize}

\begin{figure}[hbt]
\centering
\includegraphics[width=0.95\linewidth]{Figures/stack_pop_wd.pdf}
\caption{Methodology for the verification of well-definition}
\Label{fig:methodology-wd}
\end{figure}

\FloatBarrier

Note that the specifications of the various functions will explicitly
refer to the \emph{internal state} of \stacktype.
In \S\ref{sec:stack-verification} we will show that the
\emph{interplay} of these functions satisfy the stack axioms from
\S\ref{sec:stack-axioms}.


\clearpage


\subsection{The function \stackequal}
\Label{sec:stackequal}

The function \stackequal in the following listing
is  the runtime counterpart for the \logicref{StackEqual} predicate.
Note that this specifications explicitly refers to valid stacks.

\input{Listings/stack_equal.h.tex}

The implementation of \stackequal in the next listing
compares two stacks according to the same rules of predicate \StackEqual.

\input{Listings/stack_equal.c.tex}

\clearpage

\subsection{The function \stackinit}
\Label{sec:stackinit}

The following listing shows the specification of \stackinit.
Note that our specification of the post-conditions contains a redundancy
because a stack is empty if and only if its size is zero.

\input{Listings/stack_init.h.tex}

The next listing shows the implementation of \stackinit.
It simply initializes \inl{obj} and \inl{capacity} with the respective
value of the array and sets the field \inl{size} to zero.

\input{Listings/stack_init.c.tex}

\clearpage

\subsection{The function \stacksize}
\Label{sec:stacksize}
\Label{sec:stacksizewd}

The function \stacksize is the runtime version of the logic function
\logicref{StackSize}.
The specification of \stacksize in the following listing
simply states that \stacksize produces the same result as \StackSize.

\input{Listings/stack_size.h.tex}

As in the definition of the logic function \StackSize the implementation of
\stacksize in the next listing simply returns the field \inl{size}.

\input{Listings/stack_size.c.tex}

The next listing shows our check whether \stacksize is well-defined.
Since \stacksize neither modifies the state of its \stacktype argument 
nor that of any global variable we only check whether it
produces the same result for equal stacks.
Note that we simply may use operator~\inl{==} to compare integers since we
didn't introduce a nontrivial equivalence relation on that data type.

\input{Listings/stack_size_wd.c.tex}

\clearpage

\subsection{The function \stackfull}
\Label{sec:stackfull}

The function \stackfull is the runtime version of the predicate \logicref{StackFull}.

\input{Listings/stack_full.h.tex}

As in the definition of the predicate \StackFull the implementation of
\stackfull in the next listing simply checks whether the
size of the stack equals its capacity.

\input{Listings/stack_full.c.tex}

Note that with our definition of stack equality (\S\ref{sec:stack-equality})
there can be equal stack with different capacities.
As a consequence, there can are equal stacks where one is full while the other is not.
In other words, \stackfull is not well-defined!

\clearpage

\subsection{The function \stackempty}
\Label{sec:stackempty}
\Label{sec:stackemptywd}

The function \stackempty is the runtime version of the predicate \logicref{StackEmpty}.

\input{Listings/stack_empty.h.tex}

As in the definition of the predicate \StackEmpty the implementation of
\stackempty in the next listing simply checks whether the
size of the stack is zero.

\input{Listings/stack_empty.c.tex}

The following listing shows our check whether \stackempty is well-defined.

\input{Listings/stack_empty_wd.c.tex}

\clearpage

\subsection{The function \stacktop}
\Label{sec:stacktop}
\Label{sec:stacktopwd}

The function \stacktop is the runtime version of the logic function \logicref{StackTop}.
The specification of \stacktop in the following listing
simply states that for non-empty stacks
\stacktop produces the same result as \StackTop
which in turn just returns the element \inl{obj[size-1]} of \stacktype.

\input{Listings/stack_top.h.tex}

For a non-empty stack the implementation of \stacktop in
the next listing simply returns the element \inl{obj[size-1]}.
Note that our implementation of \stacktop does not crash when it is applied
to an empty stack.
In this case we return the first element of the internal,
non-empty array \inl{obj}.
This is consistent with our specification of \stacktop
which only refers to non-empty stacks.

\input{Listings/stack_top.c.tex}

The next listing shows our check whether \stacktop is well-defined.
Since our axioms in \S\ref{sec:stack-axioms} did not impose any
behavior on the behavior of \stacktop for empty stacks,
we prove the well-definition of \stacktop only for nonempty stacks.

\input{Listings/stack_top_wd.c.tex}



\subsection{The function \stackpush}
\Label{sec:stackpush}
\Label{sec:stackpushwd}

The following listing shows the specification of the function \stackpush.
In accordance with Axiom~\eqref{eq:stack-size-push}, \stackpush is supposed
to increase the number of elements of a non-full stack by one.
The specification also demands that the value that is pushed on a
non-full stack becomes the top element of the resulting stack (see 
Axiom~\eqref{eq:stack-top-push}).

\input{Listings/stack_push.h.tex}

The implementation of \stackpush is shown in the next listing.
It checks whether its argument is a non-full stack in which case it
increases the field \inl{size} by one but only after it has assigned
the function argument to the element \inl{obj[size]}.

\input{Listings/stack_push.c.tex}

\vfill

The following listing shows our formalization of the well-definition for \stackpush.
%
The function \stackpush does not return a value but rather modifies its argument.
For the well-definition of \stackpush we therefore check whether it
turns equal stacks into equal stacks.

\input{Listings/stack_push_wd.c.tex}

However, equality of the stack arguments is not sufficient for a proof
that \stackpush is well-defined.
We must also ensure that there is no \emph{aliasing} between the two stacks.
Otherwise modifying one stack could modify the other stack in unexpected ways.
In order to express that there is no aliasing between two stacks, 
we use the predicate \logicref{StackSeparated}.

In order to achieve an automatic verification of \implref{stackpushwd} we
have added the assertions \inl{top} and \equal and introduced the lemma
\logicref{StackPushEqual} in the following listing.

\input{Listings/StackLemmas.acsl.tex}

\clearpage

\subsection{The function \stackpop}
\Label{sec:stackpop}
\Label{sec:stackpopwd}

The following listing shows the specification of the function \stackpop.
In accordance with Axiom~\eqref{eq:stack-size-pop}, \stackpop is supposed
to reduce the number of elements in a non-empty stack by one.
In addition to the requirements imposed by the axioms,
our specification demands that \stackpop changes no memory location if it
is applied to an empty stack.

\input{Listings/stack_pop.h.tex}

The implementation of \stackpop is shown in the next listing.
It checks whether its argument is a non-empty stack in which case it
decreases the field \inl{size} by one.

\input{Listings/stack_pop.c.tex}

\clearpage

The next listing shows our check whether \stackpop is well-defined.
As in the case of \stackpush we use the predicate \logicref{StackSeparated}
in order to express that there is no aliasing between the two stack arguments.

\input{Listings/stack_pop_wd.c.tex}

\clearpage
\clearpage

\clearpage

\section{Verification of stack axioms}
\Label{sec:stack-verification}

In this section we show that the stack functions defined in
\S\ref{sec:stack-functions}
satisfy the stack Axioms of \S\ref{sec:stack-axioms}.

The annotated code has been obtained from the axioms in a fully systematical way.
In order to transform a condition equation $p \rightarrow s = t$:
\begin{itemize}
\item Generate a clause \inl{requires p}.
\item Generate a clause \inl{requires x1 == ... == xn} 
for each variable \inl{x} with $n$ occurrences 
in $s$ and $t$.
\item Change the $i$-th occurrence of \inl{x} to \inl{xi} 
in $s$ and $t$.
\item Translate both terms $s$ and $t$ to reversed polish notation.
\item Generate a clause \inl{ensures y1 == y2}, where \inl{y1} and
\inl{y2} denote the value corresponding to 
the translated $s$ and $t$, respectively.
\end{itemize}

This makes it easy to implement a tool that does the translation automatically,
but yields a slightly longer contract in our example.


\subsection{Resetting a stack}
\Label{sec:axiomsizeofinit}


Our formulation in \acsl\slash\isoc  of the axiom in
Equation~\eqref{eq:stack-init-size} is shown in  the following
listing.

\input{Listings/axiom_size_of_init.c.tex}

\clearpage

\subsection{Adding an element to a stack}
\Label{sec:axiomsizeofpush}
\Label{sec:axiomtopofpush}

Axioms~\eqref{eq:stack-size-push} and~\eqref{eq:stack-top-push}
describe the behavior of a stack when an element is added.

\input{Listings/axiom_size_of_push.c.tex}

Except for the \inl{assigns} clauses, the \acsl specification refers
only to encapsulating logic functions and predicates defined in
\S\ref{sec:stack-invariants}.
If \acsl would provide a means to define encapsulating logic
functions returning also sets of memory locations,
the expressions in \inl{assigns} clauses would not need to refer to
the details of our
\stacktype implementation.\footnote{
In \cite[\S 2.3.4]{ACSLSpec}, a powerful sublanguage to
build memory location set expressions is defined. 
We will explore its capabilities in a later version.
}
As an alternative, \inl{assigns} clauses could be omitted, as long as
the proofs are only used to convince a human reader.

\input{Listings/axiom_top_of_push.c.tex}

\clearpage

\subsection{Removing an element from a stack}
\Label{sec:axiompopofpush}
\Label{sec:axiomsizeofpop}
\Label{sec:axiompushofpoptop}

This section shows the Listings for Axioms~\ref{eq:stack-size-pop}, \ref{eq:stack-pop-push}
and~\ref{eq:stack-push-pop-top} which
describe the behavior of a stack when an element is removed.

\input{Listings/axiom_size_of_pop.c.tex}
\input{Listings/axiom_pop_of_push.c.tex}
\input{Listings/axiom_push_of_pop_top.c.tex}



