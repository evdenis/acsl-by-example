
\section{Verification of stack functions}
\Label{sec:stack-functions}

In this section we verify the functions
\begin{itemize}
\item \stackequal (\S\ref{sec:stackequal})
\item \stackinit (\S\ref{sec:stackinit})
\item \stacksize (\S\ref{sec:stacksize})
\item \stackfull (\S\ref{sec:stackfull})
\item \stackempty (\S\ref{sec:stackempty}) 
\item \stacktop (\S\ref{sec:stacktop})
\item \stackpush (\S\ref{sec:stackpush})
\item \stackpop (\S\ref{sec:stackpop})
\end{itemize}

of the data type \stacktype.
To be more precise, we provide for each of function \inl{stack_foo}:
\begin{itemize}
\item an \acsl specification of \inl{stack_foo} 
\item a \isoc implementation of \inl{stack_foo}
\item a \isoc function \inl{stack_foo_wd}\footnote{
  The suffix \inl{_wd} stands for \emph{well definition}
}
accompanied by a an \acsl contract that expresses that
the implementation of \inl{stack_foo} is well-defined.
Figure~\ref{fig:methodology-wd}
shows our methodology for the verification of
well-definition in the \inl{pop} example,
$(\approx)$ again indicating the user-defined \stacktype equality.
\end{itemize}

\begin{figure}[hbt]
\centering
\includegraphics[width=0.95\linewidth]{Figures/stack_pop_wd.pdf}
\caption{Methodology for the verification of well-definition}
\Label{fig:methodology-wd}
\end{figure}

\FloatBarrier

Note that the specifications of the various functions will explicitly
refer to the \emph{internal state} of \stacktype.
In \S\ref{sec:stack-verification} we will show that the
\emph{interplay} of these functions satisfy the stack axioms from
\S\ref{sec:stack-axioms}.


\clearpage


\subsection{The function \stackequal}
\Label{sec:stackequal}

The function \stackequal in the following listing
is  the runtime counterpart for the \logicref{StackEqual} predicate.
Note that this specifications explicitly refers to valid stacks.

\input{Listings/stack_equal.h.tex}

The implementation of \stackequal in the next listing
compares two stacks according to the same rules of predicate \StackEqual.

\input{Listings/stack_equal.c.tex}

\clearpage

\subsection{The function \stackinit}
\Label{sec:stackinit}

The following listing shows the specification of \stackinit.
Note that our specification of the post-conditions contains a redundancy
because a stack is empty if and only if its size is zero.

\input{Listings/stack_init.h.tex}

The next listing shows the implementation of \stackinit.
It simply initializes \inl{obj} and \inl{capacity} with the respective
value of the array and sets the field \inl{size} to zero.

\input{Listings/stack_init.c.tex}

\clearpage

\subsection{The function \stacksize}
\Label{sec:stacksize}
\Label{sec:stacksizewd}

The function \stacksize is the runtime version of the logic function
\logicref{StackSize}.
The specification of \stacksize in the following listing
simply states that \stacksize produces the same result as \StackSize.

\input{Listings/stack_size.h.tex}

As in the definition of the logic function \StackSize the implementation of
\stacksize in the next listing simply returns the field \inl{size}.

\input{Listings/stack_size.c.tex}

The next listing shows our check whether \stacksize is well-defined.
Since \stacksize neither modifies the state of its \stacktype argument 
nor that of any global variable we only check whether it
produces the same result for equal stacks.
Note that we simply may use operator~\inl{==} to compare integers since we
didn't introduce a nontrivial equivalence relation on that data type.

\input{Listings/stack_size_wd.c.tex}

\clearpage

\subsection{The function \stackfull}
\Label{sec:stackfull}

The function \stackfull is the runtime version of the predicate \logicref{StackFull}.

\input{Listings/stack_full.h.tex}

As in the definition of the predicate \StackFull the implementation of
\stackfull in the next listing simply checks whether the
size of the stack equals its capacity.

\input{Listings/stack_full.c.tex}

Note that with our definition of stack equality (\S\ref{sec:stack-equality})
there can be equal stack with different capacities.
As a consequence, there can are equal stacks where one is full while the other is not.
In other words, \stackfull is not well-defined!

\clearpage

\subsection{The function \stackempty}
\Label{sec:stackempty}
\Label{sec:stackemptywd}

The function \stackempty is the runtime version of the predicate \logicref{StackEmpty}.

\input{Listings/stack_empty.h.tex}

As in the definition of the predicate \StackEmpty the implementation of
\stackempty in the next listing simply checks whether the
size of the stack is zero.

\input{Listings/stack_empty.c.tex}

The following listing shows our check whether \stackempty is well-defined.

\input{Listings/stack_empty_wd.c.tex}

\clearpage

\subsection{The function \stacktop}
\Label{sec:stacktop}
\Label{sec:stacktopwd}

The function \stacktop is the runtime version of the logic function \logicref{StackTop}.
The specification of \stacktop in the following listing
simply states that for non-empty stacks
\stacktop produces the same result as \StackTop
which in turn just returns the element \inl{obj[size-1]} of \stacktype.

\input{Listings/stack_top.h.tex}

For a non-empty stack the implementation of \stacktop in
the next listing simply returns the element \inl{obj[size-1]}.
Note that our implementation of \stacktop does not crash when it is applied
to an empty stack.
In this case we return the first element of the internal,
non-empty array \inl{obj}.
This is consistent with our specification of \stacktop
which only refers to non-empty stacks.

\input{Listings/stack_top.c.tex}

The next listing shows our check whether \stacktop is well-defined.
Since our axioms in \S\ref{sec:stack-axioms} did not impose any
behavior on the behavior of \stacktop for empty stacks,
we prove the well-definition of \stacktop only for nonempty stacks.

\input{Listings/stack_top_wd.c.tex}



\subsection{The function \stackpush}
\Label{sec:stackpush}
\Label{sec:stackpushwd}

The following listing shows the specification of the function \stackpush.
In accordance with Axiom~\eqref{eq:stack-size-push}, \stackpush is supposed
to increase the number of elements of a non-full stack by one.
The specification also demands that the value that is pushed on a
non-full stack becomes the top element of the resulting stack (see 
Axiom~\eqref{eq:stack-top-push}).

\input{Listings/stack_push.h.tex}

The implementation of \stackpush is shown in the next listing.
It checks whether its argument is a non-full stack in which case it
increases the field \inl{size} by one but only after it has assigned
the function argument to the element \inl{obj[size]}.

\input{Listings/stack_push.c.tex}

\vfill

The following listing shows our formalization of the well-definition for \stackpush.
%
The function \stackpush does not return a value but rather modifies its argument.
For the well-definition of \stackpush we therefore check whether it
turns equal stacks into equal stacks.

\input{Listings/stack_push_wd.c.tex}

However, equality of the stack arguments is not sufficient for a proof
that \stackpush is well-defined.
We must also ensure that there is no \emph{aliasing} between the two stacks.
Otherwise modifying one stack could modify the other stack in unexpected ways.
In order to express that there is no aliasing between two stacks, 
we use the predicate \logicref{StackSeparated}.

In order to achieve an automatic verification of \implref{stackpushwd} we
have added the assertions \inl{top} and \equal and introduced the lemma
\logicref{StackPushEqual} in the following listing.

\input{Listings/StackLemmas.acsl.tex}

\clearpage

\subsection{The function \stackpop}
\Label{sec:stackpop}
\Label{sec:stackpopwd}

The following listing shows the specification of the function \stackpop.
In accordance with Axiom~\eqref{eq:stack-size-pop}, \stackpop is supposed
to reduce the number of elements in a non-empty stack by one.
In addition to the requirements imposed by the axioms,
our specification demands that \stackpop changes no memory location if it
is applied to an empty stack.

\input{Listings/stack_pop.h.tex}

The implementation of \stackpop is shown in the next listing.
It checks whether its argument is a non-empty stack in which case it
decreases the field \inl{size} by one.

\input{Listings/stack_pop.c.tex}

\clearpage

The next listing shows our check whether \stackpop is well-defined.
As in the case of \stackpush we use the predicate \logicref{StackSeparated}
in order to express that there is no aliasing between the two stack arguments.

\input{Listings/stack_pop_wd.c.tex}

\clearpage
\clearpage

