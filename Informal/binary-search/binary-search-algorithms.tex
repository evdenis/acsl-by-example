
\chapter{Binary search algorithms}
\Label{cha:binary-search}

In this chapter, we consider the four
\emph{binary search} algorithms of the \cxx Standard Library \cite[\S
28.7.3]{cxx-17-draft}, namely

\begin{itemize}
\item \lowerbound in \S\ref{sec:lowerbound}

\item \upperbound in \S\ref{sec:upperbound}

\item two variants for the implementation of  \equalrange in \S\ref{sec:equalrange}

\item two variants for the formal specification of \binarysearch in \S\ref{sec:binarysearch}
\end{itemize}

As in the case of the of maximum/minimum algorithms from Chapter~\ref{cha:maxmin}
the binary search algorithms primarily use the less-than operator~\inl{<}
(and the derived operators \inl{<=}, \inl{>} and \inl{>=}) to determine whether a particular
value is contained in an increasing range.
Thus, different to the \find algorithm in \S\ref{sec:find},
the equality operator~\inl{==} will play only a supporting part
in the specification of binary search.

In order to make the specifications of the binary search algorithms 
more compact and (arguably) more readable we re-use the predicates
\logicref{LowerBound}, \logicref{StrictLowerBound},
\logicref{UpperBound}, and \logicref{StrictUpperBound}.

All binary search algorithms require that their input array is arranged in 
increasing order.
The following listing shows two versions of predicate \logicref{Increasing}.
The first one defines when a section of an array is in increasing order.
The second version uses the first one to express that the whole array is in increasing order.

\input{Listings/Increasing.acsl.tex}

%\clearpage

There is also the overloaded predicate \logicref{WeaklyIncreasing} that we will user for 
the verification of other algorithms.

\input{Listings/WeaklyIncreasing.acsl.tex}

Users inexperienced in formal verification often have a blind spot at the
difference between \Increasing and \WeaklyIncreasing.
%
Both versions are logically equivalent,
and proving that \Increasing implies \WeaklyIncreasing is even trivial.
%
However, proving the converse direction is not, and requires an induction on the
array size~\inl{n}, employing the transitivity of \inl{<=} in the induction step.
%
Humans are trained to perform such inductions unnoticed,
but none of the automated provers supported by \framac is able to perform induction.
%
The following Listing contains several lemmas on the relationship of
\WeaklyIncreasing and \Increasing.

\input{Listings/IncreasingLemmas.acsl.tex}

\clearpage

We usually exploit the relationship of the predicates \Increasing and \WeaklyIncreasing
in the following way:

\begin{itemize}
\item We use the predicate \Increasing in the preconditions and postconditions of
      function contracts.
\item The \WeaklyIncreasing is employed for assertions and loop invariants
      whenever we have to verify that an algorithm (typically a sorting algorithm)
      produces an increasing array.
\item Finally, to conclude that a \emph{weakly increasing} array is in fact \emph{increasing}
      we rely on lemma\\
      \logicref{WeaklyIncreasingIncreasing} .
\end{itemize}

\clearpage


\section{The \lowerbound algorithm}
\Label{sec:lowerbound}

The \lowerbound algorithm is one of the four binary search algorithms
of the \cxx Standard Library \cite[\S 28.7.3.1]{cxx-17-draft}.
For our purposes we have modified
the generic implementation
to that of an array of type \valuetype.
The signature now reads:

\begin{lstlisting}[style = acsl-block]

  size_type
  lower_bound(const value_type* a, size_type n, value_type v);
\end{lstlisting}

As with the other binary search algorithms \lowerbound requires that
its input array is in increasing order.
The index~\inl{lb}, that \lowerbound returns satisfies the inequality

\begin{align}
\Label{eq:lower-bound-result}
0 \leq \mathtt{lb} \leq n  
\end{align}

and has the following properties for a valid index~\inl{k} of the array under consideration

\begin{alignat}{3}
\Label{eq:lower-bound-left}
0 &\leq k < \mathtt{lb} && \qquad\Longrightarrow\qquad && a[k] < \mathtt{v} \\
\Label{eq:lower-bound-right}
\mathtt{lb} &\leq k < n && \qquad\Longrightarrow\qquad && \mathtt{v} \leq a[k]
\end{alignat}

Conditions~\eqref{eq:lower-bound-left} and~\eqref{eq:lower-bound-right} imply that~\inl{v}
can only occur in the array section \inl{a[lb..n-1]}.
In this sense \lowerbound returns a \emph{lower bound} for the potential indices.

As an example, we consider in Figure~\ref{fig:lowerbound} an increasingly ordered array.
The arrows indicate which indices will be returned by \lowerbound for a given value.
Note that the index~9 points \emph{one past end} of the array.
Values that are not contained in the array are colored in gray.

\begin{figure}[hbt]
\centering
\includegraphics[width=0.60\textwidth]{Figures/lower_bound.pdf}
\caption{\Label{fig:lowerbound}Some examples for \lowerbound}
\end{figure}

%\FloatBarrier

Figure~\ref{fig:lowerbound} also clarifies that care must
be taken when interpreting the return value of \lowerbound.
%
An important difference to the algorithms in Chapter~\ref{cha:non-mutating}
is that a return value of \lowerbound that is less than~$n$ 
does not necessarily implies \inl{a[lb] == v}.
We can only be sure that \inl{v <= a[lb]} holds.


\subsection{Formal specification of \lowerbound}

The specification of \specref{lowerbound} is shown in the following listing.
The preconditions \inl{increasing} expresses
that the array values need to be in increasing order.
%
The postconditions reflect the conditions listed above and can be expressed
using the predicates \logicref{LowerBound} and \logicref{StrictUpperBound}.

\begin{itemize}
\item Condition~\eqref{eq:lower-bound-result} becomes postcondition \inl{result}
\item Condition~\eqref{eq:lower-bound-left} becomes postcondition \inl{left}
\item Condition~\eqref{eq:lower-bound-right} becomes postcondition \inl{right}
\end{itemize}

\input{Listings/lower_bound.h.tex}

\subsection{Implementation of \lowerbound}
\Label{subsec:lowerbound:impl}

The following listing shows our implementation of \implref{lowerbound}.
Each iteration step narrows down the range that contains the
sought-after result. 
The loop invariants express that in each iteration step all indices
less than the temporary left bound
\inl{left} contain values that are less than \inl{v} and all indices
not less than the temporary right bound \inl{right} contain values
that are greater or equal than \inl{v}.
%
The expression to compute \inl{middle} is slightly more complex than the
naïve \inl{(left+right)/2}, but it avoids potential overflows.

\input{Listings/lower_bound.c.tex}



\section{The \upperbound algorithm}
\Label{sec:upperbound}

The \upperbound algorithm of the \cxx Standard Library \cite[\S 28.7.3.2]{cxx-17-draft} is a
variant of binary search and closely related to \specref{lowerbound}.
The signature reads:

\begin{lstlisting}[style = acsl-block]

  size_type 
  upper_bound(const value_type* a, size_type n, value_type v)
\end{lstlisting}

As with the other binary search algorithms, \upperbound requires that
its input array is in increasing order.
The index~\inl{ub} returned by \upperbound satisfies the inequality

\begin{align}
\Label{eq:upper-bound-result}
0 \leq \mathtt{ub} \leq n  
\end{align}

and is involved in the following implications for a valid index~\inl{k} of the array
under consideration

\begin{alignat}{3}
\Label{eq:upper-bound-left}
0 &\leq k < \mathtt{ub} &&\qquad\Longrightarrow\qquad && a[k] \leq \mathtt{v} \\
\Label{eq:upper-bound-right}
\mathtt{ub} &\leq k < n &&\qquad\Longrightarrow\qquad &&\mathtt{v} < a[k]
\end{alignat}

Conditions~\eqref{eq:upper-bound-left} and~\eqref{eq:upper-bound-right} imply that~\inl{v}
can only occur in the array section \inl{a[0..ub-1]}.
In this sense \upperbound returns a \emph{upper bound} for the
potential indices where \inl{v} can occur.
It also means that the searched-for value \inl{v} can
\emph{never} be located at the index~\inl{ub}.

Figure~\ref{fig:upperbound} is a variant of Figure~\ref{fig:lowerbound} for the case
of \upperbound and the same example array.
The arrows indicate which indices will be returned by \upperbound for a given value.
Note how, compared to Figure~\ref{fig:lowerbound}, only the arrows from values
that \emph{are present} in the array change their target index.

\begin{figure}[hbt]
\centering
\includegraphics[width=0.60\textwidth]{Figures/upper_bound.pdf}
\caption{\Label{fig:upperbound}Some examples for \upperbound}
\end{figure}

\FloatBarrier

\subsection{Formal specification of \upperbound}

The following listing shows the specification of \specref{upperbound} which
is quite similar to the specification of \specref{lowerbound}.
The precondition \inl{increasing} expresses
that the array values need to be in increasing order.

The postconditions reflect the conditions listed above and can be expressed
using predicates \logicref{UpperBound} and \logicref{StrictLowerBound}, namely,


\begin{itemize}
\item Condition~\eqref{eq:upper-bound-result} becomes postcondition \inl{result}
\item Condition~\eqref{eq:upper-bound-left} becomes postcondition \inl{left}
\item Condition~\eqref{eq:upper-bound-right} becomes postcondition \inl{right}
\end{itemize}

\input{Listings/upper_bound.h.tex}

\subsection{Implementation of \upperbound}

Our implementation of \implref{upperbound} is shown in the following listing.
The loop invariants express that for each iteration step all indices less than 
the temporary left bound \inl{left} contain values not greater than \inl{v}
and all indices not less than the temporary right bound \inl{right} contain
values greater than \inl{v}.

\input{Listings/upper_bound.c.tex}

\clearpage


\section{The \equalrange algorithm}
\Label{sec:equalrange}
\Label{sec:equalrangeii}

The \equalrange algorithm is one of the four binary search algorithms
of the \cxx Standard Library \cite[\S 28.7.3.3]{cxx-17-draft}.
As with the other binary search algorithms \equalrange requires that
its input array is in increasing order.
The specification of \equalrange states that it \emph{combines} the results of the algorithms
\specref{lowerbound} and \specref{upperbound}.

For our purposes we have modified
\equalrange
to take an array of type \valuetype.
Moreover, instead of a pair of iterators, our version returns a pair of indices.
To be more precise, the return type of \equalrange is the
struct \sizetypepair from Listing~\ref{lst:size_type_pair}.
Thus, the signature of \equalrange now reads:

\begin{lstlisting}[style = acsl-block]

  size_type_pair
  equal_range(const value_type* a, size_type n, value_type v);
\end{lstlisting}

Figure~\ref{fig:equalrange} combines Figure~\ref{fig:lowerbound} with Figure~\ref{fig:upperbound}
in order visualize the behavior of \equalrange for select test cases.
The two types of arrows~$\rightarrow$ and~$\dashrightarrow$ represent the
respective fields \inl{first} and \inl{second} of the return value.
For values that are not contained in the array, the two arrows point to the same index.
More generally, if \equalrange returns the pair $(\mathtt{lb},\mathtt{ub})$, then
the difference $\mathtt{ub} - \mathtt{lb}$ is equal to the number of occurrences of the 
argument \inl{v} in the array.

\begin{figure}[hbt]
\centering
\includegraphics[width=0.60\textwidth]{Figures/equal_range.pdf}
\caption{\Label{fig:equalrange}Some examples for \equalrange}
\end{figure}

\FloatBarrier

We will provide two implementations of \equalrange and verify both of them.
The first implementation \implref{equalrange} just straightforwardly
calls \specref{lowerbound} and \specref{upperbound} and simply
returns the pair of their respective results.
The second implementation \implref{equalrangeii}, which is more elaborate, follows the
original \cxx code by attempting to minimize duplicate computations.
%
Let $(\mathtt{lb}, \mathtt{ub})$ be the return value  \equalrange, then
the conditions~\eqref{eq:lower-bound-result}--\eqref{eq:upper-bound-right} can
be merged into the inequality
%
\begin{align}
\Label{eq:equal-range-result}
0 \leq \mathtt{lb} \leq \mathtt{ub} \leq n 
\end{align}

and the following three implications for a valid index $k$ of the array under
consideration
%
\begin{alignat}{3}
\Label{eq:equal-range-left}
0           &\leq  k < \mathtt{lb} && \qquad\Longrightarrow\qquad &&  a[k] < \mathtt{v} \\
\Label{eq:equal-range-middle}
\mathtt{lb} &\leq k < \mathtt{ub} && \qquad\Longrightarrow\qquad &&  a[k] = \mathtt{v} \\
\Label{eq:equal-range-right}
\mathtt{ub} &\leq k < n           && \qquad\Longrightarrow\qquad &&  a[k] > \mathtt{v} 
\end{alignat}

%\clearpage

Here are some justifications for these conditions.

\begin{itemize}
\item 
Conditions~\eqref{eq:equal-range-left} and~\eqref{eq:equal-range-right} are just the 
Conditions~\eqref{eq:lower-bound-left} and~\eqref{eq:upper-bound-right}, respectively.

\item
The Inequality~\eqref{eq:equal-range-result} follows from the Inequalities~\eqref{eq:lower-bound-result}
and~\eqref{eq:upper-bound-result} and the following considerations:
If $\mathtt{ub}$ were less than $\mathtt{lb}$, then according to~\eqref{eq:equal-range-left}
we would have $a[\mathtt{ub}] < \mathtt{v}$.
One the other hand, we know from~\eqref{eq:equal-range-right} that opposite
inequality $\mathtt{v} < a[\mathtt{ub}]$ holds.
Therefore, we have $\mathtt{lb} \leq \mathtt{ub}$.

\item
Condition~\eqref{eq:equal-range-middle} follows from the combination of~\eqref{eq:lower-bound-right}
and~\eqref{eq:upper-bound-left} and the fact that~$\leq$ is a total order on the integers.
\end{itemize}

%\clearpage

\subsection{Formal specification of \equalrange}

The following listing show the specification of \specref{equalrange}
(and of \equalrangeii).

\input{Listings/equal_range.h.tex}

The precondition \inl{increasing} expresses
that the array values need to be in increasing order.

The postconditions reflect the conditions listed above and can be expressed
using the already introduced predicates
\logicref{AllEqual},
\logicref{StrictUpperBound} and \logicref{StrictLowerBound}.

\begin{itemize}
\item Condition~\eqref{eq:equal-range-result} becomes postcondition \inl{result}
\item Condition~\eqref{eq:equal-range-left} becomes postcondition \inl{left}
\item Condition~\eqref{eq:equal-range-middle} becomes postcondition \inl{middle}
\item Condition~\eqref{eq:equal-range-right} becomes postcondition \inl{right}
\end{itemize}

%\clearpage 

\subsection{Implementation of \equalrange}

Our first implementation of \implref{equalrange} is shown in the following listing.
We just call the two functions \specref{lowerbound} and \specref{upperbound}
and return their respective results as a pair.

\input{Listings/equal_range.c.tex}

In a very early version of this document we had proven the similar assertion
\inl{first <= second} with the interactive theorem prover \coq.
After reviewing this proof we formulated the new assertion \inl{aux}
that uses a fact from the postcondition of \specref{upperbound}.
The benefit of this reformulation is that both the assertion \inl{aux}
and the postcondition \inl{first <= second} can now be verified automatically.

%\clearpage 

\subsection{Implementation of \equalrangeii}

The first implementation of \implref{equalrange} does more work than needed.
In the following listing \implref{equalrangeii} we show that it is possible to
perform as much range reduction as possible before calling \specref{upperbound}
and \specref{lowerbound} on the reduced ranges.


\input{Listings/equal_range2.c.tex}

Due to the higher code complexity of the second implementation, additional
assertions had to be inserted in order to ensure that \wpframac is able to verify the
correctness of the code.
All of these assertions are related to pointer arithmetic and shifting base pointers.
They fall into three groups and are briefly discussed below.
In order to enable the automatic verification of these properties we
added the following collection of \logicref{ArrayBoundsShift}.

\input{Listings/ShiftLemmas.acsl.tex}

\subsubsection*{The \inl{increasing} properties}

Both \specref{lowerbound} and \specref{upperbound} expect that they
operate on increasingly ordered arrays.
This is of course also true for \specref{equalrange}, however,
inside our second implementation we need a more specific formulation, namely,

\begin{lstlisting}[style=acsl-block]

        Increasing(a + middle, last - middle)
\end{lstlisting}

With the three-argument form of predicate \logicref{Increasing}
we can formulate out an intermediate step.
This enables the provers to verify the preconditions of the call to
\specref{lowerbound} automatically.
A similar assertion is present before the call to \specref{upperbound}.

%\clearpage

\subsubsection*{The \inl{strict} and \inl{constant} properties}

Part of the post conditions of \specref{equalrange} is that \inl{v}
is both a strict upper and a strict lower bound.
However, the calls to \lowerbound and \upperbound only give us

\begin{lstlisting}[style=acsl-block]

        StrictUpperBound(a + first, 0, left - first, v) 

        StrictLowerBound(a + middle, right - middle, last - middle, v)
\end{lstlisting}

which is not enough to reach the desired post conditions automatically.
One intermediate step for each of the assertions was sufficient to guide
the prover to the desired result.

Conceptually similar to the \inl{strict} properties
the \inl{constant} properties guide the prover towards

\begin{lstlisting}[style=acsl-block]

        LowerBound(a, left, n, v) 

        UpperBound(a, 0, right, v)
\end{lstlisting}

Combining these properties allow the postcondition \inl{middle} to be derived automatically.

%\clearpage


\section{The \binarysearch algorithm}
\Label{sec:binarysearch}
\Label{sec:binarysearchii}


The \binarysearch algorithm is one of the four binary search
algorithms of the \cxx Standard Library \cite[\S 28.7.3.4]{cxx-17-draft}.
For our purposes we have modified
the generic implementation
to that of an array of type \valuetype.
The signature now reads:

\begin{lstlisting}[style = acsl-block]
  bool binary_search(const value_type* a, size_type n, value_type  v);
\end{lstlisting}

Again, \binarysearch requires that its input array is in increasing order.
It will return \inl{true} if there exists an index~\inl{i} 
in~\inl{a} such that \inl{a[i] == v} holds.\footnote{%
   To be more precise: The \cxx Standard Library requires that 
   \inl{(a[i] <= v)  && (v <= a[i])} holds.
   For our definition of \valuetype (see \S\ref{sec:frequentPattern}) this
   means that \inl{v} equals \inl{a[i]}.
}

\begin{figure}[hbt]
\centering
\includegraphics[width=0.60\textwidth]{Figures/binary_search.pdf}
\caption{\Label{fig:binarysearch}Some examples for \binarysearch}
\end{figure}

\FloatBarrier

In Figure~\ref{fig:binarysearch} we do not need to use arrows to visualize the
effects of \binarysearch.
The colors orange and grey of the sought-after values indicate whether the algorithm
returns true or false, respectively.

\subsection{Formal specification of \binarysearch and \binarysearchii}

The \acsl specification of \specref{binarysearch} is shown in the following listing.

\input{Listings/binary_search.h.tex}

Note that instead of the somewhat lengthy existential quantification
of \specref{binarysearch} we can use our previously introduced predicate
\logicref{SomeEqual} in order to achieve the following more concise
formal specification \specref{binarysearchii}.


\input{Listings/binary_search2.h.tex}

It is interesting to compare the specification of \specref{binarysearch}
with that of \specref{findii}.
Both algorithms allow to determine whether a value is contained in an array.
The fact that the \cxx Standard Library requires that \find has
\emph{linear} complexity whereas \binarysearch must have a
\emph{logarithmic} complexity can currently not be expressed with \acsl.


\subsection{Implementation of \binarysearch}

Our implementation \implref{binarysearchii} first calls \specref{lowerbound}.
Remember that if the latter returns an index \inl{0 <= i < n},
then we can be sure that \inl{v <= a[i]} holds.

\input{Listings/binary_search2.c.tex}



