
\section{The \upperbound algorithm}
\Label{sec:upperbound}

The \upperbound algorithm of the \cxx Standard Library \cite[\S 28.7.3.2]{cxx-17-draft} is a
variant of binary search and closely related to \specref{lowerbound}.
The signature reads:

\begin{lstlisting}[style = acsl-block]

  size_type 
  upper_bound(const value_type* a, size_type n, value_type v)
\end{lstlisting}

As with the other binary search algorithms, \upperbound requires that
its input array is in increasing order.
The index~\inl{ub} returned by \upperbound satisfies the inequality

\begin{align}
\Label{eq:upper-bound-result}
0 \leq \mathtt{ub} \leq n  
\end{align}

and is involved in the following implications for a valid index~\inl{k} of the array
under consideration

\begin{alignat}{3}
\Label{eq:upper-bound-left}
0 &\leq k < \mathtt{ub} &&\qquad\Longrightarrow\qquad && a[k] \leq \mathtt{v} \\
\Label{eq:upper-bound-right}
\mathtt{ub} &\leq k < n &&\qquad\Longrightarrow\qquad &&\mathtt{v} < a[k]
\end{alignat}

Conditions~\eqref{eq:upper-bound-left} and~\eqref{eq:upper-bound-right} imply that~\inl{v}
can only occur in the array section \inl{a[0..ub-1]}.
In this sense \upperbound returns a \emph{upper bound} for the
potential indices where \inl{v} can occur.
It also means that the searched-for value \inl{v} can
\emph{never} be located at the index~\inl{ub}.

Figure~\ref{fig:upperbound} is a variant of Figure~\ref{fig:lowerbound} for the case
of \upperbound and the same example array.
The arrows indicate which indices will be returned by \upperbound for a given value.
Note how, compared to Figure~\ref{fig:lowerbound}, only the arrows from values
that \emph{are present} in the array change their target index.

\begin{figure}[hbt]
\centering
\includegraphics[width=0.60\textwidth]{Figures/upper_bound.pdf}
\caption{\Label{fig:upperbound}Some examples for \upperbound}
\end{figure}

\FloatBarrier

\subsection{Formal specification of \upperbound}

The following listing shows the specification of \specref{upperbound} which
is quite similar to the specification of \specref{lowerbound}.
The precondition \inl{increasing} expresses
that the array values need to be in increasing order.

The postconditions reflect the conditions listed above and can be expressed
using predicates \logicref{UpperBound} and \logicref{StrictLowerBound}, namely,


\begin{itemize}
\item Condition~\eqref{eq:upper-bound-result} becomes postcondition \inl{result}
\item Condition~\eqref{eq:upper-bound-left} becomes postcondition \inl{left}
\item Condition~\eqref{eq:upper-bound-right} becomes postcondition \inl{right}
\end{itemize}

\input{Listings/upper_bound.h.tex}

\subsection{Implementation of \upperbound}

Our implementation of \implref{upperbound} is shown in the following listing.
The loop invariants express that for each iteration step all indices less than 
the temporary left bound \inl{left} contain values not greater than \inl{v}
and all indices not less than the temporary right bound \inl{right} contain
values greater than \inl{v}.

\input{Listings/upper_bound.c.tex}

\clearpage

