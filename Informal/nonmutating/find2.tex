
\section{The \findii algorithm---reuse of specification elements}
\Label{sec:findii}

In this section we specify \find in a slightly different way.
Our approach is motivated by a considerable number of closely related \acsl formulas
in the contract \specref{find} and the implementation \implref{find}.

\begin{lstlisting}[style=acsl-block]

    \exists integer i; 0 <= i < n        &&   a[i] == v;

    \forall integer i; 0 <= i < \result  ==>  a[i] != v;

    \forall integer i; 0 <= i < n        ==>  a[i] != v;

    \forall integer k; 0 <= k < i        ==>  a[k] != v;
\end{lstlisting}

Note that the first formula is the negation of the third one.

\subsection{The predicates \SomeEqual and \NoneEqual}

In order to be more explicit about the commonalities of these formulas
we define a predicate, called 
\logicref{SomeEqual},
which describes the situation that there is a valid index \inl{i} 
where~\inl{a[i]} equals~\inl{v}.

\input{Listings/SomeNone.acsl.tex}

We first remark that the \SomeEqual, its negation \NoneEqual
and the lemmas \NotSomeEqualNoneEqual and \NoneEqualNotSomeEqual are encapsulated
in the \emph{axiomatic block} 
\logicref{SomeNone}.
This is a \emph{feeble} attempt to establish some modularization for the various predicates,
logic functions and lemmas.
We say \emph{feeble} because axiomatic blocks are, in contrast to \acsl \inl{module}s,
\emph{not} name spaces.
\acsl modules, however, are not yet implemented by \framac.

We also remark that both predicates come in overloaded versions.
The first of theses versions is a definition for array sections while the
second definition is for the case of complete arrays.

Note that we have provided a label, viz.\ \inl{A}, to the
predicate \SomeEqual.
Its purposes to express that the evaluation of the predicate depends on a memory state,
viz.\ the contents of \inl{a[0..n-1]}.
In general, we have to write

\begin{lstlisting}[style=acsl-block]

    \exists integer i; 0 <= i < n && \at(a[i],A) == v;
\end{lstlisting}

in order to express that we refer to the value \inl{a[i]} in
the program state~\inl{A}.
However, \acsl allows to abbreviate \inl{\\at(a[i],A)} by \inl{a[i]} if, as in
\SomeEqual or \NoneEqual, the label~\inl{A} is the only available label.
In particular, we have omitted the label in the overloaded versions for complete arrays.

%\clearpage

\subsection{Formal specification of \findii}

With the predicates \logicref{SomeEqual}
and \logicref{NoneEqual}
we are able to encapsulate all uses of the universal and existential 
quantifiers in both the specification and implementation of \findii.

As a result, the revised contract \specref{findii} is more concise
than that of \specref{find}.
%
In particular, it can be seen immediately that the conditions in the
assumes clauses
of the two behaviors \inl{some} and \inl{none} are mutually
exclusive since
one is the literal negation of the other.
Moreover, the requirement that \find returns the smallest index can
also be expressed
using the \logicref{NoneEqual} predicate, as depicted with the last postcondition of
behavior \inl{some}.

\input{Listings/find2.h.tex}

We also enriched the specification of \find by user-defined names
(sometimes called \emph{labels}, too, the distinction to program state identifiers
being obvious)  to refer to
the \inl{requires} and \inl{ensures} clauses.
We highly recommend this practice in particular for more complex annotations.
For example, \framac can be instructed to verify only clauses with a
given name.

\clearpage

\subsection{Implementation of \findii}

The predicate \NoneEqual is also used in the loop annotation inside
the implementation of \implref{findii}.
Note that, as in the case of the specification, we use labels to name individual annotations.

\input{Listings/find2.c.tex}

%\clearpage

