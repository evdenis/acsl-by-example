
\section{The \adjacentfind algorithm}
\Label{sec:adjacentfind}

The \adjacentfind
algorithm of the \cxx Standard Library \cite[\S 28.5.8]{cxx-17-draft}

\begin{lstlisting}[style=acsl-block]

  size_type adjacent_find(const value_type* a, size_type n);
\end{lstlisting}

returns the smallest valid index i, such that i+1 is also a valid index 
and such that 
\begin{lstlisting}[style=acsl-block]

    a[i] == a[i+1]

\end{lstlisting}
holds. 
The \adjacentfind algorithm returns \inl{n} if no such index exists.


The arrow in Figure~\ref{fig:adjacentfind} indicates the smallest index
where two adjacent elements are equal.

\begin{figure}[hbt]
\centering
\includegraphics[width=0.60\textwidth]{Figures/adjacent_find.pdf}
\caption{\Label{fig:adjacentfind}A simple example for \adjacentfind}
\end{figure}

\FloatBarrier


\subsection{The predicate \HasEqualNeighbors}

As in the case of other search algorithms, we first define 
a predicate 
\logicref{HasEqualNeighbors}
that captures 
the essence of finding two adjacent indices at which the array holds equal values.

\begin{logic}[hbt]
\begin{minipage}{0.99\textwidth}
\lstinputlisting[style=acsl-block, frame=single]{Source/HasEqualNeighbors.acsl}
\end{minipage}
\caption{\Label{logic:HasEqualNeighbors}The predicate \HasEqualNeighbors}
\end{logic}

\subsection{Formal specification of \adjacentfind}

We use the predicate 
\logicref{HasEqualNeighbors}
to define the formal specification
of \specref{adjacentfind}.

\input{Listings/adjacent_find.h.tex}

\subsection{Implementation of \adjacentfind}

In the implementation of \implref{adjacentfind} we check whether the array
contains at least two elements.
Otherwise, there is no point in looking for adjacent neighbors.
Note the use of the predicate \logicref{HasEqualNeighbors}
 in the loop invariant to
match the similar postcondition of behavior \inl{some}.

\input{Listings/adjacent_find.c.tex}

