
\chapter{Non-mutating algorithms}
\Label{cha:non-mutating}

\Label{assigns-clause}
In this chapter, we consider \emph{non-mutating} algorithms
of the \cxx Standard Library \cite[\S 28.5]{cxx-17-draft}.
These algorithms neither change their arguments nor any objects outside their scope.
This requirement can be formally expressed with the following 
\emph{assigns clause}:
\begin{lstlisting}[style=acsl-block]

  assigns \nothing;
\end{lstlisting}

Each algorithm in this chapter therefore uses this assigns clause
in its specification.

The specifications of these algorithms are not very complex.
Nevertheless, we have tried to arrange them so that the earlier
examples are simpler than the later ones. 
Each algorithm works on one-dimensional arrays.

\begin{itemize}
\item \find in \S\ref{sec:find}
  provides \emph{sequential} or \emph{linear search}
  and returns the smallest index at which a given value occurs in a given range.
  In \S\ref{sec:findii}, a user-defined \acsl predicate is introduced in order to 
  simplify the reuse of various specification elements.
  We refer to the simplified version as \findii.
  We provide in \S\ref{sec:findiii} a third specification of \find (called \findiii)
  that relies on a user-defined \acsl function that expresses the ideas of 
  linear search on the logic level.

\item \findifnot in \S\ref{sec:findifnot} is a small variation of \find
that searches the first occurrence where a given value does \emph{not} occur.

\item \findfirstof in \S\ref{sec:findfirstof} 
  provides similar to \find a \emph{sequential search}.
  However, unlike \find 
  it does not search for a particular value, 
        but for an arbitrary member of a set.

\item \adjacentfind in \S\ref{sec:adjacentfind}
can be used to find equal neighbors in an array.

\item \equal and \mismatch in \S\ref{sec:equal} are useful for
comparing two ranges element-by-element and identifying where they differ. 

\item \search and \searchn in \S\ref{sec:search} and~\S\ref{sec:searchn}
find a subsequence that is identical to a given sequence
when compared element-by-element and returns the position of the first occurrence.

\item \counti in \S\ref{sec:counti} returns
the number of occurrences of a given value in a range.
Here we will explicitly define a logic function for elements
counting and show that the implementation comply with it.

\item \countii in \S\ref{sec:countii} contains
different specification for the \counti function. In this case an
inductive predicate defined for elements counting. The section allows
one to compare different approaches of writing specifications and
demonstrates the \acsl inductive predicates.

\end{itemize}

\clearpage


\section{The \find algorithm}
\Label{sec:find}

The \find algorithm in the \cxx Standard Library \cite[\S 28.5.5]{cxx-17-draft}
implements \emph{sequential search} for general sequences.
We have modified the generic implementation,
which relies heavily on \cxx templates, to that of a range of
type \valuetype.
The signature now reads:

\begin{lstlisting}[style=acsl-block]

       size_type find(const value_type* a, size_type n, value_type v);
\end{lstlisting}

The function \find returns the least \emph{valid} index \inl{i} of \inl{a}
where the condition \inl{a[i] == v} holds. 
If no such index exists then \find returns the length \inl{n} of the array.


As an example, we consider in Figure~\ref{fig:find} an array.
The arrows indicate which indices will be returned by \find for a given value.
Note that the index~9 points \emph{one past end} of the array.
Values that are not contained in the array are colored in gray.

\begin{figure}[hbt]
\centering
\includegraphics[width=0.60\textwidth]{Figures/find.pdf}
\caption{\Label{fig:find}Some simple examples for \find}
\end{figure}

\FloatBarrier


\subsection{Formal specification of \find}

The following listing shows our first attempt specify \specref{find}.

\input{Listings/find.h.tex}

The \inl{requires}-clause indicates that \inl{n} is non-negative and
that the pointer \inl{a} points to $n$~contiguously allocated objects of type
\valuetype (see~\S\ref{sec:frequentPattern}).
%
The \inl{assigns}-clause indicates that \find (as a non-mutating algorithm),
does not modify any memory location outside its scope (see~Page~\pageref{assigns-clause}).

Generally, we only know that \find returns a non-negative index that is less or
equal the length of the array.
However, once we assume more specific situations,
we can also make more precise statements about the returned valued.
This is the reason why we have subdivided the specification of
\find into two behaviors (named \inl{some} and \inl{none}).

\begin{itemize}
\item
The behavior \inl{some} applies if the sought-after value is contained in the array.
We express this condition by using the \inl{assumes}-clause.
The next line expresses that if the assumptions of the behavior are satisfied then
\find will return a valid index.
The algorithm also ensures that  the returned (valid) index \inl{i},
\inl{a[i] == v} holds.
Therefore we define this property in the second postcondition of behavior \inl{some}.
Finally, it is important to express that \find returns the smallest index~\inl{i}
for which \inl{a[i] == v} holds (see last postcondition of behavior \inl{some}).

\item
The behavior \inl{none} covers the case that the sought-after value 
is \emph{not} contained in the array (see \inl{assumes}-clause of behavior \inl{none} in
in the contract of\specref{find}.
In this case, \find must return the length \inl{n} of the range \inl{a}.
\end{itemize}

Note that the formula in the \inl{assumes}-clause of the behavior \inl{some}
is the negation of the \inl{assumes}-clause of the behavior \inl{none}.
Therefore, we can express  that these two behaviors are
\emph{complete} and \emph{disjoint}.

\subsection{Implementation of \find}

The noteworthy elements of our implementation of \implref{find} are the 
\emph{loop annotations}.
%
The first loop \emph{invariant} is needed to prove that accesses
to~\inl{a} only occur with valid indices. The second  loop \emph{invariant} is needed
for the proof of the postconditions of the
behavior~\inl{some} in the contract of \specref{find}.
It expresses that for each iteration the sought-after value is not
yet found up to that iteration step.
Finally, the  loop \emph{variant} \inl{n-i} is needed to generate correct verification
conditions for the termination of the loop.

\input{Listings/find.c.tex}

\clearpage



\section{The \findii algorithm---reuse of specification elements}
\Label{sec:findii}

In this section we specify \find in a slightly different way.
Our approach is motivated by a considerable number of closely related \acsl formulas
in the contract \specref{find} and the implementation \implref{find}.

\begin{lstlisting}[style=acsl-block]

    \exists integer i; 0 <= i < n        &&   a[i] == v;

    \forall integer i; 0 <= i < \result  ==>  a[i] != v;

    \forall integer i; 0 <= i < n        ==>  a[i] != v;

    \forall integer k; 0 <= k < i        ==>  a[k] != v;
\end{lstlisting}

Note that the first formula is the negation of the third one.

\subsection{The predicates \SomeEqual and \NoneEqual}

In order to be more explicit about the commonalities of these formulas
we define a predicate, called 
\logicref{SomeEqual},
which describes the situation that there is a valid index \inl{i} 
where~\inl{a[i]} equals~\inl{v}.

\input{Listings/SomeNone.acsl.tex}

We first remark that the \SomeEqual, its negation \NoneEqual
and the lemmas \NotSomeEqualNoneEqual and \NoneEqualNotSomeEqual are encapsulated
in the \emph{axiomatic block} 
\logicref{SomeNone}.
This is a \emph{feeble} attempt to establish some modularization for the various predicates,
logic functions and lemmas.
We say \emph{feeble} because axiomatic blocks are, in contrast to \acsl \inl{module}s,
\emph{not} name spaces.
\acsl modules, however, are not yet implemented by \framac.

We also remark that both predicates come in overloaded versions.
The first of theses versions is a definition for array sections while the
second definition is for the case of complete arrays.

Note that we have provided a label, viz.\ \inl{A}, to the
predicate \SomeEqual.
Its purposes to express that the evaluation of the predicate depends on a memory state,
viz.\ the contents of \inl{a[0..n-1]}.
In general, we have to write

\begin{lstlisting}[style=acsl-block]

    \exists integer i; 0 <= i < n && \at(a[i],A) == v;
\end{lstlisting}

in order to express that we refer to the value \inl{a[i]} in
the program state~\inl{A}.
However, \acsl allows to abbreviate \inl{\\at(a[i],A)} by \inl{a[i]} if, as in
\SomeEqual or \NoneEqual, the label~\inl{A} is the only available label.
In particular, we have omitted the label in the overloaded versions for complete arrays.

%\clearpage

\subsection{Formal specification of \findii}

With the predicates \logicref{SomeEqual}
and \logicref{NoneEqual}
we are able to encapsulate all uses of the universal and existential 
quantifiers in both the specification and implementation of \findii.

As a result, the revised contract \specref{findii} is more concise
than that of \specref{find}.
%
In particular, it can be seen immediately that the conditions in the
assumes clauses
of the two behaviors \inl{some} and \inl{none} are mutually
exclusive since
one is the literal negation of the other.
Moreover, the requirement that \find returns the smallest index can
also be expressed
using the \logicref{NoneEqual} predicate, as depicted with the last postcondition of
behavior \inl{some}.

\input{Listings/find2.h.tex}

We also enriched the specification of \find by user-defined names
(sometimes called \emph{labels}, too, the distinction to program state identifiers
being obvious)  to refer to
the \inl{requires} and \inl{ensures} clauses.
We highly recommend this practice in particular for more complex annotations.
For example, \framac can be instructed to verify only clauses with a
given name.

\clearpage

\subsection{Implementation of \findii}

The predicate \NoneEqual is also used in the loop annotation inside
the implementation of \implref{findii}.
Note that, as in the case of the specification, we use labels to name individual annotations.

\input{Listings/find2.c.tex}

%\clearpage



\section{The \findiii algorithm---using a logic function}
\Label{sec:findiii}

In this section we specify linear search yet another way.
This requires more preparing work but results in a more concise function contract.

\subsection{The logic function \Find}

We start with a \emph{recursive} definition of the \acsl function \Find.
Due to the considerable number of associated lemmas of the function
\Find we split its definition into several listings.
Note that \Find comes as two \emph{overloaded} functions.
While the first version is defined for \emph{array sections} the latter is intend
for \emph{complete arrays}.

The listings start with lemmas which express elementary
properties directly related to an incremental increase of the array \inl{a[0..n-1]}. 
The latter lemmas are somewhat more higher-level and will
be useful for the verification of \findiii.
It will be there that we also reuse the predicates 
\logicref{SomeEqual}and
\logicref{NoneEqual}.
%
At the end of this section we will also discuss in what sense the contracts
of \findii and \findiii are equivalent.

\begin{logic}[hbt]
\begin{minipage}{0.99\textwidth}
\lstinputlisting[linerange={1-60}, style=acsl-block, frame=single]{Source/Find.acsl}
\end{minipage}
\caption{\Label{logic:Find-1}The logic function \Find (1)}
\input{Listings/Find.acsl.labels.tex}
\input{Listings/Find.acsl.index.tex}
\end{logic}

\FloatBarrier

\begin{logic}[hbt]
\begin{minipage}{0.99\textwidth}
\lstinputlisting[linerange={61-92}, style=acsl-block, frame=single]{Source/Find.acsl}
\end{minipage}
\caption{\Label{logic:Find-2}The logic function \Find (2)}
\end{logic}

\FloatBarrier

\subsection{Formal specification of \findiii}

Using the logic function \Find we can now give a third specification of linear search.
The contract of \specref{findiii} is considerably shorter than that of \specref{findii}.
Of course, we had to put much more effort into the definition of the \acsl
function \logicref{Find}.

\input{Listings/find3.h.tex}

\clearpage

\subsection{Implementation of \findiii}

The following listing shows the implementation of \implref{findiii}.
In order to achieve a complete verification we had to add the assertion \inl{found}.

\input{Listings/find3.c.tex}

A question that remains is in what sense the contract of \specref{findii} is equivalent to
the one of \specref{findiii}.
We will answer this question in the following section.

\subsection{The equivalence of \findii and \findiii}
\Label{sec:findiv}
\Label{sec:findv}

We consider the contracts of \specref{findii} and \specref{findiii} as \emph{equivalent} 
if each one is sufficient to verify the other.
To this end we introduce yet another two examples \findiv and \findv.

The implementation of \implref{findiv} consists just of a call to \findiii.

\input{Listings/find4.c.tex}

\clearpage

The contract of \specref{findiv}, however, is the same as the one of \specref{findii}.

\input{Listings/find4.h.tex}

Analogously, the implementation of \implref{findv} is simply a call to \findii.

\input{Listings/find5.c.tex}

On the other hand, the contract of \specref{findv} is the same as the one of \specref{findiii}.
%
The verification of the functions \findiv and \findv 
(cf.\ Table~\ref{tbl:result-nonmutating}) then shows the equivalence
of the respective contracts of \specref{findii} and \specref{findiii}.

\input{Listings/find5.h.tex}

\clearpage




\section{The \findifnot algorithm }
\label{sec:findifnot}

Many algorithms in the \cxx standard library can be parameterized not only by the type
of sequence but also using so-called \emph{function objects}.
One example is the \findifnot algorithm that accepts a 
\emph{predicate function object}~$P$.
The algorithm then returns the first position~$i$ in the input sequence where
$P(i)$ does \emph{not} hold.

While function objects could be emulated in \isoc with \emph{pointers to functions},
we will not follow this road here.
The main reason is that function pointers are, so far, only supported momentarily by \framac.
Moreover, there is as of now no support for parameterized \acsl predicates.
For these reasons our implementation of \findifnot only returns the first position
in an array where a given value does \emph{not} occur.
The signature, thus, reads 

\begin{lstlisting}[style=acsl-block]

       size_type find_if_not(const value_type* a, size_type n, value_type v);
\end{lstlisting}

On the one hand, this is not a very exciting addition to our
collections of verified algorithms.
It gives us, however, an opportunity to introduce the predicates
\logicref{AllEqual}
and \logicref{SomeNotEqual}
and more importantly the logic function
\logicref{FindNotEqual}
that will later play
an essential role in the specification of the algorithm \removecopy,
or more precisely, its variant \specref{removecopyiii}.

\input{Listings/AllSomeNot.acsl.tex}

\clearpage

The predicate \AllEqual expresses that each member of the array section

\inl{a[m..n-1]} equals~\inl{v}.
We also introduce the predicate \SomeNotEqual which is the negation of \AllEqual.
Both predicates complement the predicates 
\logicref{SomeEqual} and 
\logicref{NoneEqual}.

There are two additional overloaded versions of \AllEqual.
The first version uses the value \inl{a[m]} as \inl{v}.
The second version is just a shortcut when the first index~\inl{m} equals~0.

\subsection{The logic function \FindNotEqual}

The definition of the overloaded logic function \FindNotEqual is shown in
Listings~\ref{logic:FindNotEqual-1} and~\ref{logic:FindNotEqual-2}.
This function is very similar to 
\logicref{Find} except that it
finds the first element in a sequence that \emph{differs} from a given value.
%
Note that in lemma \FindNotEqualUnchanged we are using the predicate \logicref{Unchanged}
that will be defined in a later chapter.

\begin{logic}[hbt]
\begin{minipage}{\textwidth}
\lstinputlisting[linerange={1-41}, style=acsl-block, frame=single]{Source/FindNotEqual.acsl}
\end{minipage}
\caption{\label{logic:FindNotEqual-1}
   The logic function \FindNotEqual (1)}
\input{Listings/FindNotEqual.acsl.labels.tex}
\input{Listings/FindNotEqual.acsl.index.tex}
\end{logic}

\FloatBarrier

\begin{logic}[hbt]
\begin{minipage}{\textwidth}
\lstinputlisting[linerange={42-100}, style=acsl-block, frame=single]{Source/FindNotEqual.acsl}
\end{minipage}
\caption{\Label{logic:FindNotEqual-2}The logic function \FindNotEqual (2)}
\end{logic}

\FloatBarrier

\clearpage

\subsection{Formal specification of \findifnot}

The contract of \specref{findifnot} is, unsurprisingly,
very similar to that of \specref{findiii}.
The only difference is that we replaced 
\logicref{Find} by 
\logicref{FindNotEqual}.

\input{Listings/find_if_not.h.tex}

\subsection{Implementation of \findifnot}

The implementation of \implref{findifnot}
also has a lot of similarities with of \implref{findiii}.
Here again we have replaced \Find by \FindNotEqual and, of course,
we check in the loop body that the value \inl{a[i]} \emph{differs} from the
given value~\inl{v}.

\input{Listings/find_if_not.c.tex}

\clearpage



\section{The \findfirstof algorithm}
\Label{sec:findfirstof}

The \findfirstof algorithm \cite[\S 28.5.7]{cxx-17-draft}
is closely related to \find (see \S\ref{sec:find} and~\S\ref{sec:findii}).


\begin{lstlisting}[style=acsl-block]

  size_type
  find_first_of(const value_type* a, size_type m,
                const value_type* b, size_type n);
\end{lstlisting}

Like \find, it performs a sequential search.
However, while \find searches for a particular value, 
the function
\findfirstof returns the least index \inl{i} such that \inl{a[i]} 
is equal to one of the values \inl{b[0..n-1]}.


\begin{figure}[hbt]
\centering
\includegraphics[width=0.60\textwidth]{Figures/find_first_of.pdf}
\caption{\Label{fig:findfirstof}A simple example for \findfirstof}
\end{figure}

\FloatBarrier


As an example, we consider in Figure~\ref{fig:findfirstof} two arrays.
The arrow indicates the smallest index where one of the elements of the three-element array
occurs.

\subsection{The predicate \HasValueOf}

Similar to our approach in \S\ref{sec:findii}, we define a predicate
\logicref{HasValueOf}
that formalizes the fact that there are valid indices~\inl{i}
and~\inl{j} of the respective arrays~\inl{a} and~\inl{b} such that \inl{a[i] == b[j]} holds.
We have chosen to reuse the predicate
\logicref{SomeEqual} to define \HasValueOf.

\input{Listings/HasValueOf.acsl.tex}

\clearpage

\subsection{Formal specification of \findfirstof}

The following listing shows the formal specification of \findfirstof.
The function contract uses the predicates \logicref{HasValueOf} and
\logicref{SomeEqual} thereby making it
very similar the specification of \specref{findii}.

\input{Listings/find_first_of.h.tex}

\clearpage

\subsection{Implementation of \findfirstof}

Our implementation of \implref{findfirstof} calls \specref{findii},
thereby emphasizing reuse.
Besides, leading to a more concise implementation, we also have to write fewer loop annotations.

\input{Listings/find_first_of.c.tex}

\clearpage



\section{The \adjacentfind algorithm}
\Label{sec:adjacentfind}

The \adjacentfind
algorithm of the \cxx Standard Library \cite[\S 28.5.8]{cxx-17-draft}

\begin{lstlisting}[style=acsl-block]

  size_type adjacent_find(const value_type* a, size_type n);
\end{lstlisting}

returns the smallest valid index i, such that i+1 is also a valid index 
and such that 
\begin{lstlisting}[style=acsl-block]

    a[i] == a[i+1]

\end{lstlisting}
holds. 
The \adjacentfind algorithm returns \inl{n} if no such index exists.


The arrow in Figure~\ref{fig:adjacentfind} indicates the smallest index
where two adjacent elements are equal.

\begin{figure}[hbt]
\centering
\includegraphics[width=0.60\textwidth]{Figures/adjacent_find.pdf}
\caption{\Label{fig:adjacentfind}A simple example for \adjacentfind}
\end{figure}

\FloatBarrier


\subsection{The predicate \HasEqualNeighbors}

As in the case of other search algorithms, we first define 
a predicate 
\logicref{HasEqualNeighbors}
that captures 
the essence of finding two adjacent indices at which the array holds equal values.

\begin{logic}[hbt]
\begin{minipage}{0.99\textwidth}
\lstinputlisting[style=acsl-block, frame=single]{Source/HasEqualNeighbors.acsl}
\end{minipage}
\caption{\Label{logic:HasEqualNeighbors}The predicate \HasEqualNeighbors}
\end{logic}

\subsection{Formal specification of \adjacentfind}

We use the predicate 
\logicref{HasEqualNeighbors}
to define the formal specification
of \specref{adjacentfind}.

\input{Listings/adjacent_find.h.tex}

\subsection{Implementation of \adjacentfind}

In the implementation of \implref{adjacentfind} we check whether the array
contains at least two elements.
Otherwise, there is no point in looking for adjacent neighbors.
Note the use of the predicate \logicref{HasEqualNeighbors}
 in the loop invariant to
match the similar postcondition of behavior \inl{some}.

\input{Listings/adjacent_find.c.tex}



\section{The \equal and \mismatch algorithms}
\Label{sec:equal}
\Label{sec:mismatch}

The algorithms \equal \cite[\S 28.5.11]{cxx-17-draft}
and \mismatch \cite[\S 28.5.10]{cxx-17-draft}
of the \cxx Standard Library compare two generic sequences.
For our purposes we have modified the generic implementation
to that of an array of type \valuetype.
The signatures read

\begin{lstlisting}[style = acsl-block]

  bool       equal(const value_type* a, size_type n, const value_type* b);

  size_type  mismatch(const value_type* a, size_type n, const value_type* b);
\end{lstlisting}

The function \equal returns~\inl{true} if and only if
\inl{a[i] == b[i]} holds for each \inl{0 <= i < n}.
Otherwise,~\equal returns~\inl{false}.

The \mismatch algorithm is slightly more general than the negation of \equal: 
it returns the smallest index where the two ranges~\inl{a} and~\inl{b} differ.
If no such index exists, that is, if both ranges are equal, then \mismatch
returns the (common) length~\inl{n} of the two ranges.

\subsection{The \Equal predicate}

The fact that two arrays \inl{a[0]..a[n-1]}
and \inl{b[0]..b[n-1]} are equal when compared element by element,
is a property we might need again in other specifications, as it
describes a very basic property.

The motto \emph{don't repeat yourself} is not just good programming
practice.\footnote{
    Compare \url{http://en.wikipedia.org/wiki/Don't_repeat_yourself}
}
It is also true for concise and easy to understand specifications.
We will therefore introduce specification elements that we can apply to
the \equal algorithm as well as to other specifications and
implementations with the described property.

We start with introducing several \emph{overloaded} versions
of the predicate~\logicref{Equal}.

\input{Listings/Equal.acsl.tex}

The letters \inl{K} and \inl{L} in the definition of \Equal
are so-called  \emph{labels}\footnote{
    Labels are used in \isoc to name the target of the \emph{goto}
    jump statement.
}
that refer to program states in which the ranges~\inl{a[..]} and~\inl{b[..]} are evaluated.
\framac defines several standard labels, e.g.\xspace \inl{Old}
and \inl{Post},
a programmer can use to refer to the pre-state or post-state,
respectively, of a function.
For more details on labels we refer to the \acsl
specification \cite[\S 2.6.9]{ACSLSpec}.


\subsection{Formal specification of \equal and \mismatch}

Using predicate \logicref{Equal} we can formulate the specification of \specref{equal}
using the predefined label~\inl{Here}. 
When used in an \inl{ensures} clause, the label~\inl{Here} refers to the post-state of a function.
Note that the equivalence is needed in the ensures clause.
Putting an equality instead is not legal in ACSL,
because \Equal is a predicate, not a function.

\input{Listings/equal.h.tex}


The formal specification of \specref{mismatch}
is more complex than that of \specref{equal} because the return value of mismatch
provides more information than just reporting whether the two arrays are equal.

\input{Listings/mismatch.h.tex}

On the other hand, the specification is conceptually quite similar to that of \specref{findii}.
%
While \findii returns the smallest index~\inl{i} where \inl{a[i] == v} holds,
\mismatch finds the smallest index \inl{a[i] != b[i]}.
Note in particular the use of \Equal in the specification of \mismatch.
As in the specification of \findii
the completeness and disjointness of \mismatch's behaviors is quite obvious,
because the \inl{assumes} clauses of
\inl{all_equal} and \inl{some_not_equal} are negations of each other.

%\clearpage

\subsection{Implementation of \equal and \mismatch}

The implementation of \implref{equal} consists of a simple call of \mismatch.

\input{Listings/equal.c.tex}

The implementation of \implref{mismatch}
has been enriched with some loop annotations to support the deductive verification.

\input{Listings/mismatch.c.tex}

We use again the predicate \logicref{Equal}
in order to express that all indices~\inl{k} that are less than the
current index~\inl{i}
satisfy the condition \inl{a[k] == b[k]}.
This is necessary to prove that \mismatch indeed returns the smallest index
where the two ranges differ.

\clearpage



\section{The \search algorithm}
\Label{sec:search}

The \search algorithm in the \cxx Standard Library \cite[\S
28.5.13]{cxx-17-draft} finds a
subsequence that is identical to a given sequence when 
compared element-by-element.
For our purposes we have modified
the generic implementation
to that of an array of type \valuetype.
The signature now reads:

\begin{lstlisting}[style = acsl-block]

    size_type search(const value_type* a, size_type n,
                     const value_type* b, size_type p);
\end{lstlisting}

The function \search returns the first
index \inl{s} of the array \inl{a} where the condition \inl{a[s+k] == b[k]} holds for each
index~\inl{k} with \inl{0 <= k < p}
(see Figure~\ref{fig:search}).
If no such index exists, then \search returns the length
\inl{n} of the array \inl{a}.

\begin{figure}[hbt]
\centering
\includegraphics[width=0.69\textwidth]{Figures/search.pdf}
\caption{\Label{fig:search} Searching the first occurrence of \inl{b[0..p-1]} in \inl{a[0..n-1]}}
\end{figure}


\subsection{The predicate \HasSubRange}

Our specification of \search starts with introducing the
predicate~\logicref{HasSubRange}.
This predicate formalizes, using the predicate \logicref{Equal},
that the sequence~\inl{a} contains a subsequence which equal the sequence~\inl{b}.
Of course, in order to contain a subsequence of length \inl{p},
\inl{a} must be at least that large; this is expressed by lemma 
\HasSubRangeSizes.

\input{Listings/HasSubRange.acsl.tex}

\subsection{Formal specification of \search}

The following listing shows the specification of \specref{search}.

\input{Listings/search.h.tex}

Conceptually, the specification of \search is very similar to that of
\specref{find}.
We therefore use again two behaviors to capture the essential aspects of \search.

\begin{itemize}
\item
The behavior \inl{has_match} applies if the sequence \inl{a} 
contains a subsequence identical to \inl{b}. 
We express this condition with \inl{assumes} using the predicate 
\logicref{HasSubRange}.

The ensures clause \inl{bound} of behavior \inl{has_match} 
indicates that the returned index value must be in the range~\inl{[0..n-p]}.
The clause \inl{result} expresses that
\search returns an index where a copy of \inl{b} can be found in \inl{a}.
Clause \inl{first}
indicates that the least index with that property is returned,
i.e.\ that \inl{b} can't be found in \inl{a[0..\\result+p-2]}.

\item
The behavior \inl{no_match} covers the case that there is no
subsequence \inl{a} that equals
\inl{b}. In this case, \search must return the length \inl{n} of the
range \inl{a}.
%
If the ranges \inl{a} or \inl{b} are empty
then the return value will be \inl{0}. 
\end{itemize}

The formula in the assumes clause of the behavior
\inl{has_match} is the negation of the assumes clause of the 
behavior \inl{no_match}.
Therefore, we can express that these two behaviors are
\emph{complete} and \emph{disjoint}.

\clearpage

\subsection{Implementation of \search}
\Label{subsec:search implementation}

The implementation of \implref{search} is relatively easy to understand, but
needs an order of magnitude of \inl{n*p} operations.
In contrast, the sophisticated algorithm from
\cite{Knuth.Morris.Pratt.1977}
needs only~\inl{n+p} operations.\footnote{
  \Label{fn:search implementation efficiency}
  The efficiency question has been also discussed by the \cxx standardization committee, 
  see \url{http://www.open-std.org/jtc1/sc22/wg21/docs/papers/2014/n3905.html}
}

The loop invariant \inl{not_found} is needed for the proof of the
postconditions of the behavior \inl{has_match} in the contract of \specref{search}.
It expresses that the subsequence \inl{b} has not been found up to the current iteration step.
%
Neither \inl{p == 0} nor \inl{n == 0} need to be handled separately, not
even for efficiency reasons:
in the former case, \inl{equal(a+i, p, b)} will succeed in the first iteration,
while in the latter, \inl{p > n} will apply.

\input{Listings/search.c.tex}

\clearpage



\section{The \searchn algorithm}
\Label{sec:searchn}

The \searchn algorithm in the \cxx Standard Library \cite[\S
28.5.13]{cxx-17-draft} finds
the first place where a given value starts to occur a given number of
times in a given sequence.
For our purposes we have modified
the generic implementation
to that of an array of type \valuetype.
The signature now reads:

\begin{lstlisting}[style = acsl-block]

   size_type
   search_n(const value_type* a, size_type n, size_type p, value_type v);
\end{lstlisting}

Note the similarity to the signature of \search (\S\ref{sec:search}).
The only difference is that \inl{v} now is a single value rather than
an array.

\begin{figure}[hbt]
\centering
\includegraphics[width=0.59\textwidth]{Figures/search_n.pdf}
\caption{\Label{fig:searchn} Searching the first occurrence a given constant sequence in \inl{a[0..n-1]}}
\end{figure}

\FloatBarrier

The function \searchn returns the first
index \inl{s} of the array \inl{a} where the condition \inl{a[s+k] == v}
holds for each index~\inl{k} with \inl{0 <= k < p} (see Figure~\ref{fig:searchn}).
If no such index exists, then \searchn returns the length
\inl{n} of the array \inl{a}.


\subsection{The predicate \HasConstantSubRange}

Our specification of \searchn starts with introducing the
predicate 
\logicref{HasConstantSubRange}.

\input{Listings/HasConstantSubRange.acsl.tex}

This predicate formalizes that the sequence~\inl{a} of length \inl{n}
contains a subsequence of \inl{p} times the value~\inl{v}.
It thereby reuses the predicate
\logicref{AllEqual}.

Similar to predicate 
\logicref{HasSubRange},
in order to contain \inl{p} repetitions, the size of the
array \inl{a[0..n-1]} must be at least that large;
this is what lemma 
\logicref{HasConstantSubRangeSizes} says.

\subsection{Formal specification of \searchn}

Like for \specref{search}, our specification of \specref{searchn}
is very similar to that of \specref{findii}.

\input{Listings/search_n.h.tex}

We again use two behaviors to capture the essential aspects of \searchn.

\begin{itemize}
\item
The behavior \inl{has_match} applies if the sequence \inl{a} 
contains an \inl{n}-fold repetition of \inl{b}. 
We express this condition with  \inl{assumes} by using the predicate
\logicref{HasConstantSubRange}.
The \inl{result} ensures clause of behavior \inl{has_match} indicates
that the return value must be in the range~\inl{[0..n-p]}.
The \inl{match} ensures clause
expresses that the return value of \searchn actually points to an index 
where \inl{b}
can be found \inl{p} or more times in \inl{a}. 
The \inl{first} ensures clause expresses that the minimal index with
this property is returned.

\item
The behavior \inl{no_match} covers the case that there is no matching
subsequence in sequence \inl{a}.
In this case, \searchn must return the length \inl{n} of the
range \inl{a}.
\end{itemize}

\input{Listings/search_n.c.tex}

\subsection{Implementation of \searchn}

Although the specification of \specref{searchn} strongly resembles that of
\specref{search}, their implementations differ significantly.
The implementation of \implref{searchn} has a time complexity of
$\mathcal{O}(n)$, whereas the implementation of
\implref{search} employs an easy, but
a non-optimal algorithm needing $\mathcal{O}(n \cdot p)$ time.

Our implementation maintains in the variable \inl{start} the beginning
of the most recent consecutive range of values~\inl{v}.
The loop invariant \inl{not_found} states that we didn't find an
\inl{p}-fold repetition of \inl{b} up to now; if we find one, we
terminate the loop, returning \inl{start}.
%
We handle the boundary cases \inl{n < p} and \inl{p == 0} in explicit else branches.
We found this easier when trying to ensure a verification by automatic provers.

\clearpage



\section{The \findend algorithm}
\Label{sec:findend}

The \findend algorithm in the \cxx Standard Library \cite[\S
28.5.6]{cxx-17-draft} searches for the last
subsequence that is identical to a given sequence when 
compared element-by-element.
For our purposes we have modified
the generic implementation
to that of an array of type \valuetype.
The signature now reads:

\begin{lstlisting}[style = acsl-block]

  size_type
  find_end(const value_type* a, size_type n, const value_type* b, size_type p);
\end{lstlisting}

The function \findend returns the greatest
index \inl{s} of the array \inl{a} where the condition \inl{a[s+k] == b[k]} holds for each
index~\inl{k} with \inl{0 <= k < p}
(see Figure~\ref{fig:findend}).
If no such index exists, then \findend returns the length
\inl{n} of the array \inl{a}. One has to remark the special case \inl{p == 0}.
In this case the last position of the empty string is found (the length \inl{n})
and returned.

\begin{figure}[hbt]
\centering
\includegraphics[width=0.69\textwidth]{Figures/find_end.pdf}
\caption{\Label{fig:findend} Finding the last occurrence \inl{b[0..p-1]} in \inl{a[0..n-1]}}
\end{figure}

\clearpage

\subsection{Formal specification of \findend}

The following listing shows the specification of \specref{findend}.
Conceptually, the specification of the function \findend is very similar to that of
\specref{findii}.
We therefore use again behaviors to capture the essential aspects of \findend.
It is quite clear that these behaviors are \emph{complete} and \emph{disjoint}.

The behavior \inl{has_match} applies if the sequence \inl{a} 
contains a subsequence identical to \inl{b}. 
We express this condition with \inl{assumes} using the predicate 
\logicref{HasSubRange}.
The \inl{ensures} clause \inl{bound} indicates that the return
value must be in the range~\inl{0..n-p}.
The clause \inl{result} of behavior \inl{has_match} expresses that \findend
returns an index where \inl{b} can be found in \inl{a}.
%
Finally, the clause \inl{last}
indicates that the sequence \inl{a} does not contain
\inl{b} beginning at a position larger than \inl{\\result}.

The behavior \inl{no_match} covers the case that there is no
subsequence of \inl{a} that equals
\inl{b}. In this case, \findend must return the length \inl{n} of the
range \inl{a}.

\input{Listings/find_end.h.tex}

\clearpage

\subsection{Implementation of \findend}

Our implementation of \implref{findend} is similar to the one of \implref{search}.

\input{Listings/find_end.c.tex}

We maintain in the variable \inl{r} the prospective value to be
returned, according to the current knowledge.
Initially, it is set to \inl{n}, meaning ``no occurrence of \inl{b}
found yet''.
Whenever an occurrence is found, \inl{r} is updated to its starting
position.

The invariant \inl{bound} states that \inl{r} either still has the value
\inl{n} or has a value up to \inl{n-p}.
For the former case, invariant \inl{not_found}
indicates that no occurrence of \inl{b} has been found.
For the latter case, the loop invariant \inl{found} indicates that an occurrence
\inl{b[0..p-1]} at \inl{r} has indeed been found.
The invariant \inl{last}, on the other hand states that
none was found \emph{after} the index~\inl{r}.

\clearpage



\section{The \counti algorithm}
\Label{sec:counti}

The \counti algorithm in the \cxx Standard Library \cite[\S
28.5.9]{cxx-17-draft} counts
the frequency of occurrences for a particular element in
a sequence.
For our purposes we have modified
the generic implementation
to that of arrays of type \valuetype.
The signature now reads:

\begin{lstlisting}[style = acsl-block]

  size_type
  count(const value_type* a, size_type n, value_type v);
\end{lstlisting}

Informally, the function returns the number of occurrences of
\inl{v} in the array \inl{a}.

\subsection{The logic function \Count}

When trying to specify \counti we are faced with the situation that
\acsl does not provide a definition of counting a value in an array.\footnote{
  This statement is not quite true because the \acsl documentation 
  lists \inl{numof} as one of several \emph{higher order logic constructions} \cite[\S 2.6.7]{ACSLSpec}.
  However, these \emph{extended quantifiers} are mentioned only as experimental features.
}
We therefore start with an axiomatic definition of \emph{logic function} \Count
that captures the basic intuitive features of counting on an array section.
The expression \inl{Count(a,m,n,v)} returns the number of
occurrences of \inl{v} in \inl{a[m],...,a[n-1]}.

The specification of \counti will then be fairly short because it employs
our \emph{logic function}
\Count whose (considerably) longer definition is given in the 
Listings~\ref{logic:Count-1} and~\ref{logic:Count-2}.\footnote{
 This definition of \Count is a generalization of
 the \emph{logic function} \inl{nb_occ} of the \acsl specification \cite{ACSLSpec}.
}


\begin{itemize}
\item
The \acsl keyword \inl{axiomatic} 
is used to structure the specification and gather the logic function \Count and related lemmas.
Note that the interval bounds \inl{m} and \inl{n} and the return value for \Count are of type \inl{integer}.

\item
The logic functions \Count is recursively defined.
It consist of two checks: whether the range is empty and for the value of
the "current" element in the array. The recursion goes down on the range length.
We also provide an overloaded version of \Count that accepts only
the length of an array, thus relieving the use the supply the argument $m=0$ for the
case of a complete array.

\item
Lemma \logicref{CountEmpty} covers the cases of empty ranges.

\item
Lemmas \logicref{CountHit} and 
\logicref{CountMiss} reduce
counting of a range of length~$n-m$ to a range of length~$n-m-1$.

\item
Lemmas \logicref{CountOne} and \logicref{CountSingle} built on on top of \CountHit
and \CountMiss. 
Using them simplifies several \coq proofs.
They also slightly change the induction scheme from $n-1 \rightarrow n$
to $n \rightarrow n+1$.
\end{itemize}


\begin{logic}[hbt]
\begin{minipage}{\textwidth}
\lstinputlisting[linerange={1-45}, style=acsl-block, frame=single]{Source/Count.acsl}
\end{minipage}
\caption{\label{logic:Count-1}The logic function \Count (1)}
\input{Listings/Count.acsl.labels.tex}
\input{Listings/Count.acsl.index.tex}
\end{logic}

\FloatBarrier

\begin{itemize}

\item
The logic function \Count depends only on the set \inl{a[m..n-1]} of memory locations.
Lemma \logicref{CountUnchanged} makes this claim explicit by ensuring that
\Count produces the same result
if the values \inl{a[0..n-1]} do not change between two program states indicated
by the labels~\inl{K} and~\inl{L}.
We use here predicate \logicref{Unchanged} to express the premise.

\item 
Lemma \logicref{CountEqual} is a generalization of lemma \CountUnchanged for
the case of comparing \Count on two arrays.

\item
Lemmas \logicref{CountUnion} and \logicref{CountCut} 
allow to deal with partitions of arrays.
\end{itemize}

\begin{logic}[hbt]
\begin{minipage}{\textwidth}
\lstinputlisting[linerange={46-90}, style=acsl-block, frame=single]{Source/Count.acsl}
\end{minipage}
\caption{\label{logic:Count-2}The logic function \Count (2)}
\end{logic}

\FloatBarrier

\begin{itemize}
\item 
Lemmas \logicref{CountSingleBounds} and \logicref{CountBounds}
express lower and upper bounds of \Count.
Lemma \logicref{CountIncreasing} states that \Count is a monotonically increasing.

\item
Finally, lemmas \logicref{CountSingleShift} and \logicref{CountShift}
state that \Count is invariant under array shifts.
\end{itemize}

We mention here also lemma \logicref{CountSomeEqual}
which brings together properties of \logicref{Count} and \logicref{Find}.

\input{Listings/CountFind.acsl.tex}

\clearpage 

\subsection{Formal specification of \counti}

In the contract of \specref{counti} we use the logic function
\logicref{Count}
Note that our specification also states that the result of \counti is non-negative
and less than or equal the size of the array.

\input{Listings/count.h.tex}

\subsection{Implementation of \counti}

The following listing shows a possible implementation of \implref{counti}.
Note that we refer to the logic function \Count in one of the loop invariants.

\input{Listings/count.c.tex}

\clearpage



\section{The \countii algorithm}
\Label{sec:countii}

In this section, we specify the \counti algorithm in a different way, namely
using the \emph{inductively} defined predicate
\logicref{CountInd} from the
following listing.
%

\input{Listings/CountInd.acsl.tex}

The definition consists of three cases.
\begin{itemize}
\item
The \inl{Nil} case states for arrays of negative pf zero length,
the predicate only holds is \inl{sum} is zero.

\item
The \inl{Hit} and \inl{Miss} define \CountInd for arrays \inl{a[0..n-1]}
of size \inl{n} referring to the array \inl{a[0..n-2]} and the value \inl{a[n-1]}.
\end{itemize}

We remark that the cases are very similar to the
lemmas \logicref{CountEmpty},
\logicref{CountHit}
and \logicref{CountMiss},
except we have use the additional argument \inl{sum} to refer to the number
of counted elements since \CountInd is a predicate.

We  have intentionally used the scheme $n-1 \Rightarrow n$ instead of $n \Rightarrow n+1$.
In this particular case, it allows theorem provers to match loop indices
with premises without additional hints to prove loop invariants.

\subsection{Additional lemmas for the inductive predicate}

The lemmas of 
\logicref{CountIndImplicit}
complement the lemmas of \logicref{Count}.
They demonstrate how existing lemmas can be rewritten for an inductive predicate.
%
These lemmas are not required to prove the \counti function,
but we provide them to complete the illustrative example of how
inductive predicates could be utilized in the specifications.

The inductive definition is the ``complete'' definition
which means that the predicate does not hold for cases outside of its domain of definition.
We state this property explicitly through lemma
\logicref{CountIndInverse}
in the following listing.
Frama-C does not automatically generate this kind of property.
The reason for not adding such a corresponding axiom apparently is that it ``could
confuse first-order theorem provers''.\footnote{\url{https://stackoverflow.com/a/32457870}}

\input{Listings/CountIndImplicit.acsl.tex}

There is also the lemma \logicref{CountIndNonNegative}
which states that the lower bound for the number of the counted elements is zero.
%
The relation between the inductive definition \CountInd and the explicit 
definition of \logicref{Count} is expressed
by lemma \logicref{CountIndCount}.

\input{Listings/CountIndLemmas.acsl.tex}

\clearpage

\subsection{Specification of \countii}

The following listing contains the contracts of \specref{countii}.
It shows the use of the inductive predicate 
\logicref{CountInd}.

\input{Listings/count2.h.tex}

\subsection{Implementation of \countii}

The only difference between the implementation of \implref{countii} 
and the implementation of \implref{counti}
is that we have to supply the value \inl{counted} as an argument
of the predicate \logicref{CountInd}.

\input{Listings/count2.c.tex}

\clearpage



