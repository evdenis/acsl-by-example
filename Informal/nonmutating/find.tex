
\section{The \find algorithm}
\Label{sec:find}

The \find algorithm in the \cxx Standard Library \cite[\S 28.5.5]{cxx-17-draft}
implements \emph{sequential search} for general sequences.
We have modified the generic implementation,
which relies heavily on \cxx templates, to that of a range of
type \valuetype.
The signature now reads:

\begin{lstlisting}[style=acsl-block]

       size_type find(const value_type* a, size_type n, value_type v);
\end{lstlisting}

The function \find returns the least \emph{valid} index \inl{i} of \inl{a}
where the condition \inl{a[i] == v} holds. 
If no such index exists then \find returns the length \inl{n} of the array.


As an example, we consider in Figure~\ref{fig:find} an array.
The arrows indicate which indices will be returned by \find for a given value.
Note that the index~9 points \emph{one past end} of the array.
Values that are not contained in the array are colored in gray.

\begin{figure}[hbt]
\centering
\includegraphics[width=0.60\textwidth]{Figures/find.pdf}
\caption{\Label{fig:find}Some simple examples for \find}
\end{figure}

\FloatBarrier


\subsection{Formal specification of \find}

The following listing shows our first attempt specify \specref{find}.

\input{Listings/find.h.tex}

The \inl{requires}-clause indicates that \inl{n} is non-negative and
that the pointer \inl{a} points to $n$~contiguously allocated objects of type
\valuetype (see~\S\ref{sec:frequentPattern}).
%
The \inl{assigns}-clause indicates that \find (as a non-mutating algorithm),
does not modify any memory location outside its scope (see~Page~\pageref{assigns-clause}).

Generally, we only know that \find returns a non-negative index that is less or
equal the length of the array.
However, once we assume more specific situations,
we can also make more precise statements about the returned valued.
This is the reason why we have subdivided the specification of
\find into two behaviors (named \inl{some} and \inl{none}).

\begin{itemize}
\item
The behavior \inl{some} applies if the sought-after value is contained in the array.
We express this condition by using the \inl{assumes}-clause.
The next line expresses that if the assumptions of the behavior are satisfied then
\find will return a valid index.
The algorithm also ensures that  the returned (valid) index \inl{i},
\inl{a[i] == v} holds.
Therefore we define this property in the second postcondition of behavior \inl{some}.
Finally, it is important to express that \find returns the smallest index~\inl{i}
for which \inl{a[i] == v} holds (see last postcondition of behavior \inl{some}).

\item
The behavior \inl{none} covers the case that the sought-after value 
is \emph{not} contained in the array (see \inl{assumes}-clause of behavior \inl{none} in
in the contract of\specref{find}.
In this case, \find must return the length \inl{n} of the range \inl{a}.
\end{itemize}

Note that the formula in the \inl{assumes}-clause of the behavior \inl{some}
is the negation of the \inl{assumes}-clause of the behavior \inl{none}.
Therefore, we can express  that these two behaviors are
\emph{complete} and \emph{disjoint}.

\subsection{Implementation of \find}

The noteworthy elements of our implementation of \implref{find} are the 
\emph{loop annotations}.
%
The first loop \emph{invariant} is needed to prove that accesses
to~\inl{a} only occur with valid indices. The second  loop \emph{invariant} is needed
for the proof of the postconditions of the
behavior~\inl{some} in the contract of \specref{find}.
It expresses that for each iteration the sought-after value is not
yet found up to that iteration step.
Finally, the  loop \emph{variant} \inl{n-i} is needed to generate correct verification
conditions for the termination of the loop.

\input{Listings/find.c.tex}

\clearpage

