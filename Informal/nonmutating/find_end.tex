
\section{The \findend algorithm}
\Label{sec:findend}

The \findend algorithm in the \cxx Standard Library \cite[\S
28.5.6]{cxx-17-draft} searches for the last
subsequence that is identical to a given sequence when 
compared element-by-element.
For our purposes we have modified
the generic implementation
to that of an array of type \valuetype.
The signature now reads:

\begin{lstlisting}[style = acsl-block]

  size_type
  find_end(const value_type* a, size_type n, const value_type* b, size_type p);
\end{lstlisting}

The function \findend returns the greatest
index \inl{s} of the array \inl{a} where the condition \inl{a[s+k] == b[k]} holds for each
index~\inl{k} with \inl{0 <= k < p}
(see Figure~\ref{fig:findend}).
If no such index exists, then \findend returns the length
\inl{n} of the array \inl{a}. One has to remark the special case \inl{p == 0}.
In this case the last position of the empty string is found (the length \inl{n})
and returned.

\begin{figure}[hbt]
\centering
\includegraphics[width=0.69\textwidth]{Figures/find_end.pdf}
\caption{\Label{fig:findend} Finding the last occurrence \inl{b[0..p-1]} in \inl{a[0..n-1]}}
\end{figure}

\clearpage

\subsection{Formal specification of \findend}

The following listing shows the specification of \specref{findend}.
Conceptually, the specification of the function \findend is very similar to that of
\specref{findii}.
We therefore use again behaviors to capture the essential aspects of \findend.
It is quite clear that these behaviors are \emph{complete} and \emph{disjoint}.

The behavior \inl{has_match} applies if the sequence \inl{a} 
contains a subsequence identical to \inl{b}. 
We express this condition with \inl{assumes} using the predicate 
\logicref{HasSubRange}.
The \inl{ensures} clause \inl{bound} indicates that the return
value must be in the range~\inl{0..n-p}.
The clause \inl{result} of behavior \inl{has_match} expresses that \findend
returns an index where \inl{b} can be found in \inl{a}.
%
Finally, the clause \inl{last}
indicates that the sequence \inl{a} does not contain
\inl{b} beginning at a position larger than \inl{\\result}.

The behavior \inl{no_match} covers the case that there is no
subsequence of \inl{a} that equals
\inl{b}. In this case, \findend must return the length \inl{n} of the
range \inl{a}.

\input{Listings/find_end.h.tex}

\clearpage

\subsection{Implementation of \findend}

Our implementation of \implref{findend} is similar to the one of \implref{search}.

\input{Listings/find_end.c.tex}

We maintain in the variable \inl{r} the prospective value to be
returned, according to the current knowledge.
Initially, it is set to \inl{n}, meaning ``no occurrence of \inl{b}
found yet''.
Whenever an occurrence is found, \inl{r} is updated to its starting
position.

The invariant \inl{bound} states that \inl{r} either still has the value
\inl{n} or has a value up to \inl{n-p}.
For the former case, invariant \inl{not_found}
indicates that no occurrence of \inl{b} has been found.
For the latter case, the loop invariant \inl{found} indicates that an occurrence
\inl{b[0..p-1]} at \inl{r} has indeed been found.
The invariant \inl{last}, on the other hand states that
none was found \emph{after} the index~\inl{r}.

\clearpage

