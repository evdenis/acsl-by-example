
\chapter{Sorting Algorithms}
\Label{cha:sorting}



Many issues in computer science can be exemplified in the field of
sorting algorithms; see e.g.\ \cite{Knuth.1973} for a famous textbook.
Therefore we arrange some of the most common classic
sorting algorithms.
In this chapter, we present algorithms of the \cxx Standard
Library \cite[\S 28.7.1]{cxx-17-draft} that are related to the task
of sorting a linear array.

Following \cite{Sellibitze.2014}, we have also used (\isoc rephrasings of) 
functions from the \cxx Standard Library as far as possible to implement the different
algorithmic approaches.

\begin{itemize}
\item \issorted in \S\ref{sec:issorted} is an algorithm that checks if a given
array is already in increasing order.

\item \partialsort in \S\ref{sec:partialsort} rearranges a given
array into two parts. All elements in the first part are less or equal than
those of the second part. Moreover, while the first part is sorted,
the order of elements in the second part is unspecified.

\item \bubblesort in \S\ref{sec:bubblesort} describes a simple, well-known and 
      sorting algorithm.\footnote{
   See \url{https://en.wikipedia.org/wiki/Bubble_sort}
}

\item
\selectionsort in \S\ref{sec:selectionsort} presents the classic
\emph{selection sort} algorithm.\footnote{
   See \url{https://en.wikipedia.org/wiki/Selection_sort}
}

\item \insertionsort in \S\ref{sec:insertionsort} the also well-known
\emph{insertion sort} algorithm.\footnote{
   See \url{https://en.wikipedia.org/wiki/Insertion_sort}
}

\item \heapsort in \S\ref{sec:heapsort} describes the quite efficient \emph{heap sort},
which relies on the algorithms presented in Chapter~\ref{cha:heap}.\footnote{
   See \url{https://en.wikipedia.org/wiki/Heapsort}
}

\item \merge in \S\ref{sec:merge} the \emph{merge} algorithm from \emph{merge sort}.\footnote{
  See \url{https://en.wikipedia.org/wiki/Merge_sort}
}
\end{itemize}

While \heapsort achieves a run-time complexity upper bound of 
${\mathcal O}(n \cdot \log(n))$ due to the efficiency of the heap
data structure,
both \selectionsort and \insertionsort
need ${\mathcal O}(n^2)$ in the average case,
and also in the worst case.

Note that the \inl{sort} algorithm from the \cxx Standard Library
is not handled here because it typically relies on \emph{introspection sort}
which is sophisticated mix of various classic algorithms.\footnote{
  See \url{https://en.wikipedia.org/wiki/Introsort}
}
In future releases we plan to handle the more algorithms related sorting.

\clearpage

The sorting algorithms in this chapter essentially share the following contract;
it is their implementations that differ fundamentally.

\begin{lstlisting}[style = acsl-block]
        /*@
           requires valid: \valid(a + (0..n-1));

           assigns  a[0..n-1];

           ensures increasing:  Increasing(a, n);
           ensures reorder:     MultisetReorder{Old, Here}(a, n);
        */
        void xxx_sort(value_type* a, size_type n);
\end{lstlisting}

As mentioned in the introduction of Chapter~\ref{cha:binary-search},
we use the predicate \logicref{Increasing} in the contracts of our
sorting algorithms but often resort to the simpler predicate
\logicref{WeaklyIncreasing} in the loop invariants and assertions.
In order to conclude that the desired postcondition \inl{Increasing(a, n)} holds,
we rely on lemma \logicref{WeaklyIncreasingIncreasing}.


%\clearpage


\section{The \issorted algorithm}
\Label{sec:issorted}

Our version of the \issorted algorithm compared to the \cxx Standard
Library \cite[\S 28.7.1.5]{cxx-17-draft} has the signature
\begin{lstlisting}[style = acsl-block]

  bool is_sorted(const value_type* a, size_type n);
\end{lstlisting}

It returns \inl{true} if the given array is in increasing order, and
\inl{false} otherwise.

\FloatBarrier

\subsection{Formal specification of \issorted}

The following listing shows the acsl specification of \issorted.
%
In the contract, we use the predicate \logicref{Increasing},
which states that any array element is always less or equal to any other element right of it.
%
We'll use an easier-to-handle predicate in the implementation of \implref{issorted}.

\input{Listings/is_sorted.h.tex}

\clearpage

\subsection{Implementation of \issorted}

The implementation of \issorted is shown in the next Listing.
%
As usual, \issorted doesn't compare every array element to all that are right
to it, but only to the immediately adjacent one, which is of course
more efficient.
For this, we use the predicate \logicref{WeaklyIncreasing}
in the loop invariant of the implementation.

\input{Listings/is_sorted.c.tex}

Since our implementation uses \WeaklyIncreasing in its loop invariant, and
follows the same principle in its code, its verification is
straight-forward---except for the final reasoning that
\inl{WeaklyIncreasing(a,n)} implies \inl{Increasing(a,n)}.

We have the lemma \logicref{WeaklyIncreasingIncreasing} for that step,
which needs to be proven manually with \coq.
%
The converse lemma \logicref{IncreasingWeaklyIncreasing}
is proven automatically, but isn't actually needed to verify our
\issorted implementation.
Alternatively, we could have dragged the predicate \Increasing along the
loop, which happens to cause no particular problems in this case.

\clearpage



\section{The \partialsort algorithm}
\Label{sec:partialsort}

Our version of the \partialsort algorithm compared to the \cxx Standard
Library \cite[\S 28.7.1.3]{cxx-17-draft} has the signature

\begin{lstlisting}[style = acsl-block]

  void partial_sort(value_type* a, size_type m, size_type n);
\end{lstlisting}

The algorithm \emph{reorders} the given array \inl{a} in such a way
that it represents a \emph{partition}:
each member of the
left part \inl{a[0..m-1]} is less or equal to each member of the right
part \inl{a[m..n-1]}.
%
Moreover, the algorithm \emph{sorts} the left part in increasing order.
The order of elements in the right part, however, is \emph{unspecified}.
%
Figure~\ref{fig:partialsort} uses a bar chart to
depict a typical result of a call \inl{partial_sort(a, m, n)}.
%
In the post-state, 
the left and the right part is colored in green and orange,
respectively.


\begin{figure}[hbt]
\begin{center}
\includegraphics[width=0.75\textwidth]{Figures/partial_sort.pdf}
\caption{\Label{fig:partialsort} Effects of \partialsort}
\end{center}
\end{figure}

\FloatBarrier

\subsection{The predicate \Partition}

We start by introducing the new predicate \logicref{Partition}
which formalizes the partitioning property.

\input{Listings/Partition.acsl.tex}

\clearpage

The lemmas in the following listing are used in proofs of properties and annotations
related to the loop invariants \inl{upper}, \inl{lower}, and \inl{partition}
of \partialsort.

\input{Listings/PartitionLemmas.acsl.tex}


\begin{itemize}
\item
Lemma \MultisetReorderSomeEqual states that a value
\inl{a[i]} taken from a range \inl{a[0..n-1]} after some reordering
must have been in that range already before reordering.
It is used to prove the subsequent lemmas.

\item
Lemma \MultisetReorderLowerBound
informally says that a lower bound \inl{v} of a
range \inl{a[0..n-1]} keeps its property even after the range is
reordered.

\item
Dually, lemma \MultisetReorderUpperBound says that reordering a range
doesn't affect any of its upper bounds.

\item
Lemma \MultisetReorderPartitionLowerBound 
describes a more particular
situation: if each element in \inl{a[0..m-1]}
is known to be a less or equal than element \inl{a[m..n-1]}
and the former range is reordered while the latter is kept untouched,
then \inl{a[0]} will still be a lower bound of \inl{a[m..n-1]}.
We employ this lemma to infer that, after \specref{pushheap} was called, the new
heap maximum \inl{a[0]}, is a lower bound of \inl{a[m..i]},

\end{itemize}

The proof of \logicref{MultisetReorderSomeEqual} relies on the lemma \logicref{CountSomeEqual}.
We also rely on the lemma \logicref{MultisetSwapMiddle}
in order to verify that the loop invariant \inl{reorder} is preserved.

\subsection{Formal specification of \partialsort}

The formal specification of the \partialsort function is shown in the following listing.
It uses the just introduced predicate \Partition and reuses the
previously defined predicates \logicref{Increasing} and \logicref{MultisetReorder}.

\input{Listings/partial_sort.h.tex}

\subsection{Implementation of \partialsort}

Our implementation of \partialsort is shown the next listing.
%
It initially calls \specref{makeheap} to rearrange the left part \inl{a[0..m-1]} into a heap.
%
After that, it scans the right part, from left to right, for elements
that are too small;
each such element is exchanged for the left part's maximum, by applying
\specref{popheap}  and \specref{pushheap}  appropriately.
%
When the scan is done, the smallest elements are collected in the left
part.
%
We finally convert it from a heap into an increasingly ordered range,
by \sortheap (\ref{sec:sortheap}).

\begin{figure}[hbt]
\begin{center}
\includegraphics[width=0.50\textwidth]{Figures/partial_sort-loop.pdf}
\caption{\Label{fig:partialsort-loop}An iteration of \partialsort}
\end{center}
\end{figure}

\clearpage

In the scan loop, we maintain as invariants
\begin{itemize}
\item that the left part is a heap (invariant \inl{heap});
\item that its maximal element, \inl{a[0]}, is a ``separating element''
  between the left part \inl{a[0..m-1]} and the right part \inl{a[m..i-1]},
  i.e., an upper bound of the left (invariant \inl{upper})
  and a lower bound of the right part (invariant \inl{lower}), respectively;
\item that \inl{a[i..m-1]} is yet unchanged (invariant \inl{unchanged}); and
\item that only permutation operations have been applied to
  \inl{a[0..i-1]} (invariant \inl{reorder}).
\end{itemize}

In order to preserve the loop invariants after \inl{i} is incremented,
nothing has to be done if \inl{a[0]} happens to be
also a lower bound for \inl{a[i]}.
Otherwise, let us follow the algorithm through the \inl{then} part code,
depicting the intermediate states in 
Figure~\ref{fig:partialsort-loop}.
The elements considered so far are shown colored similar to
Figure~\ref{fig:partialsort}; in particular the heap part is shown in green.

%\clearpage

%
The overlaid transparent red shape indicates the ranges to which
\Partition applies, in each state.
%
The figure assumes the initial contents of \inl{a[0]} and
\inl{a[i]} to be $9$ and $5$,
for sake of generality, let us
call them $p$ and $q$, respectively.

After \popheap and \swap,
we have $p$ at \inl{a[i]}, and $q$ at \inl{a[m-1]}.
%
At that point we know
%
\begin{enumerate}
\item $q < p \leq \mbox{\inl{a[k]}}$ for each $m \leq k < i$,
  since $p$ was a lower bound for \inl{a[m..i-1]};
\item $q < p = \mbox{\inl{a[i]}}$;
\item $\mbox{\inl{a[j]}} \leq p \leq \mbox{\inl{a[k]}}$ 
  for each $0 \leq j < m-1$ and each $m \leq k < i$,
  since this held on loop entry, and we didn't more than
  reordering inside the parts; and
\item $\mbox{\inl{a[j]}} \leq p = \mbox{\inl{a[i]}}$ 
  since $p$ was the heap maximum on loop entry.
\end{enumerate}

\begin{listing}[t]
\begin{minipage}{\textwidth}
\lstinputlisting[linerange={1-38}, style=acsl-block, frame=single]{Source/partial_sort.c}
\end{minipage}
\caption{\Label{lst:partialsort-impl1}Implementation of \partialsort (1)}
\end{listing}

\index[examples]{partial\_sort@\texttt{partial\_sort}}


\FloatBarrier

Altogether, we have  $\mbox{\inl{a[j]}} \leq p \leq \mbox{\inl{a[k]}}$
for each $0 \leq j < m$ and each $m \leq k < i+1$.
%
That is, \inl{Partition(a,m,i+1)} holds, although we cannot name a
separating element of \inl{a} here.


After calling \pushheap, which just performs some more 
reorderings of the left part, this property is preserved. 
We can't and we needn't tell which position $q$ is moved to;
the former is indicated in Figure~\ref{fig:partialsort}
by the vague grey triangle.
%
Moreover, we now know again that \inl{a[0]} has become an upper bound
of the left part,
and hence a separating element between
\inl{a[0..m-1]} and \inl{a[m..i]};
that is, the loop invariants \inl{upper} and \inl{lower} have been
re-established.
%
These two invariants together are eventually used to prove
the property \inl{partition} of the contract.

Compared to its size, the algorithm makes a
lot of procedure calls; in this respect it is closer to real-life
software than most other algorithms of this tutorial.
%
Therefore, we use it to illustrate a methodical point:
%
For almost every procedure call, we give the callee's contract,
tailored to its actual parameters, as a statement contract of the call.
%
For example, everything we know from the \popheap contract,
instantiated to the particular situation, is documented in the
first statement contract.
%
In contrast, we use \inl{assert} clauses to indicate intermediate
reasoning to obtain subsequently needed properties.

\begin{listing}[t]
\begin{minipage}{\textwidth}
\lstinputlisting[linerange={39-99}, style=acsl-block, frame=single]{Source/partial_sort.c}
\end{minipage}
\caption{\Label{lst:partialsort-impl2}The Implementation of
\partialsort (2)}
\end{listing}

\index[examples]{partial\_sort@\texttt{partial\_sort}}

\FloatBarrier


Our implementation has a worst-case time complexity of
${\cal O}((n+m) \cdot \log m)$.
%
On the other hand, an implementation that ignores \inl{m} and just sorts \inl{a[0..n-1]}
also satisfies the contract of \specref{partialsort},
and may have ${\cal O}(n \cdot \log n)$ complexity.
%
Some arithmetic shows that \partialsort performs better than
plain sort if, and only if,
$\log m < \dfrac{n}{m} \cdot \log\left(\dfrac{n}{m}\right)$,
that is, if $n$ is sufficiently larger than $m$.



\clearpage



\section{The \bubblesort algorithm}
\label{sec:bubblesort}

The \bubblesort algorithm traverses the given array \inl{a[0..n-1]} from left
to right, maintaining a right-adjusted, constantly growing range
\inl{a[n-i..n-1]} that is already in increasing order.
We achieve this range by iterating through the array and swapping two adjacent
elements, if their respective value are in the wrong order.


\subsection{Formal specification of \bubblesort}

The following listing shows our (generic sorting) contract for \bubblesort.

\input{Listings/bubble_sort.h.tex}


\subsection{Implementation of \bubblesort}

Our implementation of \bubblesort is shown in the next listing.
As it is typical for \bubblesort, the implementation uses two nested loops.

We first discuss the verification of the fact that
\bubblesort produces an increasing array.
For this we introduce for the \emph{outer loop} the invariant \inl{increasing}.
This loop annotation states that the subrange \inl{a[n-i+1..n-1]} is in increasing order.
An important ingredient on the verification of the \inl{increasing} property
is the claim that the first element~\inl{a[n-i+1]} of the already sorted
subrange is an upper bound of \emph{all} elements left of it.
This claim is encoded in the loop invariant \inl{upper} of the outer bound.
%
In order to support this claim up we exploit the fact
that the index~\inl{j} of the \inl{inner loop} points to the maximum element
of the subrange \inl{a[0..j]}.
We formalize this last property in the loop invariant~\inl{max}.

Note that the loop invariants~\inl{increasing} and~\inl{upper} occur also
in the inner loop. This shall ``assure'' the outer loop that
the inner loop really preserves these properties.

\input{Listings/bubble_sort.c.tex}

We now discuss briefly the verification of the postcondition \inl{reorder}.
In each iteration of the outer loop various elements of the 
not yet sorted subrange \inl{a[0..n-1]} are swapped with their respective 
neighbour.
More specifically, we know for the iteration~\inl{j} of the \emph{inner loop}
that while subrange \inl{a[0..j]} has been rearranged,
the subrange \inl{a[j+1..n-1]} has not been modified yet.
Together this ensures that the loop invariant~\inl{reorder}
holds for the \emph{outer loop}.

\clearpage



\section{The \selectionsort algorithm}
\Label{sec:selectionsort}

Our version of the \selectionsort algorithm has the signature

\begin{lstlisting}[style = acsl-block]

  void selection_sort(value_type* a, size_type n);
\end{lstlisting}

The \selectionsort algorithm sorts an array in increasing order, left to
right, by selecting in each step the minimum element of the remaining segment
and \emph{swaps} it with its first element.
%
This implies that each member of the increasingly ordered initial segment is less or equal than
each member of the remaining segment.

\begin{figure}[hbt]
\begin{center}
\includegraphics[width=0.65\textwidth]{Figures/selection_sort.pdf}
\caption{An iteration of \selectionsort}
\Label{fig:selectionsort-example}
\end{center}
\end{figure}

\FloatBarrier

Figure~\ref{fig:selectionsort-example} shows a typical situation in an
example run.
The algorithm will swap the \inl{28} at position \inl{i} with the
\inl{9} at position \inl{min} to extend the increasingly ordered initial segment
one field to the right.

\subsection{Formal specification of \selectionsort}

The following listing shows the specification of \selectionsort.

\input{Listings/selection_sort.h.tex}

\clearpage


\subsection{Implementation of \selectionsort}

The implementation of \selectionsort is shown in the next listing.
%
We use \specref{minelement} to find the minimum element of the remaining array segment.

\input{Listings/selection_sort.c.tex}

The loop invariants \inl{increasing} and \inl{lower} establish that the
initial segment \inl{a[0..i-1]} is in increasing order and, respectively,
state that \inl{a[i-1]} is a lower bound of the remaining segment \inl{a[i..n-1]}.
Since the \minelement call uses an address offset, we had
to employ again the \emph{shift lemmas} from the collection \logicref{ArrayBoundsShift}.

The loop invariant \inl{reorder}, on the other hand, states that the multiset of values in the
array \inl{a} are only \emph{rearranged} during the algorithm.
%
While this is intuitively most obvious (as the call to the \specref{swap}
routine, is the only code that modifies~\inl{a}),
it took considerable effort to prove it formally; including a statement contract
that captures the effects of calling \swap.

The main reason for introducing the statement contract is that it
\emph{transforms} the postcondition of the call to \specref{swap}
into the hypotheses for the lemma \logicref{MultisetSwapMiddle}.
This lemma, which relies on the lemmas about \logicref{MultisetReorder},
captures the fact that \emph{swapping two elements of an array} is a \emph{reordering}.

\clearpage



\section{The \insertionsort algorithm}
\Label{sec:insertionsort}

Like \selectionsort,
the algorithm \insertionsort traverses the given array \inl{a[0..n-1]}
left to right, maintaining a left-adjusted, 
constantly increasing range \inl{a[0..i-1]} that is already in increasing order.

Unlike \selectionsort, however, \insertionsort adds \inl{a[i]} to the
initial segment in the \inl{i}th step (see Figure~\ref{fig:insertionsort-example}).
%
It determines the (rightmost) appropriate position to insert \inl{a[i]}
by a call to \specref{upperbound} and then uses \specref{rotatei} to 
perform a \emph{circular shift} to establish the insertion.

\begin{figure}[hbt]
\begin{center}
\includegraphics[width=0.65\textwidth]{Figures/insertion_sort.pdf}
\caption{An iteration of \insertionsort}
\Label{fig:insertionsort-example}
\end{center}
\end{figure}

\FloatBarrier

\subsection{Formal specification of \insertionsort}

The following listing shows our (generic sorting) contract for \insertionsort.

\input{Listings/insertion_sort.h.tex}

\clearpage

\subsection{Implementation of \insertionsort}

The implementation of \insertionsort is shown in the next listing.
%
We used an \acsl statement contract to specify those aspects of the
\rotatei contract that are needed here.
%
Properties related to the result of \insertionsort being in increasing
order are labelled \inl{increasing}.
Properties related to the rearrangement of elements are labelled \inl{reorder} and,
whenever their order isn't changed, \inl{unchanged}.

\input{Listings/insertion_sort.c.tex}

When we originally 
implemented and verified \rotatei, we hadn't yet in mind to
use that function inside of \insertionsort.
%
Consequently, the properties needed for the latter
aren't directly provided by the former.
%
One approach to solve this problem is to add the new properties to
the contract of \specref{rotatei} and repeat its verification proof.
However, if \rotatei is assumed to be part of a pre-verified library,
this approach isn't feasible, since \rotatei's implementation may not
be available for re-verification.
%
Therefore, we used another approach, viz.\ to prove that \rotatei's
original specification \emph{implies} all the properties we need in
\insertionsort.
This is another use of the Hoare calculus' implication rule
(\S\ref{sec:The Implication Rule}).
%
We used several lemmas, shown below,
to make the necessary implications explicit, and to help the provers to
establish them.
%
Some of them needed manual proofs by induction.

\clearpage

Lemma \logicref{IncreasingEqual} in the following listing assumes an ordered range
\inl{a[m..n-1]} and claims that every (elementwise) equal range
range \inl{a[m+p..n+p-1]} is ordered, too.
%
It is needed to establish that the call to \specref{rotatei} preserves the order of
those elements that are shifted upwards 
(cf.\ Figure~\ref{fig:insertionsort-example}).

Similarly, lemma \logicref{CountEqual} says that two elementwise equal ranges
\inl{a[m..n-1]} and \inl{a[p..p+n-m-1]} will result in the same occurrence count,
for each value \inl{v}.
%
This lemma is useful in the proof of the lemma
\logicref{CircularShiftMultisetReorder} (discussed below),
since the predicate \logicref{MultisetReorder}
is defined via the logic function \logicref{Count}.

Lemma \logicref{CircularShiftStrictLowerBound} in the next listing
is used to prove that the range \inl{a[k..i-1]} having 
\inl{a[i]} as strict lower bound before our call to \rotatei ensures
that it has \inl{a[k]} as such a bound after the call.
Note that this lemma reflects that \rotatei is uses as a \emph{circular shift}
at the call site.
%
Similarly, lemma \CircularShiftMultisetReorder establishes that 
a circular shift just reorders the range it is applied to.

\input{Listings/CircularShiftLemmas.acsl.tex}

\clearpage



\section{The \heapsort algorithm}
\Label{sec:heapsort}

The \heapsort algorithm has the signature

\begin{lstlisting}[style = acsl-block]

  void heap_sort(value_type* a, size_type n);
\end{lstlisting}

It relies upon the heap algorithms discussed in Chapter~\ref{cha:heap}
to efficiently transform the array into increasing order.

\FloatBarrier

\subsection{Formal specification of \heapsort}

The following Listing shows the specification of \heapsort.

\input{Listings/heap_sort.h.tex}


\subsection{Implementation of \heapsort}

The implementation of \heapsort, shown in the next listing is straightforward.
%
Given the input array \inl{a[0..n-1]}, we use \specref{makeheap} to
arrange it into a heap; after that, we call \specref{sortheap} to
sort this heap into increasing order.

\input{Listings/heap_sort.c.tex}

\clearpage



\section{The \merge algorithm}
\label{sec:merge}

Our version of the \merge algorithm from the \cxx standard library\cite[28.7.5]{cxx-17-draft}
has the following signature.

\begin{lstlisting}[style = acsl-block]

void
merge(const value_type* a, size_type n,
      const value_type* b, size_type m,
      value_type* result);
\end{lstlisting}

The \merge algorithm is a part of the \emph{merge sort} algorithm.
It operates on the second step to merge two increasingly ordered sub-arrays into a new one.
The algorithm merges two increasingly ordered arrays
\inl{a[0..n-1]} and \inl{b[0..m-1]}, respectively.
The merged values are stored in the output array that starts at
\inl{result} which must be able to hold $m+n$ values of both input arrays.


\subsection{Formal specification of \merge}

The following listing~\ref{lst:merge:spec} shows the specification of \merge.
The specification expects the input arrays of the proper size and in increasing order
and the output array of enough size to contain all the input elements.
The input arrays should not overlap with the output array.
%
In the current edition of this guide, we prove only that the
resulting array is in increasing order.
Future editions will contain additional postconditions stating that the
result array consists of reordered input elements and the stability of the algorithm,
i.e., the same elements of the input arrays preserve their order in the output array.

\input{Listings/merge.h.tex}

%\clearpage

\subsection{More Lemmas on \WeaklyIncreasing}
\label{sec:WeaklyIncreasingLemmas}

We introduce in the following listing several lemmas about \logicref{WeaklyIncreasing}
that are helpful for the verification of \merge.

\begin{itemize}
\item
Lemma \logicref{WeaklyIncreasingShrink} allows to restrict the property
\emph{weakly increasing} onto a sub-array.

\item
Lemma \logicref{WeaklyIncreasingAddElement} defines the way a weakly
increasing array can be constructed.

\item
Lemma \logicref{WeaklyIncreasingShift} is used to handle
pointer arithmetic with respect to the \WeaklyIncreasing property.

\item
Lemmas  \logicref{WeaklyIncreasingUnchanged} and \logicref{WeaklyIncreasingEqual} state
that if an array is weakly increasing,
then another array (or the same array at another program point),
whose elements are in a one-to-one correspondence with the
original array, is also weakly increasing.

\item
Lemma \logicref{WeaklyIncreasingJoin} defines the conditions that two
consequent weakly increasing ranges
can be viewed as merged weakly increasing range.
\end{itemize}

\input{Listings/WeaklyIncreasingLemmas.acsl.tex}

%\clearpage

\subsection{Implementation of \merge}

The implementation of \merge, shown in the next listings is straightforward.
The algorithm operates by traversing both input arrays.
On each iteration it writes the smaller of both elements into the result array,
thus constructing an increasingly ordered array.
If the algorithm reaches the end of one of the input arrays,
it just copies the rest elements of the other array to the result array.
The listing contains a number of assertions to trigger an application of lemmas by 
the provers.
The \inl{while} loop traverses the input arrays and constructs, in accordance with
\logicref{WeaklyIncreasingAddElement}, the resulting weakly increasing array.
After the loop, the algorithm copies the remaining elements to the resulting array.


\begin{listing}[hbt]
\begin{minipage}{\textwidth}
\lstinputlisting[linerange={1-33}, style=acsl-block, frame=single]{Source/merge.c}
\end{minipage}
\caption{\label{lst:merge-impl1}Implementation of \merge (1)}
\end{listing}

\index[examples]{merge@\texttt{merge}}

%\clearpage

We also use the following lemmas to support the verification of several properties.

\begin{itemize}
\item
Lemma \logicref{WeaklyIncreasingEqual} is used to show that the
copied elements from one of the input arrays preserve the \WeaklyIncreasing property.

\item
Lemma \logicref{WeaklyIncreasingJoin} is used
to
extend the \WeaklyIncreasing 
property of the two sub-ranges of the resulting array over the whole range.
In order to deal with pointer arithmetic we employ Lemma \WeaklyIncreasingShift.

\item 
Finally, Lemma \logicref{WeaklyIncreasingIncreasing}
is used to prove the output array is in increasing order.
\end{itemize}

\begin{listing}[hbt]
\begin{minipage}{\textwidth}
\lstinputlisting[linerange={34-100}, style=acsl-block, frame=single]{Source/merge.c}
\end{minipage}
\caption{\label{lst:merge-impl2}The Implementation of \merge (2)}
\end{listing}

\index[examples]{merge@\texttt{merge}}


\clearpage


