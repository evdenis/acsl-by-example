
\section{The \bubblesort algorithm}
\label{sec:bubblesort}

The \bubblesort algorithm traverses the given array \inl{a[0..n-1]} from left
to right, maintaining a right-adjusted, constantly growing range
\inl{a[n-i..n-1]} that is already in increasing order.
We achieve this range by iterating through the array and swapping two adjacent
elements, if their respective value are in the wrong order.


\subsection{Formal specification of \bubblesort}

The following listing shows our (generic sorting) contract for \bubblesort.

\input{Listings/bubble_sort.h.tex}


\subsection{Implementation of \bubblesort}

Our implementation of \bubblesort is shown in the next listing.
As it is typical for \bubblesort, the implementation uses two nested loops.

We first discuss the verification of the fact that
\bubblesort produces an increasing array.
For this we introduce for the \emph{outer loop} the invariant \inl{increasing}.
This loop annotation states that the subrange \inl{a[n-i+1..n-1]} is in increasing order.
An important ingredient on the verification of the \inl{increasing} property
is the claim that the first element~\inl{a[n-i+1]} of the already sorted
subrange is an upper bound of \emph{all} elements left of it.
This claim is encoded in the loop invariant \inl{upper} of the outer bound.
%
In order to support this claim up we exploit the fact
that the index~\inl{j} of the \inl{inner loop} points to the maximum element
of the subrange \inl{a[0..j]}.
We formalize this last property in the loop invariant~\inl{max}.

Note that the loop invariants~\inl{increasing} and~\inl{upper} occur also
in the inner loop. This shall ``assure'' the outer loop that
the inner loop really preserves these properties.

\input{Listings/bubble_sort.c.tex}

We now discuss briefly the verification of the postcondition \inl{reorder}.
In each iteration of the outer loop various elements of the 
not yet sorted subrange \inl{a[0..n-1]} are swapped with their respective 
neighbour.
More specifically, we know for the iteration~\inl{j} of the \emph{inner loop}
that while subrange \inl{a[0..j]} has been rearranged,
the subrange \inl{a[j+1..n-1]} has not been modified yet.
Together this ensures that the loop invariant~\inl{reorder}
holds for the \emph{outer loop}.

\clearpage

