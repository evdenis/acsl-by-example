
\section{The \merge algorithm}
\label{sec:merge}

Our version of the \merge algorithm from the \cxx standard library\cite[28.7.5]{cxx-17-draft}
has the following signature.

\begin{lstlisting}[style = acsl-block]

void
merge(const value_type* a, size_type n,
      const value_type* b, size_type m,
      value_type* result);
\end{lstlisting}

The \merge algorithm is a part of the \emph{merge sort} algorithm.
It operates on the second step to merge two increasingly ordered sub-arrays into a new one.
The algorithm merges two increasingly ordered arrays
\inl{a[0..n-1]} and \inl{b[0..m-1]}, respectively.
The merged values are stored in the output array that starts at
\inl{result} which must be able to hold $m+n$ values of both input arrays.


\subsection{Formal specification of \merge}

The following listing~\ref{lst:merge:spec} shows the specification of \merge.
The specification expects the input arrays of the proper size and in increasing order
and the output array of enough size to contain all the input elements.
The input arrays should not overlap with the output array.
%
In the current edition of this guide, we prove only that the
resulting array is in increasing order.
Future editions will contain additional postconditions stating that the
result array consists of reordered input elements and the stability of the algorithm,
i.e., the same elements of the input arrays preserve their order in the output array.

\input{Listings/merge.h.tex}

%\clearpage

\subsection{More Lemmas on \WeaklyIncreasing}
\label{sec:WeaklyIncreasingLemmas}

We introduce in the following listing several lemmas about \logicref{WeaklyIncreasing}
that are helpful for the verification of \merge.

\begin{itemize}
\item
Lemma \logicref{WeaklyIncreasingShrink} allows to restrict the property
\emph{weakly increasing} onto a sub-array.

\item
Lemma \logicref{WeaklyIncreasingAddElement} defines the way a weakly
increasing array can be constructed.

\item
Lemma \logicref{WeaklyIncreasingShift} is used to handle
pointer arithmetic with respect to the \WeaklyIncreasing property.

\item
Lemmas  \logicref{WeaklyIncreasingUnchanged} and \logicref{WeaklyIncreasingEqual} state
that if an array is weakly increasing,
then another array (or the same array at another program point),
whose elements are in a one-to-one correspondence with the
original array, is also weakly increasing.

\item
Lemma \logicref{WeaklyIncreasingJoin} defines the conditions that two
consequent weakly increasing ranges
can be viewed as merged weakly increasing range.
\end{itemize}

\input{Listings/WeaklyIncreasingLemmas.acsl.tex}

%\clearpage

\subsection{Implementation of \merge}

The implementation of \merge, shown in the next listings is straightforward.
The algorithm operates by traversing both input arrays.
On each iteration it writes the smaller of both elements into the result array,
thus constructing an increasingly ordered array.
If the algorithm reaches the end of one of the input arrays,
it just copies the rest elements of the other array to the result array.
The listing contains a number of assertions to trigger an application of lemmas by 
the provers.
The \inl{while} loop traverses the input arrays and constructs, in accordance with
\logicref{WeaklyIncreasingAddElement}, the resulting weakly increasing array.
After the loop, the algorithm copies the remaining elements to the resulting array.


\begin{listing}[hbt]
\begin{minipage}{\textwidth}
\lstinputlisting[linerange={1-33}, style=acsl-block, frame=single]{Source/merge.c}
\end{minipage}
\caption{\label{lst:merge-impl1}Implementation of \merge (1)}
\end{listing}

\index[examples]{merge@\texttt{merge}}

%\clearpage

We also use the following lemmas to support the verification of several properties.

\begin{itemize}
\item
Lemma \logicref{WeaklyIncreasingEqual} is used to show that the
copied elements from one of the input arrays preserve the \WeaklyIncreasing property.

\item
Lemma \logicref{WeaklyIncreasingJoin} is used
to
extend the \WeaklyIncreasing 
property of the two sub-ranges of the resulting array over the whole range.
In order to deal with pointer arithmetic we employ Lemma \WeaklyIncreasingShift.

\item 
Finally, Lemma \logicref{WeaklyIncreasingIncreasing}
is used to prove the output array is in increasing order.
\end{itemize}

\begin{listing}[hbt]
\begin{minipage}{\textwidth}
\lstinputlisting[linerange={34-100}, style=acsl-block, frame=single]{Source/merge.c}
\end{minipage}
\caption{\label{lst:merge-impl2}The Implementation of \merge (2)}
\end{listing}

\index[examples]{merge@\texttt{merge}}


\clearpage

