
\section{The \insertionsort algorithm}
\Label{sec:insertionsort}

Like \selectionsort,
the algorithm \insertionsort traverses the given array \inl{a[0..n-1]}
left to right, maintaining a left-adjusted, 
constantly increasing range \inl{a[0..i-1]} that is already in increasing order.

Unlike \selectionsort, however, \insertionsort adds \inl{a[i]} to the
initial segment in the \inl{i}th step (see Figure~\ref{fig:insertionsort-example}).
%
It determines the (rightmost) appropriate position to insert \inl{a[i]}
by a call to \specref{upperbound} and then uses \specref{rotatei} to 
perform a \emph{circular shift} to establish the insertion.

\begin{figure}[hbt]
\begin{center}
\includegraphics[width=0.65\textwidth]{Figures/insertion_sort.pdf}
\caption{An iteration of \insertionsort}
\Label{fig:insertionsort-example}
\end{center}
\end{figure}

\FloatBarrier

\subsection{Formal specification of \insertionsort}

The following listing shows our (generic sorting) contract for \insertionsort.

\input{Listings/insertion_sort.h.tex}

\clearpage

\subsection{Implementation of \insertionsort}

The implementation of \insertionsort is shown in the next listing.
%
We used an \acsl statement contract to specify those aspects of the
\rotatei contract that are needed here.
%
Properties related to the result of \insertionsort being in increasing
order are labelled \inl{increasing}.
Properties related to the rearrangement of elements are labelled \inl{reorder} and,
whenever their order isn't changed, \inl{unchanged}.

\input{Listings/insertion_sort.c.tex}

When we originally 
implemented and verified \rotatei, we hadn't yet in mind to
use that function inside of \insertionsort.
%
Consequently, the properties needed for the latter
aren't directly provided by the former.
%
One approach to solve this problem is to add the new properties to
the contract of \specref{rotatei} and repeat its verification proof.
However, if \rotatei is assumed to be part of a pre-verified library,
this approach isn't feasible, since \rotatei's implementation may not
be available for re-verification.
%
Therefore, we used another approach, viz.\ to prove that \rotatei's
original specification \emph{implies} all the properties we need in
\insertionsort.
This is another use of the Hoare calculus' implication rule
(\S\ref{sec:The Implication Rule}).
%
We used several lemmas, shown below,
to make the necessary implications explicit, and to help the provers to
establish them.
%
Some of them needed manual proofs by induction.

\clearpage

Lemma \logicref{IncreasingEqual} in the following listing assumes an ordered range
\inl{a[m..n-1]} and claims that every (elementwise) equal range
range \inl{a[m+p..n+p-1]} is ordered, too.
%
It is needed to establish that the call to \specref{rotatei} preserves the order of
those elements that are shifted upwards 
(cf.\ Figure~\ref{fig:insertionsort-example}).

Similarly, lemma \logicref{CountEqual} says that two elementwise equal ranges
\inl{a[m..n-1]} and \inl{a[p..p+n-m-1]} will result in the same occurrence count,
for each value \inl{v}.
%
This lemma is useful in the proof of the lemma
\logicref{CircularShiftMultisetReorder} (discussed below),
since the predicate \logicref{MultisetReorder}
is defined via the logic function \logicref{Count}.

Lemma \logicref{CircularShiftStrictLowerBound} in the next listing
is used to prove that the range \inl{a[k..i-1]} having 
\inl{a[i]} as strict lower bound before our call to \rotatei ensures
that it has \inl{a[k]} as such a bound after the call.
Note that this lemma reflects that \rotatei is uses as a \emph{circular shift}
at the call site.
%
Similarly, lemma \CircularShiftMultisetReorder establishes that 
a circular shift just reorders the range it is applied to.

\input{Listings/CircularShiftLemmas.acsl.tex}

\clearpage

