
\section{The \selectionsort algorithm}
\Label{sec:selectionsort}

Our version of the \selectionsort algorithm has the signature

\begin{lstlisting}[style = acsl-block]

  void selection_sort(value_type* a, size_type n);
\end{lstlisting}

The \selectionsort algorithm sorts an array in increasing order, left to
right, by selecting in each step the minimum element of the remaining segment
and \emph{swaps} it with its first element.
%
This implies that each member of the increasingly ordered initial segment is less or equal than
each member of the remaining segment.

\begin{figure}[hbt]
\begin{center}
\includegraphics[width=0.65\textwidth]{Figures/selection_sort.pdf}
\caption{An iteration of \selectionsort}
\Label{fig:selectionsort-example}
\end{center}
\end{figure}

\FloatBarrier

Figure~\ref{fig:selectionsort-example} shows a typical situation in an
example run.
The algorithm will swap the \inl{28} at position \inl{i} with the
\inl{9} at position \inl{min} to extend the increasingly ordered initial segment
one field to the right.

\subsection{Formal specification of \selectionsort}

The following listing shows the specification of \selectionsort.

\input{Listings/selection_sort.h.tex}

\clearpage


\subsection{Implementation of \selectionsort}

The implementation of \selectionsort is shown in the next listing.
%
We use \specref{minelement} to find the minimum element of the remaining array segment.

\input{Listings/selection_sort.c.tex}

The loop invariants \inl{increasing} and \inl{lower} establish that the
initial segment \inl{a[0..i-1]} is in increasing order and, respectively,
state that \inl{a[i-1]} is a lower bound of the remaining segment \inl{a[i..n-1]}.
Since the \minelement call uses an address offset, we had
to employ again the \emph{shift lemmas} from the collection \logicref{ArrayBoundsShift}.

The loop invariant \inl{reorder}, on the other hand, states that the multiset of values in the
array \inl{a} are only \emph{rearranged} during the algorithm.
%
While this is intuitively most obvious (as the call to the \specref{swap}
routine, is the only code that modifies~\inl{a}),
it took considerable effort to prove it formally; including a statement contract
that captures the effects of calling \swap.

The main reason for introducing the statement contract is that it
\emph{transforms} the postcondition of the call to \specref{swap}
into the hypotheses for the lemma \logicref{MultisetSwapMiddle}.
This lemma, which relies on the lemmas about \logicref{MultisetReorder},
captures the fact that \emph{swapping two elements of an array} is a \emph{reordering}.

\clearpage

