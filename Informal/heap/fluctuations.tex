

\section{Reorderings and fluctuations}
\label{sec:heap-reordering}

One particular challenge posed by heap algorithms is that
while temporarily causing \emph{small fluctuations} in the number of values
within an array they essentially only \emph{reorder} it,
that is they leave the multiset of its values unchanged.
In this section we will introduce various predicates that will help
us mastering this challenge.

\subsection{Formalizing small fluctuations}

The predicate \MultisetAdd in the following listing
expresses that the number of occurrences of a specific element in an array
has increased by one between two program points~\inl{K} and~\inl{L}.

\input{Listings/MultisetOperations.acsl.tex}

The predicate \MultisetMinus, on the other hand,
expresses that the number of occurrences of a specific element in an array
has decreased by one between two program points~\inl{K} and~\inl{L}.
Note that we have defined \MultisetMinus by calling \MultisetAdd
with the labels reversed.
%
Finally, the predicate \MultisetRetain expresses that a the number
of occurrences of a given value does not change between two program points.
In order to guide the automatic provers, we also provide some
lemmas that formalize conditions under which the respective predicates hold.

Using the predicate \logicref{MultisetReorder} and the logic function \logicref{At}
we also formulate a few simple lemmas that describe when the
predicates from Listing~\logicref{MultisetOperations} hold.

\subsection{Simple properties of fluctuations}

The predicate \logicref{MultisetRetainRest} uses \logicref{MultisetRetain}
in order to express that all values of an array,
except the two given values \inl{u} and \inl{v}, occur as often in program
point \inl{K} and program point \inl{L}. 

The lemmas in this listing express conditions under which small
fluctuations---expressed by the predicates \logicref{MultisetAdd}
and \logicref{MultisetMinus}---in the number of occurrences between three
program points even with each other.

\input{Listings/MultisetRetainRest.acsl.tex}


\subsection{Combining fluctuations}

Small fluctuations are so prevalent in the central heap algorithms \implref{pushheap}
and \implref{popheap} that it is worthwhile to introduce another predicate
to concisely capture this feature.
We refer to this predicate as \logicref{MultisetParity} because it describes
the situation where the number of occurrences 

\begin{itemize}
\item of the first of two given values increases by one
\item while that of the second value decreases by one
\item and the remaining values retain their respective number of occurrences.
\end{itemize}

With this predicate we can formulate several lemmas that describe useful
combinations of reorderings and fluctuations.
For example, lemma \logicref{MultisetParityMultisetReorder} describes
the situation where two fluctuation cancel each other and consequently
establish a reordering of an array.

\input{Listings/MultisetParity.acsl.tex}

\subsection{How do fluctuations arise?}

The simplest way to creation a small fluctuation is to
update an array element with a different value.
Thus, similar to the predicate \logicref{ArrayUpdate} we introduce 
predicate \logicref{MultisetUpdate} which in turn relies on \logicref{MultisetParity}.
Lemma \logicref{ArrayUpdateMultisetUpdate} then formalizes the claim
that updating an array element with a different value creates a small fluctuation.

\input{Listings/MultisetUpdate.acsl.tex}

\clearpage

