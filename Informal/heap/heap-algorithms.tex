\chapter{Heap Algorithms}
\Label{cha:heap}

The heap algorithms of the \cxx Standard 
Library \cite[28.7.7]{cxx-17-draft}
were already part of \emph{\acsl by Example} from 2010--2012.
In this chapter we re-introduce them and discuss---based on the
bachelor thesis of one of the authors---the verification efforts in some 
detail \cite{Lapawczyk_2016_bachelor}.


The \cxx standard\footnote{
  See \url{http://www.open-std.org/jtc1/sc22/wg21/docs/papers/2011/n3242.pdf}
} introduces the concept of a \emph{heap} as follows:

\begin{small}
\begin{quote}
\begin{enumerate}
\item A \emph{heap} is a particular organization of elements in a range between two
random access iterators \inl{[a,b)}. Its two key properties are:
\begin{enumerate}
\item There is no element greater than \inl{*a} in the range and
\item \inl{*a} may be removed by \inl{pop_heap()}, or a new element added by \inl{push_heap()}, in
       $O(\log(N))$ time.
\end{enumerate}
\item These properties make heaps useful as priority queues.
\item \inl{make_heap()} converts a range into a heap and \inl{sort_heap()}
      turns a heap into an increasing sequence.
\end{enumerate}
\end{quote}
\end{small}


Figure~\ref{fig:heap-overview} gives an overview on the five heap algorithms
by means of an example.
Algorithms, which in a typical implementation are in a caller-callee relation, have the same color.

\begin{figure}[hbt]
\centering
\includegraphics[width=0.85\linewidth]{Figures/heap-overview.pdf}
\caption{\Label{fig:heap-overview}Overview on heap algorithms}
\end{figure}

\clearpage

Roughly speaking, the algorithms from Figure~\ref{fig:heap-overview} have
the following behavior.

\begin{itemize}
\item In \S\ref{sec:heap-concepts} we briefly recapitulate basic
      heap concepts.

\item In \S\ref{sec:heap-acsl} we show how these heap concepts
      can be described in \acsl.

\item In \S\ref{sec:auxiliary-heap-functions} we verify two
      auxiliary heap functions.

\item The algorithms \isheapuntil and \isheap from
      \S\ref{sec:isheapuntil} and~\S\ref{sec:isheap}
      allow to test at run time whether a given array is arranged as a heap

\item The algorithm \pushheap from \S\ref{sec:pushheap} \emph{adds} an
        element to a given heap in such a way
        that resulting array is again a heap

\item The algorithm \popheap from \S\ref{sec:popheap}, on the other hand,
        \emph{removes} an element from a given heap in
        such a way that the resulting array is again a heap

\item The algorithm \makeheap from \S\ref{sec:makeheap} rearranges a given array
        into a heap.

\item Finally, the algorithm \sortheap from \S\ref{sec:sortheap} sorts a heap
        into an increasing range.
\end{itemize}


In \S\ref{sec:heap-concepts} we present in more detail how heaps are defined.
The \acsl logic functions and predicate that formalize the basic heap
properties of heaps are introduced in \S\ref{sec:heap-acsl}.


\clearpage


\section{Basic heap concepts}
\Label{sec:heap-concepts}

The description of heaps at the beginning of this chapter is of course fairly vague.
It outlines only the most important properties
of various operations but does not clearly state what specific and verifiable
properties a range must satisfy such that it may be called a heap.

A more detailed description can be found in the Apache \cxx Standard Library User's Guide:\footnote{
  See \url{http://stdcxx.apache.org/doc/stdlibug/14-7.html}
}

\begin{quote}
A heap is a binary tree in which every node is larger than the values
associated with either child. A heap and a binary tree, for that matter,
can be very efficiently stored in a vector, by placing the children of
node $i$
at positions $2i + 1$ and $2i + 2$.
\end{quote}

We have, in other words, the following basic relations between indices of a heap:

\begin{align}
\Label{eq:heap-left}
   &\text{left child for index $i$}   && \mathrm{child_l}: i \mapsto 2i + 1  \\
\Label{eq:heap-right}
   &\text{right child for index $i$}  && \mathrm{child_r}: i \mapsto 2i + 2  \\
\intertext{and}
\Label{eq:heap-parent}
   &\text{parent index for index $i$}  && \mathrm{parent}: i \mapsto \frac{i - 1}{2}
\end{align}

%\clearpage 

These function are related through the following two equations
that hold for all integers~$i$.
Note that in \acsl integer division rounds towards zero (cf.\ \cite[\S 2.2.4]{ACSLSpec}).

\begin{align}
\Label{eq:heap-parent-left}
   \mathrm{parent}(\mathrm{child_l}(i)) &= i \\
\Label{eq:heap-parent-right}
   \mathrm{parent}(\mathrm{child_r}(i)) &= i
\end{align}


In order to given an example for the usefulness of heaps
we consider the following multiset of integers $X$.

\begin{align}
\Label{eq:heap-multiset}
  X &= \{2,3,3,3,6,7,8,8,9,11,13,14\}
\end{align}

\clearpage

Figure~\ref{fig:heap-tree} shows how the multiset from Equation~\eqref{eq:heap-multiset} 
can, according to the parent-child relations of a heap, be represented as a tree.

\begin{figure}[hbt]
\centering
\includegraphics[width=0.75\linewidth]{Figures/heap_tree_color.pdf}
\caption{\Label{fig:heap-tree}Tree representation of the multiset~$X$}
\end{figure}

\FloatBarrier

The numbers outside the nodes in Figure~\ref{fig:heap-tree} are the indices at which
the respective node value is stored in the underlying array of a heap (cf.\ Figure~\ref{fig:heap-array}).

\begin{figure}[hbt]
\centering
\includegraphics[width=0.65\linewidth]{Figures/heap_array_color.pdf}
\caption{\Label{fig:heap-array}Underlying array of a heap}
\end{figure}

\FloatBarrier
\clearpage

It is important to understand that there can be various representations of a multiset
as a heap.
Figure~\ref{fig:heap-alternative-tree}, for example, arranges the elements of
the multiset~$X$ as a heap in a different tree.

\begin{figure}[hbt]
\centering
\includegraphics[width=0.75\linewidth]{Figures/heap_tree_alternative_color.pdf}
\caption{\Label{fig:heap-alternative-tree} An alternative representation of the multiset~$X$}
\end{figure}

\FloatBarrier

Figure~\ref{fig:heap-array-alternative} then shows the underlying array that 
corresponds to the tree in Figure~\ref{fig:heap-alternative-tree}.

\begin{figure}[hbt]
\centering
\includegraphics[width=0.65\linewidth]{Figures/heap_array_alternative_color.pdf}
\caption{\Label{fig:heap-array-alternative}Underlying array of the alternative representation}
\end{figure}

\FloatBarrier
\clearpage



\section{Representation of heap concepts in \acsl}
\Label{sec:heap-acsl}

The following listing shows three logic functions
\HeapLeft, \HeapRight and \HeapParent
that correspond to the definitions~\eqref{eq:heap-left},
\eqref{eq:heap-right} and~\eqref{eq:heap-parent}, respectively.
This listing also contains a number of \acsl lemma that state among other things that

\begin{itemize}
\item
the \HeapParent function satisfies the equations~\eqref{eq:heap-parent-left}
and~\eqref{eq:heap-parent-right} and
\item
the function \HeapParent 
is the \emph{left inverse} to the \HeapLeft and \HeapRight functions.\footnote{
 See Section \emph{Left and right inverses} at
 \url{http://en.wikipedia.org/wiki/Inverse_function}
}
\end{itemize}

\input{Listings/HeapNodes.acsl.tex}

\clearpage

On top of these basic definitions we introduce the predicate \logicref{Heap}.
The fact that element at index~0 of a (maximum) heap, is always the largest element of the heap
is express by Lemma \logicref{HeapMaximum} using the predicate \logicref{MaxElement}.

\input{Listings/Heap.acsl.tex}

The lemmas \HeapShrink and \HeapUnchanged formulate simple rules to
``transfer'' the heap property from an array to a related (sub-)array.

The predicate \HeapCompatible expresses under which
conditions the changing of an individual heap element does maintain the heap
property.
This predicate together with lemma \HeapCompatibleUpdate 
will be useful in the verification of the 
algorithms \implref{pushheap} and \implref{popheap}.

\clearpage



\section{The auxiliary functions \heapparent and \heapchild}
\label{sec:auxiliary-heap-functions}
\label{sec:heapparent}
\label{sec:heapchild}

This section features the two auxiliary heap functions
We start with the function \specref{heapparent}
which is in principle the \isoc~counterpart of the \acsl function \logicref{HeapParent}.
We say \emph{in principle} because our definition avoids
the border case of the parent node of~0.

\input{Listings/heap_parent.h.tex}

Neither do we provide exact \isoc-counterparts for
the logic functions \logicref{HeapLeft} and \logicref{HeapRight}.
In fact, we have encountered only one situation (in the implementation of \implref{popheap}),
where such functions would have been useful.
However, what we really need in \popheap is to determine
for a given index~\inl{p} a child index~\inl{c} where the maximum
of the respective values \inl{a[HeapLeft(p)]} and\\
\inl{a[HeapRight(p)]} resides.
This computation is performed by the function \specref{heapchild}.

\input{Listings/heap_child.h.tex}

\clearpage

Note that in the implementation of \implref{heapchild}
we explicitly handle the case that the computation of child indices
could overflow. If this occurs, the function \heapchild returns~\inl{n}.

\input{Listings/heap_child.c.tex}




\section{The \isheapuntil algorithm}
\Label{sec:isheapuntil}

The \isheapuntil algorithm of the \cxx Standard Library \cite[\S
28.7.7.5]{cxx-17-draft} works on generic sequences. 
For our purposes we have modified the generic implementation
to that of an array of type \valuetype.
The signature now reads:

\begin{lstlisting}[style = acsl-block]

    size_type is_heap_until(const value_type* a, int n);
\end{lstlisting}

The algorithm \isheapuntil returns the largest range of an array, beginning at the first position, where it still satisfies the heap properties
we have semi-formally described in the beginning of this chapter.
In particular, \isheapuntil will return the size of the array,
called with the array argument from Figure~\ref{fig:heap-array}.

\clearpage

\subsection{Formal specification of \isheapuntil}

The specification of \isheapuntil is shown in the following listing.
The index \inl{\\result} returned by \isheapuntil indicates
that the array \inl{a[0..\\result-1]} is a heap.
In addition the postcondition \inl{last} states, that for all indices
greater than or equal to \inl{i} the predicate \logicref{Heap} is not satisfied.

\input{Listings/is_heap_until.h.tex}

\subsection{Implementation of \isheapuntil}

The following listing shows one way to implement the function \isheapuntil.

\input{Listings/is_heap_until.c.tex}

The algorithms starts at the index~1, which is the smallest index,
where a child node of the heap might reside.
The algorithms checks for each (child) index whether
the value at the corresponding parent index 
is greater than or equal to the value at the child index.
If the value at a parent index is smaller than the value at a (child) index,
\isheapuntil returns the (child) index.
Otherwise, if the algorithm iterates through the whole array,
the size of the array is returned.


\section{The \isheap algorithm}
\Label{sec:isheap}

The \isheap algorithm of the \cxx Standard Library \cite[\S 28.7.7.5]{cxx-17-draft}
works on generic sequences. 
For our purposes we have modified the generic implementation
to that of an array of type \valuetype.
The signature now reads:

\begin{lstlisting}[style = acsl-block]

    bool is_heap(const value_type* a, int n);
\end{lstlisting}

The algorithm \isheap checks whether a given array satisfies the heap properties
we have semi-formally described in the beginning of this chapter.
In particular, \isheap will return \inl{true}
called with the array argument from Figure~\ref{fig:heap-array}.

%\clearpage

\subsection{Formal specification of \isheap}

The specification of \isheap is shown in the following listing.
The function returns~\inl{true} if and only if the input array
satisfies the predicate \logicref{Heap}.

\input{Listings/is_heap.h.tex}

\subsection{Implementation of \isheap}

Our implementation of \isheap in the following listing
utilizes the function \specref{isheapuntil}.

\input{Listings/is_heap.c.tex}

\clearpage




\section{Reorderings and fluctuations}
\label{sec:heap-reordering}

One particular challenge posed by heap algorithms is that
while temporarily causing \emph{small fluctuations} in the number of values
within an array they essentially only \emph{reorder} it,
that is they leave the multiset of its values unchanged.
In this section we will introduce various predicates that will help
us mastering this challenge.

\subsection{Formalizing small fluctuations}

The predicate \MultisetAdd in the following listing
expresses that the number of occurrences of a specific element in an array
has increased by one between two program points~\inl{K} and~\inl{L}.

\input{Listings/MultisetOperations.acsl.tex}

The predicate \MultisetMinus, on the other hand,
expresses that the number of occurrences of a specific element in an array
has decreased by one between two program points~\inl{K} and~\inl{L}.
Note that we have defined \MultisetMinus by calling \MultisetAdd
with the labels reversed.
%
Finally, the predicate \MultisetRetain expresses that a the number
of occurrences of a given value does not change between two program points.
In order to guide the automatic provers, we also provide some
lemmas that formalize conditions under which the respective predicates hold.

Using the predicate \logicref{MultisetReorder} and the logic function \logicref{At}
we also formulate a few simple lemmas that describe when the
predicates from Listing~\logicref{MultisetOperations} hold.

\subsection{Simple properties of fluctuations}

The predicate \logicref{MultisetRetainRest} uses \logicref{MultisetRetain}
in order to express that all values of an array,
except the two given values \inl{u} and \inl{v}, occur as often in program
point \inl{K} and program point \inl{L}. 

The lemmas in this listing express conditions under which small
fluctuations---expressed by the predicates \logicref{MultisetAdd}
and \logicref{MultisetMinus}---in the number of occurrences between three
program points even with each other.

\input{Listings/MultisetRetainRest.acsl.tex}


\subsection{Combining fluctuations}

Small fluctuations are so prevalent in the central heap algorithms \implref{pushheap}
and \implref{popheap} that it is worthwhile to introduce another predicate
to concisely capture this feature.
We refer to this predicate as \logicref{MultisetParity} because it describes
the situation where the number of occurrences 

\begin{itemize}
\item of the first of two given values increases by one
\item while that of the second value decreases by one
\item and the remaining values retain their respective number of occurrences.
\end{itemize}

With this predicate we can formulate several lemmas that describe useful
combinations of reorderings and fluctuations.
For example, lemma \logicref{MultisetParityMultisetReorder} describes
the situation where two fluctuation cancel each other and consequently
establish a reordering of an array.

\input{Listings/MultisetParity.acsl.tex}

\subsection{How do fluctuations arise?}

The simplest way to creation a small fluctuation is to
update an array element with a different value.
Thus, similar to the predicate \logicref{ArrayUpdate} we introduce 
predicate \logicref{MultisetUpdate} which in turn relies on \logicref{MultisetParity}.
Lemma \logicref{ArrayUpdateMultisetUpdate} then formalizes the claim
that updating an array element with a different value creates a small fluctuation.

\input{Listings/MultisetUpdate.acsl.tex}

\clearpage



\section{The \pushheap algorithm}
\Label{sec:pushheap}

The \pushheap algorithm assumes that the first $n-1$ elements of an array
of length~$n$ form already a heap and adds to it the element \inl{a[n-1]}.

Whereas in the \cxx Standard Library \cite[\S 28.7.7.1]{cxx-17-draft}
\pushheap works on a range of random access iterators,
our version operates on an array of \valuetype.
We therefore use the following signature for \pushheap

\begin{lstlisting}[style = acsl-block]

    void push_heap(value_type* a, size_type n);
\end{lstlisting}


The \pushheap algorithm expects that \inl{n} is greater or equal than~1.
It also assumes that the array\\
\inl{a[0..n-2]} forms a heap.
The algorithms then \emph{rearranges} the array \inl{a[0..n-1]} such that the 
resulting array is a heap.
In this sense the algorithm \emph{pushes} the element \inl{a[n-1]} on the given heap.

\subsection{Formal specification of \pushheap}

The following listing shows our specification of \pushheap.
Note that the post condition \inl{reorder} states
that \pushheap is not allowed to change the number of occurrences
of an array element.
Without this post condition,
an implementation that assigns \inl{0} to each array element
would satisfy the post condition \inl{heap}---surely
not what a user of the algorithm has in mind.

\input{Listings/push_heap.h.tex}

Pushing an element on a heap usually \emph{rearranges} several elements of the array
(cf.\xspace Figures~\ref{fig:pushheap-pre} and~\ref{fig:pushheap-post}).
We therefore must be able express that \pushheap only reorders
the elements of the array.
We re-use the predicate \logicref{MultisetReorder} to formally describe this property.


\subsection{Implementation of \pushheap}

The following two figures illustrate how \pushheap affects an array,
which is shown as a tree with blue and grey nodes, representing
heap and non-heap nodes, respectively.
Figure~\ref{fig:pushheap-pre} shows the heap from
Figure~\ref{fig:heap-tree} together with the additional element~\inl{12} that
is to be pushed on the heap.
To be quite clear about it: the new element~\inl{12} is the last element of the
array and not yet part of the heap.

\begin{figure}[hbt]
\centering
\includegraphics[width=0.70\linewidth]{Figures/push_heap_pre.pdf}
\caption{\Label{fig:pushheap-pre}Example heap before the call of \pushheap}
\end{figure}

\FloatBarrier

Figure~\ref{fig:pushheap-post} shows the array
after the call of \pushheap.
We can see that now all nodes are colored in blue, i.e., they are part of the heap.
The dashed nodes highlight which heap nodes have changed during the function call.
The element to be pushed into the heap is now at its correct position.
The arrows indicate the \emph{cyclic reordering} of array elements to achieve the
desired result.

\begin{figure}[hbt]
\centering
\includegraphics[width=0.70\linewidth]{Figures/push_heap_post.pdf}
\caption{\Label{fig:pushheap-post}Example heap after the call of \pushheap}
\end{figure}

\FloatBarrier


Verifying our implementation of \implref{pushheap} is a non-trivial undertaking.
In order to better structure our discussion we refer to the central
loop of the algorithm as the \emph{main act} and the parts before and after
it as \emph{prologue} and \emph{epilogue}.

We can establish the \inl{heap} property of \specref{pushheap}
already in the prologue.
The \inl{reorder} property, however, only holds at the function boundaries
and is violated while \pushheap manipulates the array.
To be more precise:
We loose the \inl{reorder} property in the prologue and
formally capture and maintain a slightly more general property
in the main act.
From this we will recover the \inl{reorder} property in the epilogue.

We will illustrate the changes to the underlying array after each stage 
by figures of the array in tree form,
based on the \pushheap example from Figure~\ref{fig:pushheap-pre}.

\input{Listings/push_heap.c.tex}

\subsubsection{Prologue}
\Label{sec:push-heap-prologue}

In the prologue we check whether the initial heap is nonempty,
initialize some variables,
\emph{and} also check by comparing with the parent node
whether \inl{a[c]}, which is the value to be pushed on the heap and which
is the last element of the array, is by chance already at the right place.
If not we set aside this value in the variable \inl{v} and
assign the parent value \inl{a[p]} to \inl{a[c]}.
Note that this assignment only occurs if the respective values
differ. 
This allows us to formally describe the effect of the assignment using the predicate
\logicref{ArrayUpdate}.
%
Figure~\ref{fig:pushheap-prologue} highlights the main effects of the prologue.
Here and in the following figures we highlight the currently active node.

\begin{figure}[hbt]
\centering
\includegraphics[width=0.70\linewidth]{Figures/push_heap_prologue.pdf}
\caption{\Label{fig:pushheap-prologue}Heap after the prologue of \pushheap}
\end{figure}

At this point we have achieved several things.
\begin{enumerate}
\item The array \inl{a[0..n-1]} is now a heap.
\item Regarding their respective number of occurences in the array \inl{a[0..n-1]}
\begin{itemize}
\item the original value \inl{a[c]} occurs one time less
\item the original value \inl{a[p]} occurs one time more
\item whereas all other values have not changed their number of occurences.
\end{itemize}
\end{enumerate}

The first observation is expressed in the assertion \inl{heap}
whereas the small fluctuation of array elements described in
the second observation is expressed by using the predicate \logicref{MultisetParity}
in the assertion \inl{reorder}.


\subsubsection{Main act}
\Label{sec:push-heap-main-act}

In the main act, we start at the parent location, which is now stored
in the variable \inl{c} (\emph{child}).
Compared to the pre-state of \pushheap
at the beginning of the main act the array \inl{a[0..n-1]} 
\begin{itemize}
\item contains the value \inl{v} one time less
\item contains the value \inl{a[c]} one time more
\item whereas all other values have not changed their number of occurences.
\end{itemize}

Now, as long as the index~\inl{c} is not yet the root of the heap
and its consequently existing parent value \inl{a[p]} is less than \inl{v},
we haven't found yet an index \inl{c} where we could insert \inl{v} without
violating the heap property.

In the loop body we proceed as follows.

\begin{itemize}
\item
If \inl{a[c]} is less than \inl{a[p]} we copy the latter value on the former.
Note that this assignment preserves the heap property of the array.
The value \inl{a[p]} now occurs one time more than in the pre-state whereas
the now overwritten value \inl{a[c]} occurs as often as in the pre-state.
The value \inl{v} continues to occur one time less.
%
We then proceed to the next iteration by setting \inl{c} to \inl{p}.

The verification of tracking the number of occurences happens
in smaller steps than just described.
It relies on the predicates \logicref{ArrayUpdate} and \logicref{MultisetUpdate}
which we can apply in this guarded assignment.
Lemma \logicref{MultisetParityCombined} also plays an important role here.

\item
Otherwise, since \inl{c} is a child of \inl{p},  we can conclude that
\inl{a[c]} equals \inl{a[p]} and we continue with the next iteration
after setting \inl{c} to \inl{p}.
\end{itemize}

This means that at the begin of the next iteration again
the following conditions hold.
Compared to the pre-state of \pushheap the array \inl{a[0..n-1]} 

\begin{itemize}
\item contains the value \inl{v} one time less
\item contains the value \inl{a[c]} one time more
\item whereas all other values have not changed their number of occurences.
\end{itemize}

Figure~\ref{fig:pushheap-main-act} shows the our example heap after the main act.
For this particular heap, only one iteration is performed until a node is reached
whose parent value \inl{a[p]} is greater or equal than \inl{v}.
Note that assignments which have previously occurred are marked with dashed arrows. 

\begin{figure}[hbt]
\centering
\includegraphics[width=0.70\linewidth]{Figures/push_heap_main_act.pdf}
\caption{\Label{fig:pushheap-main-act}Heap after the main act of \pushheap}
\end{figure}

\FloatBarrier
\clearpage

\subsubsection{Epilogue}
\Label{sec:push-heap-epilogue}

At this point, we have arrived at an index \inl{c} where 
the assignment of the value \inl{v} preserves the heap property.
We express this formally using the predicate \logicref{HeapCompatible}.

Moreover, this assignment also corrects the imbalance in the number of
occurences of the values \inl{a[c]} and \inl{v} and consequently
establishes the desired property \inl{reorder} of \pushheap.
The verification that this correction leads to a proper reordering relies
on lemma \logicref{MultisetParityMultisetReorder}.

Figure~\ref{fig:pushheap-epilogue} shows the final assignment and highlights the
completion of the cycle depicted in Figure~\ref{fig:pushheap-prologue}.
The figure also makes clear that the value \inl{v} acts like an additional
element in this assignment cycle.

\begin{figure}[hbt]
\centering
\includegraphics[width=0.70\linewidth]{Figures/push_heap_epilogue.pdf}
\caption{\Label{fig:pushheap-epilogue}Heap after the epilogue of \pushheap}
\end{figure}

\clearpage


\section{The \popheap algorithm}
\Label{sec:popheap}

The algorithm \popheap moves the first element of the heap, which holds
the heap's largest value, and places it at the the end of the underlying sequence.
Whereas in the \cxx Standard Library \cite[\S 28.7.7.2]{cxx-17-draft}
\popheap works on a range of random access iterators,
our version operates on an array of \valuetype.
We therefore use the following signature for \popheap

\begin{lstlisting}[style = acsl-block]

    void pop_heap(value_type* a, size_type n);
\end{lstlisting}

The \popheap algorithm expects that \inl{n} is greater or equal than~1
and that the array \inl{a[0..n-1]} forms a heap.
The algorithms then \emph{rearranges} the array \inl{a[0..n-1]} such that the
resulting array satisfies the following properties.

\begin{itemize}
\item \inl{a[n-1] = \\old(a[0])}, that is, the largest element
of the original heap is transferred to the end of the array.

\item the subarray \inl{a[0..n-2]} is a heap
\end{itemize}

In this sense the algorithm \emph{pops} the largest element from a heap.

\subsection{Formal specification of \popheap}

Based on the above semi-formal description we propose the
following function contract for \specref{popheap}.

\input{Listings/pop_heap.h.tex}

\subsection{Implementation of \popheap}
\Label{sec:pop-heap:impl}

In an abstract sense \popheap is quite similar to \pushheap.
In  \pushheap we started at the last array element and 
climbed from there up the tree until we would find a node where to
insert the new value into the heap.
Every time we had reached the next parent node we
moved its value down to where we had just come from.

With \popheap its the other way round.
We start at the root of the tree and descend from there
by selecting an appropriate child.
Every time we lift the value of the selected child to the node where
just are.
We repeat this process until we find a node where we can insert
the last array element into the heap.
Once this is done, we can safely place the maximum element (that is the
the original root node) at the last element of the array.

\clearpage

The following two figures illustrate how \popheap affects an array,
which is shown again as a tree with blue and grey nodes, representing
heap and non-heap nodes, respectively.
Figure~\ref{fig:popheap-pre} is in fact the same figure as
Figure~\ref{fig:heap-tree}.

\begin{figure}[hbt]
\centering
\includegraphics[width=0.70\linewidth]{Figures/pop_heap_pre.pdf}
\caption{\Label{fig:popheap-pre}Heap before the call of \popheap}
\end{figure}

\FloatBarrier

Figure~\ref{fig:popheap-post}, on the other hand, shows the heap after the call of \popheap
together with arrows that indicate how our implementation moves around elements
in the underlying array.
We can see that the first element of the original array,
where the maximum of the heap resides, is now the last element of the array.
Furthermore, the last array element is not part of the heap anymore.
The dashed nodes highlight which heap nodes have changed during the call to \popheap.
The arrows indicate the \emph{cyclic reordering} of array elements to achieve the
desired result.

\begin{figure}[hbt]
\centering
\includegraphics[width=0.70\linewidth]{Figures/pop_heap_post.pdf}
\caption{\Label{fig:popheap-post}Heap after the call of \popheap}
\end{figure}

\FloatBarrier


As in the case of \implref{pushheap} we will subdivide the discussion of the 
implementation of \implref{popheap} into a prologue, main act, and epilogue.

\input{Listings/pop_heap.c.tex}


\subsection{Prologue}

In the prologue we check whether the initial heap contains at least two elements,
initialize some variables, and also check whether
the last array element is by chance equal to the maximum element of the heap,
which resides at the index \inl{p == 0} of the array.
If this is not the case, then we set aside for future
reference the last array element in the variable \inl{v}.
Finally we copy the value \inl{a[p]} to its final destination at the end
of the array.
Note that this assignment only occurs if the respective values differ.
This allows us, as in the case of \implref{pushheap}, to formally describe the
effect of the assignment using the predicate \logicref{ArrayUpdate}.

Figure~\ref{fig:popheap-prologue} highlights the main effects of the prologue
at the hand of our exemplary heap.
Note that we have highlighted the root of the heap as the currently active node.

\begin{figure}[hbt]
\centering
\includegraphics[width=0.65\linewidth]{Figures/pop_heap_prologue.pdf}
\caption{\Label{fig:popheap-prologue}Heap after the prologue of \popheap}
\end{figure}

\FloatBarrier


\subsection{Main act}

In the main act, we start at a child node \inl{c} of the prologue's index \inl{p}.
This means that compared to the pre-state of \popheap
at the beginning of the main act the array \inl{a[0..n-1]}
\begin{itemize}
\item contains the value \inl{v} one time less
\item contains the value \inl{a[p]} one time more
\item whereas all other values have not change their number of occurences.
\end{itemize}

Moreover, the maximum element of the original heap is now at the end of the array
and we can only guarantee that the first $n-1$ array elements got a heap.
These observations are necessary reason for our loop invariants.

To be more precise, when we talk in the context of \popheap
of a \emph{child node} we usually mean one of the possibly two children
where the maximum of the values resides.
We do this because copying that larger value to its parent node guarantees
that the resulting tree is still a heap.
We compute the maximum child of a node using the function \specref{heapchild}.

Now, as long as the index~\inl{c} is not yet the index of the last array element
of the heap and its value \inl{a[c]} is less than \inl{v},
we haven't found yet an index where we could insert \inl{v} without
violating the heap property.

\clearpage

In the loop body we proceed as follows.

\begin{itemize}
\item
If \inl{a[c]} is less than \inl{a[p]} we copy the former value on 
the latter.
As mentioned above, using the index~\inl{c} of the maximum child
maintains  heap property of the array.
We use here the predicate \logicref{HeapCompatible} to express
that the insertion of the new value \inl{a[p]} maintains the heap
property of the array.

The value \inl{a[c]} now occurs one time more than in the pre-state whereas
the now overwritten value \inl{a[p]} occurs as often as in the pre-state of \popheap.
The value \inl{v} continues to occur one time less than in the pre-state.
%
We then proceed to the next iteration by setting \inl{p} to \inl{c}
and computing the next maximum child node.

As in the case of \implref{pushheap} the verification of 
the correct number of occurences of the involved values
relies on the predicates \logicref{ArrayUpdate} and \logicref{MultisetUpdate}
and on lemma \logicref{MultisetParityCombined}.

\item
Otherwise, the array being a heap, we can conclude that
\inl{a[c]} equals \inl{a[p]} and we continue with the next iteration
after setting \inl{p} to \inl{c} and computing the corresponding
new maximum child node.
\end{itemize}

The following three figures depict how the main act of \popheap 
modifies step by step our example heap.
In each step we highlight the currently active node \inl{c}.


\begin{figure}[hbt]
\centering
\includegraphics[width=0.65\linewidth]{Figures/pop_heap_main_act1.pdf}
\caption{\Label{fig:popheap-main_act1}Heap after the first iteration of of \popheap}
\end{figure}

\FloatBarrier

\begin{figure}[hbt]
\centering
\includegraphics[width=0.65\linewidth]{Figures/pop_heap_main_act2.pdf}
\caption{\Label{fig:popheap-main_act2}Heap after the second iteration of \popheap}
\end{figure}

\FloatBarrier
\clearpage

Note that in the final step no value is actually copied as the involved nodes
hold the same value.

\begin{figure}[hbt]
\centering
\includegraphics[width=0.65\linewidth]{Figures/pop_heap_main_act3.pdf}
\caption{\Label{fig:popheap-main_act3}Heap after the third iteration of \popheap}
\end{figure}

\FloatBarrier

We finally remark that in the main act the the last array element is never modified.
Thus, the root element of the original element is still safely stored there.

\subsection{Epilogue}

After leaving the loop, we know that value \inl{v} can be the inserted
in the array at the index \inl{p} without violating the heap property of
the first $n-1$ elements.
%
Moreover, compared to the pre-state of \popheap the array \inl{a[0..n-1]} still
\begin{itemize}
\item contains the value \inl{v} one time less
\item contains the value \inl{a[p]} one time more
\item whereas all other values have not change their number of occurences.
\end{itemize}

In other words, assigning the value \inl{v} to \inl{a[p]}
cancels this imbalance and establishes that \popheap
only reorders the array elements.

\begin{figure}[hbt]
\centering
\includegraphics[width=0.65\linewidth]{Figures/pop_heap_epilogue.pdf}
\caption{\Label{fig:popheap-epilogue}Heap after the epilogue of \popheap}
\end{figure}

\FloatBarrier

In Figure~\ref{fig:popheap-epilogue} we have marked the value \inl{v}
as the currently active node despite not being an array element.

\clearpage



\section{The \makeheap algorithm}
\Label{sec:makeheap}


Whereas in the \cxx Standard Library \cite[\S 28.7.7.3]{cxx-17-draft}
\makeheap works on a pair of generic random access
iterators,
our version operators on a range of \valuetype.
Thus the signature of \makeheap reads

\begin{lstlisting}[style = acsl-block]

    void make_heap(value_type* a, size_type n);
\end{lstlisting}

The function \makeheap rearranges the elements of the given
array \inl{a[0..n-1]} such that they form a heap.

As an examples we look at the array in Figure~\ref{fig:makeheap-array-pre}.
The elements of this array do not form a heap, as indicated by the grey colouring.
Executing the \makeheap algorithm on this array rearranges its elements
so that they form a heap as shown in Figure~\ref{fig:heap-array}.

\begin{figure}[hbt]
\centering
\includegraphics[width=0.65\linewidth]{Figures/make_heap_array_pre.pdf}
\caption{\Label{fig:makeheap-array-pre}Array before the call of \makeheap}
\end{figure}


\FloatBarrier

\subsection{Formal specification of \makeheap}

The following listing shows the specification of \makeheap.

\input{Listings/make_heap.h.tex}

Like with \pushheap the formal specification of \makeheap
must ensure that the resulting array is a heap of size \inl{n}
and contains the same multiset of elements as in the pre-state of the function.
These properties are expressed by the \inl{heap} and \inl{reorder}
postconditions respectively.
The \inl{reorder} postcondition uses the predicate \logicref{MultisetReorder}
to ensure that \makeheap only rearranges the array elements.

\clearpage

\subsection{Implementation of \makeheap}

The implementation of \makeheap, shown in the next listing, is straightforward.
%
From low to high the array's elements are pushed to the growing heap.
%
We used \inl{i < n} as loop condition, rather than the more tempting
\inl{i <= n}, in order to admit also \inl{n == SIZE_TYPE_MAX};
as a consequence, we had to call \specref{pushheap} with \inl{i+1}.
%
The iteration starts at \inl{i+1 == 2}, because an array with length one is
a heap already.

\input{Listings/make_heap.c.tex}

Since the loop statement consists just of a call to \specref{pushheap}
we obtain the both loop invariants \inl{heap} and \inl{reorder} by simply 
lifting them from the contract of \pushheap.

The postcondition of \pushheap only specifies the multiset
of elements from index 0 to \inl{i}.
%
We therefore also have to specify
that the elements from index \inl{i+1} to \inl{n-1}  are only reordered.
%
This property can be derived from the \inl{unchanged} property of \pushheap.

\clearpage


\section{The \sortheap algorithm}
\Label{sec:sortheap}

Whereas in the \cxx Standard Library \cite[\S 28.7.7.4]{cxx-17-draft}
\sortheap works on a range of random access iterators,
our version operates on an array of \valuetype.
We therefore use the following signature for \sortheap

\begin{lstlisting}[style = acsl-block]

    void sort_heap(value_type* a, size_type n);
\end{lstlisting}

The function \sortheap rearranges the elements of a given heap \inl{a[0..n-1]}
in increasing order.
Thus, applying \sortheap to the heap in Figure~\ref{fig:heap-array}
produces the increasing array in Figure~\ref{fig:sortheap-array-post}.

\begin{figure}[hbt]
\centering
\includegraphics[width=0.65\linewidth]{Figures/sort_heap_array_post.pdf}
\caption{\Label{fig:sortheap-array-post}Array after the call of \sortheap}
\end{figure}

\FloatBarrier

\subsection{Formal specification of \sortheap}

The following listing shows our specification of \sortheap.
The formal specification of \sortheap must ensure that the
resulting array is increasing.
Furthermore the multiset contained by the array must be the same
as in the pre-state of the function.
The postconditions \inl{increasing} and \inl{reorder} express these properties, respectively.
The specification effort is relatively simple because we can reuse

\input{Listings/sort_heap.h.tex}

\clearpage

\subsection{Implementation of \sortheap}

The implementation of \sortheap is relatively simple because it relies on
\specref{popheap} performing essential work.
Our implementation of \sortheap repeatedly calls \popheap
to extract the maximum of the shrinking heap and adding it
to the part of the array that is already in increasing order.
The loop invariants of \sortheap describe the content of the array
in two parts.
The first \inl{i} elements form a heap and are described by the \inl{heap}
invariant.
The last \inl{n-i} elements are already arranged in increasing order.

As already mentioned in the introduction of Chapter~\ref{cha:binary-search},
we use the predicate \logicref{WeaklyIncreasing} for the loop annotation \inl{increasing}.
Thus, after leaving the loop we have in fact ``only'' shown that \inl{WeaklyIncreasing(a, n)}
holds.
In order to derive from this fact the final assertion \inl{increasing} that uses
the predicate \logicref{Increasing} we rely on lemma \logicref{WeaklyIncreasingIncreasing}.

\input{Listings/sort_heap.c.tex}

To verify the property \inl{reorder} we rely on the lemmas \logicref{MultisetReorder}
that express that the properties 

\begin{itemize}
\item \inl{MultisetReorder\{K,L\}(a, 0, i)} and
\item \inl{Unchanged\{Old,Here\}(a, i, n)}
\end{itemize}

imply the desired loop invariant \inl{MultisetReorder\{K,L\}(a, 0, n)}.

%\clearpage



