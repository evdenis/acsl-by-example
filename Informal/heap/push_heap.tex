
\section{The \pushheap algorithm}
\Label{sec:pushheap}

The \pushheap algorithm assumes that the first $n-1$ elements of an array
of length~$n$ form already a heap and adds to it the element \inl{a[n-1]}.

Whereas in the \cxx Standard Library \cite[\S 28.7.7.1]{cxx-17-draft}
\pushheap works on a range of random access iterators,
our version operates on an array of \valuetype.
We therefore use the following signature for \pushheap

\begin{lstlisting}[style = acsl-block]

    void push_heap(value_type* a, size_type n);
\end{lstlisting}


The \pushheap algorithm expects that \inl{n} is greater or equal than~1.
It also assumes that the array\\
\inl{a[0..n-2]} forms a heap.
The algorithms then \emph{rearranges} the array \inl{a[0..n-1]} such that the 
resulting array is a heap.
In this sense the algorithm \emph{pushes} the element \inl{a[n-1]} on the given heap.

\subsection{Formal specification of \pushheap}

The following listing shows our specification of \pushheap.
Note that the post condition \inl{reorder} states
that \pushheap is not allowed to change the number of occurrences
of an array element.
Without this post condition,
an implementation that assigns \inl{0} to each array element
would satisfy the post condition \inl{heap}---surely
not what a user of the algorithm has in mind.

\input{Listings/push_heap.h.tex}

Pushing an element on a heap usually \emph{rearranges} several elements of the array
(cf.\xspace Figures~\ref{fig:pushheap-pre} and~\ref{fig:pushheap-post}).
We therefore must be able express that \pushheap only reorders
the elements of the array.
We re-use the predicate \logicref{MultisetReorder} to formally describe this property.


\subsection{Implementation of \pushheap}

The following two figures illustrate how \pushheap affects an array,
which is shown as a tree with blue and grey nodes, representing
heap and non-heap nodes, respectively.
Figure~\ref{fig:pushheap-pre} shows the heap from
Figure~\ref{fig:heap-tree} together with the additional element~\inl{12} that
is to be pushed on the heap.
To be quite clear about it: the new element~\inl{12} is the last element of the
array and not yet part of the heap.

\begin{figure}[hbt]
\centering
\includegraphics[width=0.70\linewidth]{Figures/push_heap_pre.pdf}
\caption{\Label{fig:pushheap-pre}Example heap before the call of \pushheap}
\end{figure}

\FloatBarrier

Figure~\ref{fig:pushheap-post} shows the array
after the call of \pushheap.
We can see that now all nodes are colored in blue, i.e., they are part of the heap.
The dashed nodes highlight which heap nodes have changed during the function call.
The element to be pushed into the heap is now at its correct position.
The arrows indicate the \emph{cyclic reordering} of array elements to achieve the
desired result.

\begin{figure}[hbt]
\centering
\includegraphics[width=0.70\linewidth]{Figures/push_heap_post.pdf}
\caption{\Label{fig:pushheap-post}Example heap after the call of \pushheap}
\end{figure}

\FloatBarrier


Verifying our implementation of \implref{pushheap} is a non-trivial undertaking.
In order to better structure our discussion we refer to the central
loop of the algorithm as the \emph{main act} and the parts before and after
it as \emph{prologue} and \emph{epilogue}.

We can establish the \inl{heap} property of \specref{pushheap}
already in the prologue.
The \inl{reorder} property, however, only holds at the function boundaries
and is violated while \pushheap manipulates the array.
To be more precise:
We loose the \inl{reorder} property in the prologue and
formally capture and maintain a slightly more general property
in the main act.
From this we will recover the \inl{reorder} property in the epilogue.

We will illustrate the changes to the underlying array after each stage 
by figures of the array in tree form,
based on the \pushheap example from Figure~\ref{fig:pushheap-pre}.

\input{Listings/push_heap.c.tex}

\subsubsection{Prologue}
\Label{sec:push-heap-prologue}

In the prologue we check whether the initial heap is nonempty,
initialize some variables,
\emph{and} also check by comparing with the parent node
whether \inl{a[c]}, which is the value to be pushed on the heap and which
is the last element of the array, is by chance already at the right place.
If not we set aside this value in the variable \inl{v} and
assign the parent value \inl{a[p]} to \inl{a[c]}.
Note that this assignment only occurs if the respective values
differ. 
This allows us to formally describe the effect of the assignment using the predicate
\logicref{ArrayUpdate}.
%
Figure~\ref{fig:pushheap-prologue} highlights the main effects of the prologue.
Here and in the following figures we highlight the currently active node.

\begin{figure}[hbt]
\centering
\includegraphics[width=0.70\linewidth]{Figures/push_heap_prologue.pdf}
\caption{\Label{fig:pushheap-prologue}Heap after the prologue of \pushheap}
\end{figure}

At this point we have achieved several things.
\begin{enumerate}
\item The array \inl{a[0..n-1]} is now a heap.
\item Regarding their respective number of occurences in the array \inl{a[0..n-1]}
\begin{itemize}
\item the original value \inl{a[c]} occurs one time less
\item the original value \inl{a[p]} occurs one time more
\item whereas all other values have not changed their number of occurences.
\end{itemize}
\end{enumerate}

The first observation is expressed in the assertion \inl{heap}
whereas the small fluctuation of array elements described in
the second observation is expressed by using the predicate \logicref{MultisetParity}
in the assertion \inl{reorder}.


\subsubsection{Main act}
\Label{sec:push-heap-main-act}

In the main act, we start at the parent location, which is now stored
in the variable \inl{c} (\emph{child}).
Compared to the pre-state of \pushheap
at the beginning of the main act the array \inl{a[0..n-1]} 
\begin{itemize}
\item contains the value \inl{v} one time less
\item contains the value \inl{a[c]} one time more
\item whereas all other values have not changed their number of occurences.
\end{itemize}

Now, as long as the index~\inl{c} is not yet the root of the heap
and its consequently existing parent value \inl{a[p]} is less than \inl{v},
we haven't found yet an index \inl{c} where we could insert \inl{v} without
violating the heap property.

In the loop body we proceed as follows.

\begin{itemize}
\item
If \inl{a[c]} is less than \inl{a[p]} we copy the latter value on the former.
Note that this assignment preserves the heap property of the array.
The value \inl{a[p]} now occurs one time more than in the pre-state whereas
the now overwritten value \inl{a[c]} occurs as often as in the pre-state.
The value \inl{v} continues to occur one time less.
%
We then proceed to the next iteration by setting \inl{c} to \inl{p}.

The verification of tracking the number of occurences happens
in smaller steps than just described.
It relies on the predicates \logicref{ArrayUpdate} and \logicref{MultisetUpdate}
which we can apply in this guarded assignment.
Lemma \logicref{MultisetParityCombined} also plays an important role here.

\item
Otherwise, since \inl{c} is a child of \inl{p},  we can conclude that
\inl{a[c]} equals \inl{a[p]} and we continue with the next iteration
after setting \inl{c} to \inl{p}.
\end{itemize}

This means that at the begin of the next iteration again
the following conditions hold.
Compared to the pre-state of \pushheap the array \inl{a[0..n-1]} 

\begin{itemize}
\item contains the value \inl{v} one time less
\item contains the value \inl{a[c]} one time more
\item whereas all other values have not changed their number of occurences.
\end{itemize}

Figure~\ref{fig:pushheap-main-act} shows the our example heap after the main act.
For this particular heap, only one iteration is performed until a node is reached
whose parent value \inl{a[p]} is greater or equal than \inl{v}.
Note that assignments which have previously occurred are marked with dashed arrows. 

\begin{figure}[hbt]
\centering
\includegraphics[width=0.70\linewidth]{Figures/push_heap_main_act.pdf}
\caption{\Label{fig:pushheap-main-act}Heap after the main act of \pushheap}
\end{figure}

\FloatBarrier
\clearpage

\subsubsection{Epilogue}
\Label{sec:push-heap-epilogue}

At this point, we have arrived at an index \inl{c} where 
the assignment of the value \inl{v} preserves the heap property.
We express this formally using the predicate \logicref{HeapCompatible}.

Moreover, this assignment also corrects the imbalance in the number of
occurences of the values \inl{a[c]} and \inl{v} and consequently
establishes the desired property \inl{reorder} of \pushheap.
The verification that this correction leads to a proper reordering relies
on lemma \logicref{MultisetParityMultisetReorder}.

Figure~\ref{fig:pushheap-epilogue} shows the final assignment and highlights the
completion of the cycle depicted in Figure~\ref{fig:pushheap-prologue}.
The figure also makes clear that the value \inl{v} acts like an additional
element in this assignment cycle.

\begin{figure}[hbt]
\centering
\includegraphics[width=0.70\linewidth]{Figures/push_heap_epilogue.pdf}
\caption{\Label{fig:pushheap-epilogue}Heap after the epilogue of \pushheap}
\end{figure}

\clearpage

