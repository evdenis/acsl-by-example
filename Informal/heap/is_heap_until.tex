
\section{The \isheapuntil algorithm}
\Label{sec:isheapuntil}

The \isheapuntil algorithm of the \cxx Standard Library \cite[\S
28.7.7.5]{cxx-17-draft} works on generic sequences. 
For our purposes we have modified the generic implementation
to that of an array of type \valuetype.
The signature now reads:

\begin{lstlisting}[style = acsl-block]

    size_type is_heap_until(const value_type* a, int n);
\end{lstlisting}

The algorithm \isheapuntil returns the largest range of an array, beginning at the first position, where it still satisfies the heap properties
we have semi-formally described in the beginning of this chapter.
In particular, \isheapuntil will return the size of the array,
called with the array argument from Figure~\ref{fig:heap-array}.

\clearpage

\subsection{Formal specification of \isheapuntil}

The specification of \isheapuntil is shown in the following listing.
The index \inl{\\result} returned by \isheapuntil indicates
that the array \inl{a[0..\\result-1]} is a heap.
In addition the postcondition \inl{last} states, that for all indices
greater than or equal to \inl{i} the predicate \logicref{Heap} is not satisfied.

\input{Listings/is_heap_until.h.tex}

\subsection{Implementation of \isheapuntil}

The following listing shows one way to implement the function \isheapuntil.

\input{Listings/is_heap_until.c.tex}

The algorithms starts at the index~1, which is the smallest index,
where a child node of the heap might reside.
The algorithms checks for each (child) index whether
the value at the corresponding parent index 
is greater than or equal to the value at the child index.
If the value at a parent index is smaller than the value at a (child) index,
\isheapuntil returns the (child) index.
Otherwise, if the algorithm iterates through the whole array,
the size of the array is returned.
