
\section{The \makeheap algorithm}
\Label{sec:makeheap}


Whereas in the \cxx Standard Library \cite[\S 28.7.7.3]{cxx-17-draft}
\makeheap works on a pair of generic random access
iterators,
our version operators on a range of \valuetype.
Thus the signature of \makeheap reads

\begin{lstlisting}[style = acsl-block]

    void make_heap(value_type* a, size_type n);
\end{lstlisting}

The function \makeheap rearranges the elements of the given
array \inl{a[0..n-1]} such that they form a heap.

As an examples we look at the array in Figure~\ref{fig:makeheap-array-pre}.
The elements of this array do not form a heap, as indicated by the grey colouring.
Executing the \makeheap algorithm on this array rearranges its elements
so that they form a heap as shown in Figure~\ref{fig:heap-array}.

\begin{figure}[hbt]
\centering
\includegraphics[width=0.65\linewidth]{Figures/make_heap_array_pre.pdf}
\caption{\Label{fig:makeheap-array-pre}Array before the call of \makeheap}
\end{figure}


\FloatBarrier

\subsection{Formal specification of \makeheap}

The following listing shows the specification of \makeheap.

\input{Listings/make_heap.h.tex}

Like with \pushheap the formal specification of \makeheap
must ensure that the resulting array is a heap of size \inl{n}
and contains the same multiset of elements as in the pre-state of the function.
These properties are expressed by the \inl{heap} and \inl{reorder}
postconditions respectively.
The \inl{reorder} postcondition uses the predicate \logicref{MultisetReorder}
to ensure that \makeheap only rearranges the array elements.

\clearpage

\subsection{Implementation of \makeheap}

The implementation of \makeheap, shown in the next listing, is straightforward.
%
From low to high the array's elements are pushed to the growing heap.
%
We used \inl{i < n} as loop condition, rather than the more tempting
\inl{i <= n}, in order to admit also \inl{n == SIZE_TYPE_MAX};
as a consequence, we had to call \specref{pushheap} with \inl{i+1}.
%
The iteration starts at \inl{i+1 == 2}, because an array with length one is
a heap already.

\input{Listings/make_heap.c.tex}

Since the loop statement consists just of a call to \specref{pushheap}
we obtain the both loop invariants \inl{heap} and \inl{reorder} by simply 
lifting them from the contract of \pushheap.

The postcondition of \pushheap only specifies the multiset
of elements from index 0 to \inl{i}.
%
We therefore also have to specify
that the elements from index \inl{i+1} to \inl{n-1}  are only reordered.
%
This property can be derived from the \inl{unchanged} property of \pushheap.

\clearpage

