\section{The \popheap algorithm}
\Label{sec:popheap}

The algorithm \popheap moves the first element of the heap, which holds
the heap's largest value, and places it at the the end of the underlying sequence.
Whereas in the \cxx Standard Library \cite[\S 28.7.7.2]{cxx-17-draft}
\popheap works on a range of random access iterators,
our version operates on an array of \valuetype.
We therefore use the following signature for \popheap

\begin{lstlisting}[style = acsl-block]

    void pop_heap(value_type* a, size_type n);
\end{lstlisting}

The \popheap algorithm expects that \inl{n} is greater or equal than~1
and that the array \inl{a[0..n-1]} forms a heap.
The algorithms then \emph{rearranges} the array \inl{a[0..n-1]} such that the
resulting array satisfies the following properties.

\begin{itemize}
\item \inl{a[n-1] = \\old(a[0])}, that is, the largest element
of the original heap is transferred to the end of the array.

\item the subarray \inl{a[0..n-2]} is a heap
\end{itemize}

In this sense the algorithm \emph{pops} the largest element from a heap.

\subsection{Formal specification of \popheap}

Based on the above semi-formal description we propose the
following function contract for \specref{popheap}.

\input{Listings/pop_heap.h.tex}

\subsection{Implementation of \popheap}
\Label{sec:pop-heap:impl}

In an abstract sense \popheap is quite similar to \pushheap.
In  \pushheap we started at the last array element and 
climbed from there up the tree until we would find a node where to
insert the new value into the heap.
Every time we had reached the next parent node we
moved its value down to where we had just come from.

With \popheap its the other way round.
We start at the root of the tree and descend from there
by selecting an appropriate child.
Every time we lift the value of the selected child to the node where
just are.
We repeat this process until we find a node where we can insert
the last array element into the heap.
Once this is done, we can safely place the maximum element (that is the
the original root node) at the last element of the array.

\clearpage

The following two figures illustrate how \popheap affects an array,
which is shown again as a tree with blue and grey nodes, representing
heap and non-heap nodes, respectively.
Figure~\ref{fig:popheap-pre} is in fact the same figure as
Figure~\ref{fig:heap-tree}.

\begin{figure}[hbt]
\centering
\includegraphics[width=0.70\linewidth]{Figures/pop_heap_pre.pdf}
\caption{\Label{fig:popheap-pre}Heap before the call of \popheap}
\end{figure}

\FloatBarrier

Figure~\ref{fig:popheap-post}, on the other hand, shows the heap after the call of \popheap
together with arrows that indicate how our implementation moves around elements
in the underlying array.
We can see that the first element of the original array,
where the maximum of the heap resides, is now the last element of the array.
Furthermore, the last array element is not part of the heap anymore.
The dashed nodes highlight which heap nodes have changed during the call to \popheap.
The arrows indicate the \emph{cyclic reordering} of array elements to achieve the
desired result.

\begin{figure}[hbt]
\centering
\includegraphics[width=0.70\linewidth]{Figures/pop_heap_post.pdf}
\caption{\Label{fig:popheap-post}Heap after the call of \popheap}
\end{figure}

\FloatBarrier


As in the case of \implref{pushheap} we will subdivide the discussion of the 
implementation of \implref{popheap} into a prologue, main act, and epilogue.

\input{Listings/pop_heap.c.tex}


\subsection{Prologue}

In the prologue we check whether the initial heap contains at least two elements,
initialize some variables, and also check whether
the last array element is by chance equal to the maximum element of the heap,
which resides at the index \inl{p == 0} of the array.
If this is not the case, then we set aside for future
reference the last array element in the variable \inl{v}.
Finally we copy the value \inl{a[p]} to its final destination at the end
of the array.
Note that this assignment only occurs if the respective values differ.
This allows us, as in the case of \implref{pushheap}, to formally describe the
effect of the assignment using the predicate \logicref{ArrayUpdate}.

Figure~\ref{fig:popheap-prologue} highlights the main effects of the prologue
at the hand of our exemplary heap.
Note that we have highlighted the root of the heap as the currently active node.

\begin{figure}[hbt]
\centering
\includegraphics[width=0.65\linewidth]{Figures/pop_heap_prologue.pdf}
\caption{\Label{fig:popheap-prologue}Heap after the prologue of \popheap}
\end{figure}

\FloatBarrier


\subsection{Main act}

In the main act, we start at a child node \inl{c} of the prologue's index \inl{p}.
This means that compared to the pre-state of \popheap
at the beginning of the main act the array \inl{a[0..n-1]}
\begin{itemize}
\item contains the value \inl{v} one time less
\item contains the value \inl{a[p]} one time more
\item whereas all other values have not change their number of occurences.
\end{itemize}

Moreover, the maximum element of the original heap is now at the end of the array
and we can only guarantee that the first $n-1$ array elements got a heap.
These observations are necessary reason for our loop invariants.

To be more precise, when we talk in the context of \popheap
of a \emph{child node} we usually mean one of the possibly two children
where the maximum of the values resides.
We do this because copying that larger value to its parent node guarantees
that the resulting tree is still a heap.
We compute the maximum child of a node using the function \specref{heapchild}.

Now, as long as the index~\inl{c} is not yet the index of the last array element
of the heap and its value \inl{a[c]} is less than \inl{v},
we haven't found yet an index where we could insert \inl{v} without
violating the heap property.

\clearpage

In the loop body we proceed as follows.

\begin{itemize}
\item
If \inl{a[c]} is less than \inl{a[p]} we copy the former value on 
the latter.
As mentioned above, using the index~\inl{c} of the maximum child
maintains  heap property of the array.
We use here the predicate \logicref{HeapCompatible} to express
that the insertion of the new value \inl{a[p]} maintains the heap
property of the array.

The value \inl{a[c]} now occurs one time more than in the pre-state whereas
the now overwritten value \inl{a[p]} occurs as often as in the pre-state of \popheap.
The value \inl{v} continues to occur one time less than in the pre-state.
%
We then proceed to the next iteration by setting \inl{p} to \inl{c}
and computing the next maximum child node.

As in the case of \implref{pushheap} the verification of 
the correct number of occurences of the involved values
relies on the predicates \logicref{ArrayUpdate} and \logicref{MultisetUpdate}
and on lemma \logicref{MultisetParityCombined}.

\item
Otherwise, the array being a heap, we can conclude that
\inl{a[c]} equals \inl{a[p]} and we continue with the next iteration
after setting \inl{p} to \inl{c} and computing the corresponding
new maximum child node.
\end{itemize}

The following three figures depict how the main act of \popheap 
modifies step by step our example heap.
In each step we highlight the currently active node \inl{c}.


\begin{figure}[hbt]
\centering
\includegraphics[width=0.65\linewidth]{Figures/pop_heap_main_act1.pdf}
\caption{\Label{fig:popheap-main_act1}Heap after the first iteration of of \popheap}
\end{figure}

\FloatBarrier

\begin{figure}[hbt]
\centering
\includegraphics[width=0.65\linewidth]{Figures/pop_heap_main_act2.pdf}
\caption{\Label{fig:popheap-main_act2}Heap after the second iteration of \popheap}
\end{figure}

\FloatBarrier
\clearpage

Note that in the final step no value is actually copied as the involved nodes
hold the same value.

\begin{figure}[hbt]
\centering
\includegraphics[width=0.65\linewidth]{Figures/pop_heap_main_act3.pdf}
\caption{\Label{fig:popheap-main_act3}Heap after the third iteration of \popheap}
\end{figure}

\FloatBarrier

We finally remark that in the main act the the last array element is never modified.
Thus, the root element of the original element is still safely stored there.

\subsection{Epilogue}

After leaving the loop, we know that value \inl{v} can be the inserted
in the array at the index \inl{p} without violating the heap property of
the first $n-1$ elements.
%
Moreover, compared to the pre-state of \popheap the array \inl{a[0..n-1]} still
\begin{itemize}
\item contains the value \inl{v} one time less
\item contains the value \inl{a[p]} one time more
\item whereas all other values have not change their number of occurences.
\end{itemize}

In other words, assigning the value \inl{v} to \inl{a[p]}
cancels this imbalance and establishes that \popheap
only reorders the array elements.

\begin{figure}[hbt]
\centering
\includegraphics[width=0.65\linewidth]{Figures/pop_heap_epilogue.pdf}
\caption{\Label{fig:popheap-epilogue}Heap after the epilogue of \popheap}
\end{figure}

\FloatBarrier

In Figure~\ref{fig:popheap-epilogue} we have marked the value \inl{v}
as the currently active node despite not being an array element.

\clearpage

