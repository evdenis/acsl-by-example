
\section{The \minmaxelement algorithm}
\Label{sec:minmaxelement}

The \minmaxelement algorithm in the
\cxx Standard Library \cite[\S 28.7.8]{cxx-17-draft} searches
\emph{both} the minimum \emph{and} the maximum in a sequence.
The signature of our version of \minelement reads:

\begin{lstlisting}[style = acsl-block]

        size_type_pair minmax_element(const value_type* a, size_type n);
\end{lstlisting}

Note that \minmaxelement returns a \emph{pair} of indices (see \S\ref{sec:makepair}).
This pair contains the \emph{first} position where the minimum occurs in
the sequence \inl{a[0..n-1]} and the \emph{last} position where maximum occurs.

The properties of the index for the minimum value are the same as the properties
of \specref{minelement}.
However, the properties of the index that marks the maximum
element, are slightly different from the properties of \specref{maxelement}.
The \maxelement algorithm returns the position of the \emph{first} occurrence of
the maximum element if it occurs multiple times in the sequence. The
\minmaxelement algorithm returns the position of the last occurrence of the
maximum element.

%\clearpage

\subsection{Formal specification of \minmaxelement}

The following listing shows the acsl specification of \specref{minmaxelement}.
Note that we use the predicates \logicref{StrictLowerBound} and
\logicref{StrictUpperBound} in order to express that the algorithm returns
the positions of both the \emph{first minimum} and the \emph{last maximum}.
We also use the predicates \logicref{MinElement} and \logicref{MaxElement}.
Thus reflects of course the use of this predicates for the algorithms
\specref{minelement} and \specref{maxelement}.

\input{Listings/minmax_element.h.tex}


The specification is similar to the specifications of \minelement and
\maxelement. The only difference lies in the postcondition \inl{last}. Here the
postcondition states that after the position of the maximum element there is no
value greater or equal the maximum element. This differs from the specification
of \maxelement, where the first occurrence of the maximum value has to be
returned.

%\clearpage

\subsection{Implementation of \minmaxelement}

The implementation of \implref{minmaxelement} uses the auxiliary
function \specref{makepair} to construct a pair of indices.
We will focus on the loop invariant \inl{last}, because it is the
only loop invariant that differs from the implementations of \implref{minelement} and
\implref{maxelement}.

\input{Listings/minmax_element.c.tex}

As already mentioned we had to alter the range for the predicate
\logicref{StrictUpperBound} to fit into the property of returning
the last maximum position that occurred.

