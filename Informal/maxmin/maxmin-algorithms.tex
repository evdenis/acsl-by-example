
\chapter{Maximum and minimum algorithms}
\Label{cha:maxmin}

In this chapter we discuss the formal specification of algorithms
in the \cxx Standard Library \cite[\S 28.7.8]{cxx-17-draft}
that compute the maximum or minimum values of their arguments.
As the algorithms in Chapter~\ref{cha:non-mutating}, they also do not modify
any memory locations outside their scope.
The most important new feature of the algorithms in this chapter
is that they compare values using binary operators such as \inl{<}.

We consider in this chapter the following algorithms.

\begin{itemize}

\item
We discuss some properties of relations operators in \S\ref{sec:relationaloperators}.

\item
We introduce in \S\ref{sec:arraybounds} various predicates
that describe basic order properties for arrays whose elements are of~\valuetype.

\item
\clamp, which is discussed in \S\ref{sec:clamp},
is a very simple algorithms that ``clamps'' (or ``clips'') a value between a
pair of boundary values.
  
\item \maxelement  
returns an index to
a maximum element in a range. Similar to \find it also
returns the smallest of all possible indices.
This algorithm is discussed in \S\ref{sec:maxelement}.
In \S\ref{sec:maxelementii}, we introduce an alternative
specification \maxelementii which relies on user-defined predicates.

\item \maxseq  in \S\ref{sec:maxseq}
is very similar to \maxelement and will serve as an
example of \emph{modular verification}.
It returns the maximum value itself rather than an index to
it.

\item \minelement in \S\ref{sec:minelement}
can be used to find the smallest element in an array.

\item \minmaxelement in \S\ref{sec:minmaxelement}
is used to find simultaneously the smallest and largest element in a given range.
This algorithms relies on the auxiliary function
\makepair (\S\ref{sec:makepair}).
\end{itemize}

First, however, we discuss in \S\ref{sec:relationaloperators}
general properties that must be satisfied by the relational operators.

%\clearpage

\section{A note on relational operators}
\Label{sec:relationaloperators}

Note that in order to compare values, algorithms in the
\cxx Standard Library \cite[\S 28.7.8]{cxx-17-draft} usually rely solely on 
the \emph{less than} operator~\inl{<} or special function
objects.
To be precise, the operator~\inl{<} must be a \emph{partial
order},\footnote{
    See~\url{http://en.wikipedia.org/wiki/Partially_ordered_set}
}
which means that the following rules must hold.
%
\begin{alignat*}{5}
&\text{irreflexivity} &&\qquad  \forall x     &&: \neg(x < x)         \\
&\text{asymmetry}     &&\qquad  \forall x,y   &&: x < y             &&\implies \neg(y < x)\\
&\text{transitivity}  &&\qquad  \forall x,y,z &&: x < y \wedge y < z &&\implies x < z
\end{alignat*}


If you wish to check that the operator~\inl{<} of our \valuetype\footnote{
    See \S\ref{sec:frequentPattern}
}  satisfies these properties
one can formulate the lemmas of \logicref{Less} and verify them with \framac.

\input{Listings/LessThanComparable.acsl.tex}

It is of course possible to specify and implement the algorithms
of this chapter by only using operator~\inl{<}.
For example, \inl{a <= b} can be written as \texttt{a < b || a == b},
or, for our particular ordering on \valuetype, as \inl{!(b < a)}.
Listing~\logicref{Less} therefor also contains lemmas on representing
the operator~\inl{>}, \inl{<=}, and~\inl{>=} through operator \inl{<}.



\section{Predicates for bounds and extrema of arrays}
\label{sec:arraybounds}

We define in the following listing the predicates \logicref{MaxElement}
and \logicref{MinElement} that we will use for the specification of various algorithms.
We will discuss these predicates in more detail in \S\ref{sec:maxelementii}
and \S\ref{sec:minelement}.

\input{Listings/ArrayExtrema.acsl.tex}


\clearpage

The aforementioned predicates rely on the predicates
\logicref{LowerBound} and \logicref{UpperBound} which
are shown in the following listing together with 
the related predicates \logicref{StrictUpperBound} and \logicref{StrictLowerBound}.

\input{Listings/ArrayBounds.acsl.tex}

These predicates concisely express the
comparison of the elements in an array (segment) with a given value.
We will heavily rely on these predicates both in this chapter and
in Chapter~\ref{cha:binary-search}.

\clearpage


\section{The \clamp algorithm}
\label{sec:clamp}


The \clamp algorithm in the \cxx Standard Library \cite[\S 28.7.9]{cxx-17-draft} ``clamps'' a value
between a pair of boundary values.
The signature of our version of \clamp reads:

\begin{lstlisting}[style = acsl-block]

  value_type clamp(value_type v, value_type lower, value_type upper);
\end{lstlisting}

The function \clamp returns \inl{v} if the value is greater
than \inl{lower} and smaller than \inl{upper}.
Otherwise, if \inl{v} is smaller than \inl{lower}, then \inl{lower} is returned.
Finally, if \inl{v} is greater than \inl{upper}, \inl{upper} is the returned.

\subsection{Formal specification of \clamp}

The following listing contains the specification of \specref{clamp}.
Note that we require that \inl{lower} must be less or equal than \inl{upper}.

\input{Listings/clamp.h.tex}

\clearpage

\subsection{Implementation of \clamp}

The implementation of \implref{clamp} can be verified without any additional annotations.

\input{Listings/clamp.c.tex}

%\clearpage



\section{The auxiliary function \makepair}
\label{sec:makepair}

In order to be able to specify functions that work
on pairs of indices we introduce in the following listing
the type \sizetypepair.

\begin{listing}[hbt]
\begin{minipage}{0.99\textwidth}
\lstinputlisting[style=acsl-block, frame=single]{Source/size_type_pair.h}
\end{minipage}
  \caption{\Label{lst:size_type_pair}The type \sizetypepair}
\end{listing}

\index[examples]{size\_pair\_type@\texttt{size\_pair\_type}}

%\clearpage

We will also use the auxiliary function \makepair which turns
two indices \inl{first} and \inl{second} into an object of \sizetypepair.
The specification and implementation of \specref{makepair} is shown here.

\input{Listings/make_pair.h.tex}

\clearpage



\section{The \maxelement algorithm}
\Label{sec:maxelement}

The \maxelement algorithm in the \cxx Standard Library \cite[\S 28.7.8]{cxx-17-draft}
searches the maximum of a general sequence. 
The signature of our version of \maxelement reads:

\begin{lstlisting}[style = acsl-block]

  size_type max_element(const value_type* a, size_type n);
\end{lstlisting}

The function finds the largest element in the range
\inl{a[0..n-1]}.
More precisely, it returns the unique valid index \inl{i} such that:
\begin{enumerate}
\item for each index \inl{k} with \inl{0 <= k < n} the condition
\inl{a[k] <= a[i]} holds and
\item for each index \inl{k} with \inl{0 <= k < i} the condition
\inl{a[k] < a[i]} holds.
\end{enumerate}
The return value of \maxelement is \inl{n} if and only if there is no
maximum, which can only occur if \inl{n == 0}.

\subsection{Formal specification of \maxelement}

The following listings shows the formal specification of \specref{maxelement}.
Note that we have subdivided the specification of \maxelement into the two
behaviors~\inl{empty} and \inl{not_empty}.
The behavior \inl{empty} contains the specification for the case that the
range contains no elements.
The behavior \inl{not_empty} applies if the range has a positive length.

The ensures clause \inl{max} of behavior \inl{not_empty} indicates that
the returned valid index \inl{k} refers to a maximum value of the array.
The postcondition \inl{first} expresses that \inl{k} is indeed the
\emph{first} occurrence of a maximum value in the array.

\input{Listings/max_element.h.tex}

\subsection{Implementation of \maxelement}

In our description, we concentrate on the \emph{loop annotations}
of the implementation of \implref{maxelement}.

\input{Listings/max_element.c.tex}

The loop invariant \inl{max} is needed to prove the  postcondition
\inl{result} of the behavior \inl{not_empty} of \specref{maxelement}.
Using loop invariant \inl{upper} we prove the postcondition \inl{upper}
of the behavior \inl{not_empty} of \specref{maxelement}.
Finally, the postcondition \inl{first} of this behavior can be
verified with the loop invariant \inl{first}.

\clearpage



\section{The \maxelement algorithm with predicates}
\Label{sec:maxelementii}

In this section we present another specification of the \maxelement algorithm.
The main difference is that we employ the predicate \logicref{UpperBound}
which basically expresses that a given value is greater or equal than all
elements of a given array.
Closely related to the predicate \UpperBound is the predicate \logicref{StrictUpperBound}.

We also employ the predicate \logicref{MaxElement}.
This predicate states that the element at a given index \inl{max} is an 
\emph{upper bound} of the sequence \inl{a[0..n-1]}, and, by
construction, a member of that sequence.

\subsection{Formal specification of \maxelementii}

The formal specification of \specref{maxelementii} is shown in the following listing.
Note that we also use the predicate  \logicref{StrictUpperBound}
in order to express that \maxelementii returns the \emph{first} maximum position in \inl{a[0..n-1]}.

\input{Listings/max_element2.h.tex}

\clearpage

\subsection{Implementation of \maxelementii}

The implementation of \implref{maxelementii} is of course
very similar to that of \implref{maxelement}---except that the
loop invariants now also use the above mentioned predicates.

\input{Listings/max_element2.c.tex}

\clearpage



\section{The \maxseq algorithm}
\Label{sec:maxseq}

In this section we consider the function \maxseq \cite[Ch.~3]{ACSLTutorial})
which is very similar to the function \specref{maxelement}.
The main difference between \maxseq and \maxelement is that
\maxseq returns the maximum value (not just the index of it).
Therefore, it requires a \emph{non-empty} range as an argument.

Of course, \maxseq can easily be implemented using \implref{maxelementii}.
Moreover, relying only on the formal specification of \specref{maxelementii},
we are also able to deductively verify the correctness of this implementation.
Thus, we have a simple example of \emph{modular verification} in the following sense:
\begin{quote}
Any implementation of \maxelementii that is separately
 proven to implement the contract \specref{maxelementii}
 makes \maxseq behave correctly.
 Once the contracts have been defined, the function \maxelementii
 could be implemented in parallel, or just after \maxseq,
 without affecting the verification of \maxseq.
\end{quote}


\subsection{Formal specification of \maxseq}

The following listing shows the formal specification of \specref{maxseq}.

\input{Listings/max_seq.h.tex}

Using the first \inl{requires}-clause we express that \maxseq needs a \emph{non-empty} range as input.
Our postconditions
formalize that \maxseq indeed returns the maximum value of the range.

%\clearpage

\subsection{Implementation of \maxseq}

The implementation of \implref{maxseq} consists of a simple call to \implref{maxelementii}.
Since \maxseq requires a non-empty range the call of \maxelementii
returns an index to a maximum value in the range.
The fact that \maxelementii returns the smallest index is of no importance
in this context.

\input{Listings/max_seq.c.tex}

\clearpage



\section{The \minelement algorithm}
\Label{sec:minelement}

The \minelement  algorithm in the \cxx Standard Library \cite[\S 28.7.8]{cxx-17-draft}
searches the minimum in a general sequence. 
The signature of our version of \minelement reads:

\begin{lstlisting}[style = acsl-block]

  size_type min_element(const value_type* a, size_type n);
\end{lstlisting}

The function \minelement finds the smallest element in the range \inl{a[0..n-1]}.
More precisely, it returns the unique valid index \inl{i} such that
\inl{a[i]} is minimal among the values \inl{a[0]}, \ldots,
\inl{a[n-1]}, and \inl{i} is the first position with that property.
The return value of \minelement is \inl{n} if and only if \inl{n == 0}.

We use the predicate \logicref{LowerBound} that
basically expresses that a given value is less or equal than all
elements of a given array (section).
%
Closely related to the predicate \LowerBound is the predicate \logicref{StrictLowerBound}.
%
We also use the predicate \logicref{MinElement} which states that the element
at a given index \inl{min} is a \emph{lower bound} of the sequence \inl{a[0..n-1]},
and, by construction, a member of that sequence.


\subsection{Formal specification of \minelement}


The following listing contains the specification of \specref{minelement}.
Note that we also use the predicate \logicref{StrictLowerBound} in order to
express that \minelement returns the \emph{first} minimum position in \inl{a[0..n-1]}.

\input{Listings/min_element.h.tex}

\clearpage

\subsection{Implementation of \minelement}

The implementation of \implref{minelement} uses the predicates \logicref{LowerBound}
and \logicref{StrictLowerBound} in its loop annotations.

\input{Listings/min_element.c.tex}

\clearpage



\section{The \minmaxelement algorithm}
\Label{sec:minmaxelement}

The \minmaxelement algorithm in the
\cxx Standard Library \cite[\S 28.7.8]{cxx-17-draft} searches
\emph{both} the minimum \emph{and} the maximum in a sequence.
The signature of our version of \minelement reads:

\begin{lstlisting}[style = acsl-block]

        size_type_pair minmax_element(const value_type* a, size_type n);
\end{lstlisting}

Note that \minmaxelement returns a \emph{pair} of indices (see \S\ref{sec:makepair}).
This pair contains the \emph{first} position where the minimum occurs in
the sequence \inl{a[0..n-1]} and the \emph{last} position where maximum occurs.

The properties of the index for the minimum value are the same as the properties
of \specref{minelement}.
However, the properties of the index that marks the maximum
element, are slightly different from the properties of \specref{maxelement}.
The \maxelement algorithm returns the position of the \emph{first} occurrence of
the maximum element if it occurs multiple times in the sequence. The
\minmaxelement algorithm returns the position of the last occurrence of the
maximum element.

%\clearpage

\subsection{Formal specification of \minmaxelement}

The following listing shows the acsl specification of \specref{minmaxelement}.
Note that we use the predicates \logicref{StrictLowerBound} and
\logicref{StrictUpperBound} in order to express that the algorithm returns
the positions of both the \emph{first minimum} and the \emph{last maximum}.
We also use the predicates \logicref{MinElement} and \logicref{MaxElement}.
Thus reflects of course the use of this predicates for the algorithms
\specref{minelement} and \specref{maxelement}.

\input{Listings/minmax_element.h.tex}


The specification is similar to the specifications of \minelement and
\maxelement. The only difference lies in the postcondition \inl{last}. Here the
postcondition states that after the position of the maximum element there is no
value greater or equal the maximum element. This differs from the specification
of \maxelement, where the first occurrence of the maximum value has to be
returned.

%\clearpage

\subsection{Implementation of \minmaxelement}

The implementation of \implref{minmaxelement} uses the auxiliary
function \specref{makepair} to construct a pair of indices.
We will focus on the loop invariant \inl{last}, because it is the
only loop invariant that differs from the implementations of \implref{minelement} and
\implref{maxelement}.

\input{Listings/minmax_element.c.tex}

As already mentioned we had to alter the range for the predicate
\logicref{StrictUpperBound} to fit into the property of returning
the last maximum position that occurred.



