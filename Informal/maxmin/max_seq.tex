
\section{The \maxseq algorithm}
\Label{sec:maxseq}

In this section we consider the function \maxseq \cite[Ch.~3]{ACSLTutorial})
which is very similar to the function \specref{maxelement}.
The main difference between \maxseq and \maxelement is that
\maxseq returns the maximum value (not just the index of it).
Therefore, it requires a \emph{non-empty} range as an argument.

Of course, \maxseq can easily be implemented using \implref{maxelementii}.
Moreover, relying only on the formal specification of \specref{maxelementii},
we are also able to deductively verify the correctness of this implementation.
Thus, we have a simple example of \emph{modular verification} in the following sense:
\begin{quote}
Any implementation of \maxelementii that is separately
 proven to implement the contract \specref{maxelementii}
 makes \maxseq behave correctly.
 Once the contracts have been defined, the function \maxelementii
 could be implemented in parallel, or just after \maxseq,
 without affecting the verification of \maxseq.
\end{quote}


\subsection{Formal specification of \maxseq}

The following listing shows the formal specification of \specref{maxseq}.

\input{Listings/max_seq.h.tex}

Using the first \inl{requires}-clause we express that \maxseq needs a \emph{non-empty} range as input.
Our postconditions
formalize that \maxseq indeed returns the maximum value of the range.

%\clearpage

\subsection{Implementation of \maxseq}

The implementation of \implref{maxseq} consists of a simple call to \implref{maxelementii}.
Since \maxseq requires a non-empty range the call of \maxelementii
returns an index to a maximum value in the range.
The fact that \maxelementii returns the smallest index is of no importance
in this context.

\input{Listings/max_seq.c.tex}

\clearpage

